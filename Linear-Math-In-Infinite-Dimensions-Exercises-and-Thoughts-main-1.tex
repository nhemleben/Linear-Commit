
\documentclass[12pt]{article} 
\usepackage{graphicx,relsize, verbatim, amssymb, amsmath, amsthm, mathabx,dcolumn,mathrsfs, dsfont, enumerate, soul, titlesec}

\newcolumntype{2}{D{.}{}{2.0}}
\textwidth = 7 in
\textheight = 9.5 in
\oddsidemargin = -0.3 in
\evensidemargin = -0.3 in
\topmargin = -0.4 in
\headheight = 0.0 in
\headsep = 0.0 in
\parskip = 0.2in
\parindent = 0.0in
\titlespacing*{\section}
{0pt}{0ex plus .1ex minus .2ex}{.1ex plus .1ex}
\titlespacing*{\subsection}
{0pt}{.5ex plus .1ex minus .2ex}{.3ex plus .2ex}

\DeclareMathOperator{\Gal}{Gal}	
\DeclareMathOperator{\gal}{Gal}	
\DeclareMathOperator{\im}{Image}
\DeclareMathOperator{\Ann}{Ann}	
\DeclareMathOperator{\spn}{span}	
\DeclareMathOperator{\sgn}{sgn}	


\newcommand{\CC}{\mathbb{C}}	
\newcommand{\GG}{\mathbb{G}}	
\newcommand{\LL}{\mathbb{L}}	
\newcommand{\NN}{\mathbb{N}}	
\newcommand{\PP}{\mathbb{P}}	
\newcommand{\QQ}{\mathbb{Q}}	
\newcommand{\RR}{\mathbb{R}}	
\newcommand{\ZZ}{\mathbb{Z}}
\newcommand{\one}{\mathds{1}}
\newcommand{\eps}{\epsilon}

\newcommand{\floor}[1]{\lfloor #1 \rfloor}
\newcommand{\s}[1]{\sqrt{#1}}
\newcommand{\norm}[1]{\left\lVert#1\right\rVert}
\newcommand{\I}[1]{#1^{-1}}
\newcommand{\Rw}{\Rightarrow}
\newcommand{\Lw}{\Leftarrow}
\newcommand{\lw}{\leftarrow}
\newcommand{\rw}{\rightarrow}
\newcommand{\p}{\partial}
\newcommand{\var}{\mathrm{Var}}
\newcommand{\sumi}{\sum_{i=0}}
\newcommand{\sumo}{\sum_{i=1}}
\newcommand{\num}[1]{{\large \bf  #1)}}
\newcommand{\del}{\Delta}
\newcommand{\lamb}{\lambda}	
\newcommand{\al}{\alpha}	
\newcommand{\nab}{\nabla}	
\newcommand{\pp}[2]{\frac{\p #1}{\p #2}}
\newcommand{\di}{\mathrm{d}}
\newcommand{\dd}[2]{\frac{\di #1}{\di #2}}
\newcommand{\intinf}{\int_{-\infty}^\infty}	
\newcommand{\suminf}[1]{\sum_{#1 = -\infty}^\infty}	

\newtheorem{thm}{Theorem}
\theoremstyle{definition}
\newtheorem{defi}{Definition}[section]
\newtheorem{lem}{Lemma}[section]

% \unlhd = normal subgroup symbol%
%%%%% End of preamble %%%%%
\begin{document}

\title{Linear Math In Infinte Dimensions}

{\Large Nicholas Hemleben} \hfill
{\large Linear Math In Infinte Dimensions}  
\hfill  \today

\section{Chapter 1}

\subsection{ Subsection 5}

\begin{enumerate}
\item 
\hrule
{\large 

Show that 
\vspace{-2mm}
\[
\langle f , g \rangle = \sum_{k=1}^\infty \bar c_k d_k 
\]
}
\hrule
\[
\langle f , g \rangle = \int_a^b \bar f (x) g(x) \rho(x) \di x
\]

Since we have that $f(x) = \sum_{k=1}^\infty u_k(x) c_k$ we can substitiute and get:

\[
= \int_a^b \overline { \sum_{k=1}^\infty u_k(x) c_k } g(x) \rho(x) \di x
=  \sum_{k=1}^\infty \int_a^b \bar u_k(x) \bar c_k  g(x) \rho(x) \di x
\]
\[
=  \sum_{k=1}^\infty \int_a^b \bar u_k(x) \bar c_k  g (x)\rho (x)\di x
=  \sum_{k=1}^\infty \bar c_k \int_a^b \bar u_k(x)  g (x)\rho (x)\di x
= \sum_{k=1}^\infty \bar c_k d_k 
\]

\hrule
\item
\[
T f(\omega, t) = 
\int_{-\infty}^\infty \bar g(x-t) e^{-i\omega x} f(x) \di x
\]

\[
\langle h_1, h_2 \rangle = \intinf \intinf \bar h_1(\omega,t) h_2(\omega,t) \di \omega \di t
\]
Find a formula for:
$\langle Tf_1, Tf_2 \rangle $
in terms of 
\[
\intinf \bar f_1 f_2 \di x
\]

\hrule
\[
\langle Tf_1, Tf_2 \rangle 
=
\intinf \intinf 
\overline{Tf_1} Tf_2  \di \omega \di t
=
\]

\[
\intinf \intinf 
\overline{
\left[
\int_{-\infty}^\infty \bar g(x-t) e^{-i\omega x} f_1(x) \di x
\right]
}
\left[
\int_{-\infty}^\infty \bar g(x-t) e^{-i\omega x} f_2(x) \di x
\right]
\di \omega \di t
\]

\[
=
\intinf \intinf 
\intinf \intinf 
\overline{
\bar g(x-t) e^{-i\omega x} f_1(x) 
}
\bar g(y-t) e^{-i\omega y} f_2(y) 
\di x \di y
\di \omega \di t
\]

We now return to Calc III and need to do a replacment of variables:
\[
\begin{matrix}
u = x-y,& v =y \\
x = u+v,& y =y 
\end{matrix}
\text{ which has deterimnate:\ }
J = 
\left|
\begin{bmatrix}
1 & -1 \\
0 & 1 
\end{bmatrix}
\right|
=1
\]

\[
=
\intinf \intinf 
\intinf  
\overline{
\bar g(u+v-t)  f_1(u+v) 
}
\bar g(v-t)  f_2(v) 
\intinf
e^{i\omega u}
\di \omega 
\di u \di v
\di t
\]

\[
\intinf
e^{i\omega u}
\di \omega 
= \delta( u)
\]

\[
=
\intinf \intinf 
\intinf  
\overline{
\bar g(u+v-t)  f_1(u+v) 
}
\bar g(v-t)  f_2(v) 
\delta(u)
\di u \di v
\di t
\]

\[
=
\intinf \intinf 
\overline{
\bar g(v-t)  f_1(v) 
}
\bar g(v-t)  f_2(v) 
 \di v
\di t
=
\intinf 
\bar  f_1(v)  f_2(v) 
 \intinf g(v-t)\bar g(v-t) 
\di t
 \di v
\]
\[
=
\intinf 
\bar  f_1(v)  f_2(v) 
|g|^2
 \di v
=
|g|^2
\intinf \bar f_1 f_2 \di x
= |g|^2 \langle f_1, f_2 \rangle
\]

\hrule
\item 1.5.3
\hrule
    
    ************************************
    [This seems tedious and maybe not worth it]

\hrule

\section{Chapter 2}
\subsection{Subsection 1}
\hrule
\item 2.1.1
Suppose $f(x) =f(x+2\pi) \ \forall x$ is periodic with period $2\pi$.
Show 
\[
\int_a^{2\pi +a} f(x) \di x
=
\int_0^{2\pi} f(x) \di x
, \ \forall a \in \RR
\]
\hrule
As all great math proofs, no words are needed just equalities and beautiful integrals.
Let a be given then:
\[
\int_a^{2\pi +a} f(x) \di x
=
\int_a^{2\pi} f(x) \di x +
\int_{2\pi}^{2\pi +a} f(x) \di x
=
\int_a^{2\pi} f(x) \di x +
\int_{0}^{a} f(x+2\pi) \di x
\]
\[
=
\int_a^{2\pi} f(x) \di x +
\int_{0}^{a} f(x) \di x
=
\int_0^{2\pi} f(x) \di x
\]

\hrule
\item 2.1.2
	{\bf Dirichelet Basis}
	\[
		W_{2N+1} = \text{span} \{ \frac 1 {\sqrt {2\pi}} e^{ikt} \}_{k=\pm N}
	\]
	Consider the set 
	\[
g_k(t) = \frac {2 \pi} { 2 N+1} \delta_N (t-x_k) = 
\frac 1 {2N+1} \sum_{n=-N}^N e^{in(t-k\pi / (N+\frac 1 2) } 
	\]
\hrule
*
\hrule


\hrule
\item 2.1.3
Riemann-Lebesgue Lemma

$G(u)$ piecewise continuous and has left and right derivatives on $[0,2\pi]$. Show that 
\[
\lim_{N \rw \infty} \int_0^{2\pi} G(u) \sin ( N +\frac 1 2)u \di u=0
\]
\hrule
WLOG $\exists a,b \in [0,\pi]$ st. $\forall x \in [a,b]$ $G(x) >0$ or $G(x) <0$.

Now it suffices to show 
\[
\lim_{N \rw \infty} \int_a^{b} G(u) \sin ( N +\frac 1 2)u \di u =0
\]
[since the interval $[0,2\pi]$ can be sliced into a countable number of these intervals, and hten you can sum over them]
WLOG we assume $G(x)$ is positve.

\[
0 \leq 
\lim_{N \rw \infty} \int_a^{b} G(u) \sin ( N +\frac 1 2)u \di u 
\leq
\lim_{N \rw \infty} \int_a^{b} [\max_u G(u)] \sin ( N +\frac 1 2)u \di u 
\]
Let $G_m$ be the max above, then we have
\[
0 \leq G_m\lim_{N \rw \infty} \int_a^{b} \sin ( N +\frac 1 2)u \di u 
= G_m\lim_{N \rw \infty}  
\frac{\cos ( N +\frac 1 2)u }
{N + \frac 1 2}|_a^{b}
\leq
 G_m\lim_{N \rw \infty}  
 \frac 2
{N + \frac 1 2}
\leq 0
\]
Thus we get the 0 value for the limit as desired.
\hrule
\item 2.1.4
Prove or disprove:
\[
\suminf{m} 
\frac 1 {(m+x)^2 + a^2} = \frac \pi a \frac {\coth \pi a } {\cos^2 \pi x + \sin^2 \pi x \coth^2 \pi a}
\]
\[
\suminf{m} 
\frac 1 {(m+x)^2} = \frac {\pi^2} { \sin^2 \pi x }
\]
\[
\suminf{m} 
\frac 1 {(2 \pi m+x)^2} = \frac {1} { 4 \sin^2 x/2 }
\]
\hrule
\hrule
\item 2.1.5
My man Stephane G. Mallat claims the following:
The family of funcitons $\phi(x-k) k = 0 ,\pm 1, \pm 2, \cdots$ is orthonormal iff 
\[
\suminf{k} |\hat \phi ( \omega +2\pi k)|^2 
\]
is constant wrt $\omega$.
Prove my boy wrong or right.
\hrule
Stephane is no chump and said a true thing. 
Lets investigate the sum:
\[
\suminf{k} |\hat \phi ( \omega +2\pi k)|^2 
=
\suminf{k} \overline{\hat \phi ( \omega +2\pi k)}
\hat \phi ( \omega +2\pi k) 
\]

\[
=
\suminf{k} 
\intinf e^{i (\omega+ 2\pi k)y}\bar \phi (y) \di y 
\intinf e^{-i (\omega+ 2\pi k)x} \phi (x) \di x 
\]
\[
=
\suminf{k} 
\intinf\intinf \phi (x)\bar \phi (y) e^{i (\omega+ 2\pi k)y}
 e^{-i (\omega+ 2\pi k)x}  \di x \di y 
\]
\[
=
\intinf\intinf \phi (x)\bar \phi (y) 
\suminf{k} 
 e^{-i (\omega + 2\pi k)(x-y)}  \di x \di y 
\]
Via formula on page 62
\[
=
\intinf\intinf \phi (x)\bar \phi (y) 
 e^{-i \omega(x-y)}  
\suminf{k} 
\delta(x-y-k)  \di x \di y 
\]

Now for the change of variables $u= x-y, v = y$
\[
=
\intinf\intinf \phi (u+v)\bar \phi (v) 
 e^{-i \omega u}  
\suminf{k} \delta(u-k)  \di u \di v 
=
\intinf\bar \phi (v) 
\suminf{k} \intinf e^{-i \omega u} \phi (u+v)\delta(u-k)  \di u \di v 
\]

\[
=
\intinf\bar \phi (v) 
\suminf{k}  e^{-i \omega k} \phi (k+v) \di v 
=
\suminf{k}
e^{-i \omega u}
\intinf\bar \phi (v) 
  \phi (k+v) \di v 
=
\suminf{k}
\langle \phi (v) ,
  \phi (k+v) \rangle 
  e^{-i \omega u}
\]

Thus the sum above is in fact a Fourier series with $c_k = 
\langle \phi (v) , \phi (k+v) \rangle $.
Now this series being constnat is equivalent to $c_k = \delta_{0k}$, which is equivalent to the $\phi(v+k)$'s being an orthogonal system.

Moreover if $c_0 =1$ then we have an orthonormal system aswell. Thus the system is orthonormal if the series is constant and equal to 1.


\hrule
\item 2.1.6

\hrule

\hrule



\end{enumerate}










































\end{document}














