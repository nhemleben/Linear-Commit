
\documentclass[12pt]{article} 
\usepackage{graphicx,relsize, verbatim, amssymb, amsmath, amsthm, mathabx,dcolumn,mathrsfs, dsfont, enumerate, soul, titlesec}

\newcolumntype{2}{D{.}{}{2.0}}
\textwidth = 7 in
\textheight = 9.5 in
\oddsidemargin = -0.3 in
\evensidemargin = -0.3 in
\topmargin = -0.4 in
\headheight = 0.0 in
\headsep = 0.0 in
\parskip = 0.2in
\parindent = 0.0in
\titlespacing*{\section}
{0pt}{0ex plus .1ex minus .2ex}{.1ex plus .1ex}
\titlespacing*{\subsection}
{0pt}{.5ex plus .1ex minus .2ex}{.3ex plus .2ex}

\DeclareMathOperator{\Gal}{Gal}	
\DeclareMathOperator{\gal}{Gal}	
\DeclareMathOperator{\im}{Image}
\DeclareMathOperator{\Ann}{Ann}	
\DeclareMathOperator{\spn}{span}	
\DeclareMathOperator{\sgn}{sgn}	
\DeclareMathOperator{\sinc}{sinc}	


\newcommand{\CC}{\mathbb{C}}	
\newcommand{\GG}{\mathbb{G}}	
\newcommand{\LL}{\mathbb{L}}	
\newcommand{\NN}{\mathbb{N}}	
\newcommand{\PP}{\mathbb{P}}	
\newcommand{\QQ}{\mathbb{Q}}	
\newcommand{\RR}{\mathbb{R}}	
\newcommand{\ZZ}{\mathbb{Z}}
\newcommand{\FF}{\mathfrak{F}}	
\newcommand{\one}{\mathds{1}}
\newcommand{\eps}{\epsilon}

\newcommand{\floor}[1]{\lfloor #1 \rfloor}
\newcommand{\s}[1]{\sqrt{#1}}
\newcommand{\norm}[1]{\left\lVert#1\right\rVert}
\newcommand{\I}[1]{#1^{-1}}
\newcommand{\Rw}{\Rightarrow}
\newcommand{\Lw}{\Leftarrow}
\newcommand{\lw}{\leftarrow}
\newcommand{\rw}{\rightarrow}
\newcommand{\p}{\partial}
\newcommand{\var}{\mathrm{Var}}
\newcommand{\sumi}{\sum_{i=0}}
\newcommand{\sumo}{\sum_{i=1}}
\newcommand{\num}[1]{{\large \bf  #1)}}
\newcommand{\del}{\Delta}
\newcommand{\lamb}{\lambda}	
\newcommand{\al}{\alpha}	
\newcommand{\nab}{\nabla}	
\newcommand{\pp}[2]{\frac{\p #1}{\p #2}}
\newcommand{\di}{\mathrm{d}}
\newcommand{\dd}[2]{\frac{\di #1}{\di #2}}
\newcommand{\intinf}{\int_{-\infty}^\infty}	
\newcommand{\suminf}[1]{\sum_{#1 = -\infty}^\infty}	
\newcommand{\ispi}{\frac 1 {\sqrt{2 \pi}} }

\newtheorem{thm}{Theorem}
\theoremstyle{definition}
\newtheorem{defi}{Definition}[section]
\newtheorem{lem}{Lemma}[section]

\newcounter{Prb}%[subsection]
\newenvironment{Porb}[1][]{
	\refstepcounter{Prb}\par\medskip 
   \noindent \textbf{\thesubsection~Problem~\thePrb. 
   #1
   }
		       }
   {\medskip \hrule}
   %Just need one hline at the end [otheriwse double thicc lines
\newenvironment{Boxed}{\begin{center}
	           \begin{tabular}{|p{0.9\textwidth}|}
			       \hline\\
	           }
	       { 
           \\\\\hline
	       \end{tabular} 
           \end{center}
		       }
%So basicly the problem envorment does not want ot box stuff 
%To combat this we use 2 enviroments, the first being Problem
%the second being boxed
%Problem goes in box
%Solution goes outside box
%Hlines for everyone and the world is happy
%%%%% End of preamble %%%%%
\begin{document}
\title{Linear Math In Infinte Dimensions}
{\Large Nicholas Hemleben} \hfill
{\large Linear Math In Infinte Dimensions}  
\hfill  \today

\section{Chapter 1}
\setcounter{subsection}{4}
\subsection{ Subsection 5}
%\setcounter{Prb}{-1}
%		\begin{Porb}
%		\begin{Boxed}
%			A Demo porblem
%			Inside a Box
%		\end{Boxed}
%			Outside a Box
%		\end{Porb}

%THe first hrule
\hrule
\begin{Porb}
\begin{Boxed}
{\large 

Show that 
\vspace{-2mm}
\[
\langle f , g \rangle = \sum_{k=1}^\infty \bar c_k d_k 
\]
}
with
\[
\langle f , g \rangle = \int_a^b \bar f (x) g(x) \rho(x) \di x
\]
\end{Boxed}

Since we have that $f(x) = \sum_{k=1}^\infty u_k(x) c_k$ we can substitiute and get:

\[
= \int_a^b \overline { \sum_{k=1}^\infty u_k(x) c_k } g(x) \rho(x) \di x
=  \sum_{k=1}^\infty \int_a^b \bar u_k(x) \bar c_k  g(x) \rho(x) \di x
\]
\[
=  \sum_{k=1}^\infty \int_a^b \bar u_k(x) \bar c_k  g (x)\rho (x)\di x
=  \sum_{k=1}^\infty \bar c_k \int_a^b \bar u_k(x)  g (x)\rho (x)\di x
= \sum_{k=1}^\infty \bar c_k d_k 
\]
\end{Porb}


\begin{Porb}
\begin{Boxed}


\[
T f(\omega, t) = 
\int_{-\infty}^\infty \bar g(x-t) e^{-i\omega x} f(x) \di x
\]

\[
\langle h_1, h_2 \rangle = \intinf \intinf \bar h_1(\omega,t) h_2(\omega,t) \di \omega \di t
\]
Find a formula for:
$\langle Tf_1, Tf_2 \rangle $
in terms of 
\[
\intinf \bar f_1 f_2 \di x
\]

\end{Boxed}
\[
\langle Tf_1, Tf_2 \rangle 
=
\intinf \intinf 
\overline{Tf_1} Tf_2  \di \omega \di t
=
\]

\[
\intinf \intinf 
\overline{
\left[
\int_{-\infty}^\infty \bar g(x-t) e^{-i\omega x} f_1(x) \di x
\right]
}
\left[
\int_{-\infty}^\infty \bar g(x-t) e^{-i\omega x} f_2(x) \di x
\right]
\di \omega \di t
\]

\[
=
\intinf \intinf 
\intinf \intinf 
\overline{
\bar g(x-t) e^{-i\omega x} f_1(x) 
}
\bar g(y-t) e^{-i\omega y} f_2(y) 
\di x \di y
\di \omega \di t
\]

We now return to Calc III and need to do a replacment of variables:
\[
\begin{matrix}
u = x-y,& v =y \\
x = u+v,& y =y 
\end{matrix}
\text{ which has deterimnate:\ }
J = 
\left|
\begin{bmatrix}
1 & -1 \\
0 & 1 
\end{bmatrix}
\right|
=1
\]

\[
=
\intinf \intinf 
\intinf  
\overline{
\bar g(u+v-t)  f_1(u+v) 
}
\bar g(v-t)  f_2(v) 
\intinf
e^{i\omega u}
\di \omega 
\di u \di v
\di t
\]

\[
\intinf
e^{i\omega u}
\di \omega 
= \delta( u)
\]

\[
=
\intinf \intinf 
\intinf  
\overline{
\bar g(u+v-t)  f_1(u+v) 
}
\bar g(v-t)  f_2(v) 
\delta(u)
\di u \di v
\di t
\]

\[
=
\intinf \intinf 
\overline{
\bar g(v-t)  f_1(v) 
}
\bar g(v-t)  f_2(v) 
 \di v
\di t
=
\intinf 
\bar  f_1(v)  f_2(v) 
 \intinf g(v-t)\bar g(v-t) 
\di t
 \di v
\]
\[
=
\intinf 
\bar  f_1(v)  f_2(v) 
|g|^2
 \di v
=
|g|^2
\intinf \bar f_1 f_2 \di x
= |g|^2 \langle f_1, f_2 \rangle
\]

\end{Porb}
\begin{Porb}
\begin{Boxed}
\begin{enumerate}[i)]
	\item
	Show that the set of functions
	\[
		\left\{
			\frac{ \sin \pi ( 2 \omega z - k) } { \pi 2 \omega z -k)} = \sinc(2 \omega z-k), \ k \in \ZZ
			\right\}
		\]
		is an orthognal set satisfying:
		\[
			\intinf \sinc(2 \omega z -k) \sinc( 2 \omega z -l) \di z = A \delta_{kl}
			\]
			What is A?

\begin{enumerate}[a)]
	\item
		\[
			\int_{-\pi}^\pi \delta_N \left( t - \frac{ 2\pi} {2N+1}k \right)
	\delta_N \left( t - \frac{ 2\pi} {2N+1}l \right) \di t
	=	
		\frac{2N+1}{2 \pi} \delta_{kl}
		\]
		where 
		\[
			\delta_N(u) = \frac 1 {2\pi} \frac{ \sin( N+\frac 1 2)u } { \sin \frac u 2} = \frac 1 {2\pi} \sum_{n=-N}^N e^{inu}
			\]
	\item
		Then rescaling the integration domain by using $z= \frac{ N+ \frac 12 } { 2\pi \omega} t$.


	\item
		and finally going to the limit $N \rw \infty$.
\end{enumerate}

\item
	This set of functions 
		\[
			\left\{
				u_k = \frac 1 {\sqrt A} \sinc ( 2 \omega z - k), \ k \in \ZZ
			\right\}
		\]
		is not complete on $L^2$, but is complete on a specific subset. What is this subset, ie what property must a function $f(t)$ satisfy in order to be in this subset?
\end{enumerate}
\end{Boxed}


*****************************************************
\begin{enumerate}[i)]
	\item
		If we take $k=0=l$ then we can calculate A:
\begin{enumerate}[i)]
	\item
		\[
			\int_{-\pi}^\pi \delta_N \left( t - \frac{ 2\pi} {2N+1}k \right)
	\delta_N \left( t - \frac{ 2\pi} {2N+1}l \right) \di t
	=	
		\frac{2N+1}{2 \pi} \delta_{kl}
		\]

\end{enumerate}
\end{enumerate}







    ************************************
\end{Porb}



\section{Chapter 2}
\subsection{Subsection 1}
\setcounter{Prb}{0}

\begin{Porb}
\begin{Boxed}
Suppose $f(x) =f(x+2\pi) \ \forall x$ is periodic with period $2\pi$.
Show 
\[
\int_a^{2\pi +a} f(x) \di x
=
\int_0^{2\pi} f(x) \di x
, \ \forall a \in \RR
\]
\end{Boxed}
As all great math proofs, no words are needed just equalities and beautiful integrals.
Let a be given then:
\[
\int_a^{2\pi +a} f(x) \di x
=
\int_a^{2\pi} f(x) \di x +
\int_{2\pi}^{2\pi +a} f(x) \di x
=
\int_a^{2\pi} f(x) \di x +
\int_{0}^{a} f(x+2\pi) \di x
\]
\[
=
\int_a^{2\pi} f(x) \di x +
\int_{0}^{a} f(x) \di x
=
\int_0^{2\pi} f(x) \di x
\]

\end{Porb}
\begin{Porb}
\begin{Boxed}
	{\bf Dirichelet Basis}
	\[
		W_{2N+1} = \text{span} \left\{ \frac 1 {\sqrt{2\pi} } e^{ikt} \right\} _{k=\pm N}
	\]
	Consider the set 
	\[
g_k(t) = \frac {2 \pi} { 2 N+1} \delta_N (t-x_k) = 
\frac 1 {2N+1} \sum_{n=-N}^N e^{in(t-k\pi / (N+\frac 1 2) } 
	\]
	Show that
	\begin{enumerate}[A)]
		\item $B = \{ g_k , k \in 1,2, \cdots \}$ is linearly independent.
		\item B spans $W_{2N+1}$
	\end{enumerate}
\end{Boxed}
	\begin{enumerate}[A)]
	\item
		It suffices to notice that $g_k(x_l) = \delta_{kl}$. Thus we can see for any given k that $g_k$ is independent of all the other's as $\sum_{k' \neq k} \lambda_{k'} g_{k'} (x_k) = 0$. Thus we can not have a non trivial linear relationship between the functions. 
	\item
		It is clear that $g_k(t) \in W_{2N+1}$ since each of the elements in its sum namely $e^{in(t-k\pi/(N+\frac 1 2)}$ is just a multiple of $e^{int}$ a basis element of $W_{2N+1}$.
		Notice there are $2N+1$ of these independnet vectors in the vector space of dimmension $2N+1$. 
			Thus they must be a spanning set and there must exist coefficents for any function in the space to be written as a sum of this basis.

		To actually exhibit coefficents one would use $f(t) = \sum_k f(x_k) g_k(t)$. 
		
	\end{enumerate}

\end{Porb}
\begin{Porb}
\begin{Boxed}
	{\bf Riemann-Lebesgue Lemma}\\
$G(u)$ piecewise continuous and has left and right derivatives on $[0,2\pi]$. Show that 
\[
\lim_{N \rw \infty} \int_0^{2\pi} G(u) \sin ( N +\frac 1 2)u \di u=0
\]
\end{Boxed}
WLOG $\exists a,b \in [0,\pi]$ st. $\forall x \in [a,b]$ $G(x) >0$ or $G(x) <0$.

Now it suffices to show 
\[
\lim_{N \rw \infty} \int_a^{b} G(u) \sin ( N +\frac 1 2)u \di u =0
\]
[since the interval $[0,2\pi]$ can be sliced into a countable number of these intervals, and hten you can sum over them]
WLOG we assume $G(x)$ is positve.

\[
0 \leq 
\lim_{N \rw \infty} \int_a^{b} G(u) \sin ( N +\frac 1 2)u \di u 
\leq
\lim_{N \rw \infty} \int_a^{b} [\max_u G(u)] \sin ( N +\frac 1 2)u \di u 
\]
Let $G_m$ be the max above, then we have
\[
0 \leq G_m\lim_{N \rw \infty} \int_a^{b} \sin ( N +\frac 1 2)u \di u 
= G_m\lim_{N \rw \infty}  
\frac{\cos ( N +\frac 1 2)u }
{N + \frac 1 2}|_a^{b}
\leq
 G_m\lim_{N \rw \infty}  
 \frac 2
{N + \frac 1 2}
\leq 0
\]
Thus we get the 0 value for the limit as desired.

\end{Porb}
\begin{Porb}
\begin{Boxed}
Prove or disprove:
\[
\suminf{m} 
\frac 1 {(m+x)^2 + a^2} = \frac \pi a \frac {\coth \pi a } {\cos^2 \pi x + \sin^2 \pi x \coth^2 \pi a}
\]
\[
\suminf{m} 
\frac 1 {(m+x)^2} = \frac {\pi^2} { \sin^2 \pi x }
\]
\[
\suminf{m} 
\frac 1 {(2 \pi m+x)^2} = \frac {1} { 4 \sin^2 x/2 }
\]


\end{Boxed}
Let $f(m) = \frac 1 { (m+x)^2 +a^2} $, then we wish to find $\sum f(m)$. 
To this end we consider $F(k) = \intinf e^{-ikm} f(m) \di m$. To find this we use the u sub: $u = m+x$
\[
\intinf e^{-ikm} \frac 1 { (m+x)^2 +a^2}  \di m
=
e^{ikx} \intinf e^{-iku} \frac 1 { u^2 +a^2}  \di u
= 
e^{ikx} \frac \pi a e^{-|k| a}
\]


Using the Poisson formula we thus see:
\[
\suminf{m} 
\frac 1 {(m+x)^2 + a^2}  	
=
\suminf{k}
e^{i2\pi kx} \frac \pi a e^{-|2\pi k| a}
=
 \frac \pi a
 \left[
\sum{k\geq0}
e^{i 2 \pi kx}  e^{-| 2 \pi k| a}
+
\sum{k\leq0}
e^{i 2 \pi kx}  e^{-| 2 \pi k| a}
-1
\right]
\]

\[
=
 \frac \pi a
 \left[
\sum{k\geq0}
e^{i 2 \pi k(x-a)}
+
\sum{k\leq0}
e^{i 2 \pi k(x+a)}
-1
\right]
=
 \frac \pi a
 \left[
\sum{k\geq0}
e^{i 2 \pi k(x-a)}
+
e^{-i 2 \pi k(x+a)}
-1
\right]
\]

\[
=
 \frac \pi a
 \left[
\frac{ 1} { 1 -  e^{-i 2 \pi (x+a)} } +
\frac{ 1} { 1 -  e^{i 2 \pi (x-a)} }
-1
\right]
\]
\[
=
 \frac \pi a
 \left[
\frac{
1 -  e^{-i 2 \pi (x+a)}  
+1 -  e^{i 2 \pi (x-a)} 
-
\left(1 -  e^{i 2 \pi (x-a)} \right) 
\left(1 -  e^{-i 2 \pi (x+a)} \right) 
 	}
 { \left(1 -  e^{i 2 \pi (x-a)} \right) 
  \left(1 -  e^{-i 2 \pi (x+a)} \right) 
}
\right]
\]

\[
=
 \frac \pi a
 \left[
\frac{
1-
 e^{i 2 \pi (x-a)} 
e^{-i 2 \pi (x+a)} 
 	}
 { \left(1 -  e^{i 2 \pi (x-a)} \right) 
  \left(1 -  e^{-i 2 \pi (x+a)} \right) 
}
\right]
=
 \frac \pi a
 \left[
\frac{
1-
 e^{-i 4 \pi a} 
 	}
 { \left(1 -  e^{i 2 \pi (x-a)} \right) 
  \left(1 -  e^{-i 2 \pi (x+a)} \right) 
}
\right]
\]

\[
=
 \frac \pi a
 \left[
\frac{
	\left(1-e^{-i 2 \pi a}\right) 
	\left(1+e^{-i 2 \pi a}\right) 
 	}
 { \left(1 -  e^{i 2 \pi (x-a)} \right) 
  \left(1 -  e^{-i 2 \pi (x+a)} \right) 
}
\right]
=
 \frac \pi a
 \left[
\frac{
	\left(1-e^{-i 2 \pi a}\right) 
	\left(1+e^{-i 2 \pi a}\right) 
 	}
 { \left(1 -  e^{i 2 \pi (x-a)} \right) 
  \left(1 -  e^{-i 2 \pi (x+a)} \right) 
}
\right]
\]

************BELIEVE THE ABOVE DISPROVES THE SUM********\\

For the last 2 the answers are somewhat lack luster. 


Let $f(m) = \frac 1 { (m+x)^2} $, then we wish to find $\sum f(m)$. 
To this end we consider $F(k) = \intinf e^{-ikm} f(m) \di m$. To find this we use the u sub: $u = m+x$
\[
\intinf e^{-ikm} \frac 1 { (m+x)^2 }  \di m
=
e^{ikx} \intinf e^{-iku} \frac 1 { u^2 }  \di u
\]
This is an error function producing integral. This is bad.
Notice also that the claimed value on the right hand side of the equation doesn't even make sense for $x=0$.
Thus the equation as stated is clearly false, 
Besides as any good student of Bergelson knows: $\sum 1/n^2 = \frac {\pi^2} 6$



For the final one we again have issues with $\sin (0 ) = 0$
in the denominator.


\end{Porb}
\begin{Porb}
\begin{Boxed}
My man Stephane G. Mallat claims the following:
The family of funcitons $\phi(x-k) k = 0 ,\pm 1, \pm 2, \cdots$ is orthonormal iff 
\[
\suminf{k} |\hat \phi ( \omega +2\pi k)|^2 
\]
is constant wrt $\omega$.
Prove my boy wrong or right.

\end{Boxed}
Stephane is no chump and said a true thing. 
Lets investigate the sum:
\[
\suminf{k} |\hat \phi ( \omega +2\pi k)|^2 
=
\suminf{k} \overline{\hat \phi ( \omega +2\pi k)}
\hat \phi ( \omega +2\pi k) 
\]

Now to avoid a factor out front the rest of the analysis, the $\frac 1 {\sqrt {2\pi}} $ is suppressed when expanding the Fourier transform.
\[
=
\suminf{k} 
\intinf e^{i (\omega+ 2\pi k)y}\bar \phi (y) \di y 
\intinf e^{-i (\omega+ 2\pi k)x} \phi (x) \di x 
\]
\[
=
\suminf{k} 
\intinf\intinf \phi (x)\bar \phi (y) e^{i (\omega+ 2\pi k)y}
 e^{-i (\omega+ 2\pi k)x}  \di x \di y 
\]
\[
=
\intinf\intinf \phi (x)\bar \phi (y) 
\suminf{k} 
 e^{-i (\omega + 2\pi k)(x-y)}  \di x \di y 
\]
Via formula on page 62
\[
=
\intinf\intinf \phi (x)\bar \phi (y) 
 e^{-i \omega(x-y)}  
\suminf{k} 
\delta(x-y-k)  \di x \di y 
\]

Now for the change of variables $u= x-y, v = y$
\[
=
\intinf\intinf \phi (u+v)\bar \phi (v) 
 e^{-i \omega u}  
\suminf{k} \delta(u-k)  \di u \di v 
=
\intinf\bar \phi (v) 
\suminf{k} \intinf e^{-i \omega u} \phi (u+v)\delta(u-k)  \di u \di v 
\]

\[
=
\intinf\bar \phi (v) 
\suminf{k}  e^{-i \omega k} \phi (k+v) \di v 
=
\suminf{k}
e^{-i \omega u}
\intinf\bar \phi (v) 
  \phi (k+v) \di v 
=
\suminf{k}
\langle \phi (v) ,
  \phi (k+v) \rangle 
  e^{-i \omega u}
\]

Thus the sum above is in fact a Fourier series with $c_k = 
\langle \phi (v) , \phi (k+v) \rangle $.
Now this series being constnat is equivalent to $c_k = \delta_{0k}$, which is equivalent to the $\phi(v+k)$'s being an orthogonal system.

Moreover if $c_0 =1$ then we have an orthonormal system aswell. Thus the system is orthonormal if the series is constant and equal to 1.
Now in actuallity we rememeber that we have a secret factor of $\frac 1 {  2 \pi }$ hanging around. Thus the constants value is actually that.


\end{Porb}
\begin{Porb}
\begin{Boxed}
Prove or Disprove the following identities:
\begin{enumerate}[i)]
\item
		\[
	\suminf{m} f( [2m+1] \pi) = \frac 1 {2\pi} \suminf{n} (-1)^n F(n)
		\]
\item
		\[
		2 \pi \suminf{m} (-1)^m f( 2\pi m) = \suminf{n} F(n + \frac 1 2)
		\]
\item
		\[
			\suminf{m} \delta(u- [2m+1] \pi) = \frac 1 {2\pi} \suminf{n} (-1)^n e^{inu}
		\]
\item
	And in greater generality 
		\[
	\suminf{m} f(\frac{ [2m+1] \pi} {a} ) =
		\frac a {2\pi} \suminf{n} (-1)^n F(na)
		\]
\item
		\[
	\suminf{m} \frac 1 {|a|} \delta(u- \frac{[2m+1] \pi} {a}) = \frac 1 {2\pi} \suminf{n} (-1)^n e^{inau}
		\]
\end{enumerate}

\end{Boxed}
The main equation to keep in mind here is the general Poisson formula:
\[
	\frac 1 {2 \pi} \suminf{n} \intinf e^{in(x-t)} f(t) \di t = \suminf{m} f(x+ 2 \pi m)
\]
\begin{enumerate}[i)]
	\item
		Begin with $x=\pi$ in the formula above and we see:
\[
	\frac 1 {2 \pi} \suminf{n} \intinf e^{in(\pi -t)} f(t) \di t = \suminf{m} f([2 m+1]\pi )
\]
\[
=	\frac 1 {2 \pi} \suminf{n} (-1)^n \intinf e^{-int} f(t) \di t
		=	\frac 1 {2 \pi} \suminf{n} (-1)^n F(n)
	\]



	\item
		Begining with the right hand side:
		\[
			\suminf{n} F(n + \frac 1 2) 
	= \suminf{n} \intinf e^{-i(n+\frac 1 2) t} f(t) \di t
	= \suminf{n} \intinf e^{-in t} f(t)e^{-i t/2} \di t
		\]
		Now we notice this is the Fourier transform of not $f$ but of $f(t) e^{-i t/2}$, applying Poisson sum with this:
		\[
			= 2\pi \suminf{m} f(2 \pi m) e^{-i (2 \pi m)/2}
			= 2\pi \suminf{m} f(2 \pi m) e^{-i \pi m}
		=2 \pi \suminf{m} (-1)^m f( 2\pi m) 
		\]

	\item
		It is straightforward to see that this is actually just 1) but in the supressed function notation.
		To see this we note that
		\[
			\delta( u - [2m+1] \pi) \rw f([2m+1] \pi), \quad  e^{inu} \rw \intinf e^{inu} f(u) \di u = F(-n)
		\]
		But wait we get $\suminf n (-1)^n F(-n)$ and not the exact sum we wanted! Thankfully $(-1)^n = (-1)^{-n}$ and we just switch the order of the sum and get the identity.
		
	\item
		Let $\bar f(x) = f(\frac x a )$, then by 1) we have:
		\[
			\suminf m f ( \frac {[2m+1] \pi} a) = \suminf m \bar f ( [2m+1] \pi) =
			\frac 1 {2\pi} \suminf n (-1)^n \bar F (n) 
		\]
		\[
			=
			\frac 1 {2\pi} \suminf n (-1)^n \intinf e^{-int} \bar f (t) \di t
			=
			\frac 1 {2\pi} \suminf n (-1)^n \intinf e^{-int} f (t/a) \di t
		\]
		\[
			=
			\frac 1 {2\pi} \suminf n (-1)^n a  \intinf e^{-inau} f (u) \di u
			=
			\frac 1 {2\pi} \suminf n (-1)^n a F(na) 
			=
			\frac a {2\pi} \suminf n (-1)^n  F(na) 
		\]
		
	\item
		Similar to 3) we note that this is just an earlier identity. A constant is shifted around but this is basicly just 4).
		
\end{enumerate}
\end{Porb}

\subsection{Dirac Delta Distribution}
\setcounter{Prb}{0}

\begin{Porb}
\begin{Boxed}
	Show that 
	\[
		\lim_{\omega \rw \infty} \frac { \sin 2 \pi \omega x } { \pi x}, \ \omega >0
	\]
	is a representation of the Dirac $\delta$ DISTRIBUTION.
\end{Boxed}
This equality can only be expressed inside of an integral, thus we must apply the above to test functions and see that the answer is the same as with the delta \emph{distribution}.

Thus if we consider f continuous on some $[-a,a]$ then we get:
\[
	\lim_{\omega \rw \infty} \int_{-a}^a \delta_\omega (x) f(x) \di x
	=
	\lim_{\omega \rw \infty} \int_{-a}^a\frac { \sin 2 \pi \omega x } { \pi x} f(x) \di x
\]
Now in following with the style of the Fourier series theorem we add and subtract the same term, namely a $f(0)$ (inside some paranthesis but basicly the same)

\[
	=
	\lim_{\omega \rw \infty} \int_{-a}^a\frac { \sin 2 \pi \omega x } { \pi x} \left(f(x)  - f(0) + f(0) \right)\di x
	=
	\lim_{\omega \rw \infty} \int_{-a}^a\frac { \sin 2 \pi \omega x } \pi  \frac{ f(x)  - f(0)}  { x}
+f(0)\frac { \sin 2 \pi \omega x } { \pi x}\di x
\]

Now we can consider WLOG just the positive side of the integral.
\[
	\lim_{\omega \rw \infty} \int_{0}^a\frac { \sin 2 \pi \omega x } \pi  \frac{ f(x)  - f(0)}  { x}
+f(0)\frac { \sin 2 \pi \omega x } { \pi x}\di x
\]
Notice for the exact same resonaning as $G(u)$ on page 57 that we get $\frac{ f(x) - f(0)} {x}$ is continuous at 0 and converges to $f'(0^+)$. 
Thus again we see that the integral:
\[
	\int_{0}^a\frac { \sin 2 \pi \omega x } \pi  \frac{ f(x)  - f(0)}  { x} \di x 
	\rw 0
\]
as $\omega \rw \infty$. Thus we only have:
\[
	\lim_{\omega \rw \infty} \int_0^a f(0)\frac { \sin 2 \pi \omega x } { \pi x}\di x
	=
	f(0) \lim_{\omega \rw \infty} \int_0^a \frac { \sin 2 \pi \omega x } { \pi x}\di x
\]

	\[
	y = 2 \pi \omega x, \quad 
	dy = 2 \pi \omega dx
	\]

	\[
	\frac y {2 \pi \omega } = x, \quad 
	\frac {dy} { 2 \pi \omega } = dx
	\]

\[
	=
	f(0) \lim_{\omega \rw \infty} \int_0^{a 2 \pi \omega} \frac { \sin y } { \pi y/ (2 \pi \omega) }\frac{ \di y} {2 \pi \omega}
	=
	\frac {f(0)} \pi  \int_0^{\infty} \frac { \sin y } {  y } \di y
	=
	\frac {f(0)} \pi  \frac \pi 2
	=
	\frac {f(0)} 2
\]
Using a isomorphic version of the logic above one can get the $f(0^-)$ term and complete the proof.



\end{Porb}
\begin{Porb}
\begin{Boxed}
	Assuming that $f(x)$ is nearlly linear, that is to say that 
	\[
		f(-a) = f(0) - a f'(0) + \text{ H.O.T.}
	\]
	Show that 
	\[
	I = \intinf \delta ( x+a) f(x) \di x
	\]
	can be evaluated by means of the formal equation:
	\[
	\delta(x+a) = \delta(x) + a \delta'(xx)
	\]

\end{Boxed}

By the definition of the $\delta$ funciton we have:
\[
	I = \intinf \delta ( x+a) f(x) \di x = f(-a)
\]
\[
	=
	f(0) - a f'(0) + \text{ H.O.T.}
	= 
	\intinf  \delta (x)f(x)  - a \delta(x)f'(x) \di x 
	= 
	\intinf  \delta (x)f(x)\di x  - a \intinf \delta(x)f'(x) \di x 
\]
Via integration by parts we know that:
\[
\intinf \delta(x)f'(x) \di x 
=
\delta(x) f(x) |_{\pm \infty}
- \intinf \delta'(x)f(x) \di x 
=
- \intinf \delta'(x)f(x) \di x 
\]
Putting stuff together:
\[
	= 
	\intinf  \delta (x)f(x)\di x  + a \intinf \delta'(x)f(x) \di x 
	= 
	\intinf  \left[\delta (x)+ a \delta'(x)\right]f(x) \di x 
\]
Thus it makes some sense to claim $\delta (x +a ) = \delta(x) + a\delta'(x)$

\end{Porb}


\subsection{The Fourier Integral}
\setcounter{Prb}{0}

\begin{Porb}
\begin{Boxed}
	\begin{enumerate}[a)]
		\item Consider the LInear Operator $\mathfrak{F}^2$ and its eigenvalue equatio
			\[
				\mathfrak{F}^2 f = \lambda f
				\]
				What are the eigenvalues and eigenfunctions of $\mathfrak{F}^2$?
		\item Same with $\mathfrak{F}^4$?
		\item Same with $\mathfrak{F}$?
	\end{enumerate}

\end{Boxed}
For the sake of clarity:
	\[
		\FF (f) =\frac 1 {\sqrt{2 \pi}}  \intinf e^{-ikx} f(x) \di x
		\]
		\begin{enumerate}[a)]
			\item
				Obviously we begin with the calculation in question
	\[
		\FF^2 (f)[k] 
				=\frac 1 {2 \pi}  \intinf e^{-iky}  \intinf e^{-iyx} f(x) \di x\di y
				=\frac 1 {2 \pi}  \intinf  \intinf e^{-i(ky+yx)} f(x) \di x\di y
	\]
				This looks close to the idenity we are given on page 70, namely:
		\[
		\delta(x-t) = \intinf \frac 1 {2\pi} e^{ik(x-t)} dk
		\]
	which carries the note that we must inegrate on the outside with $\di t$ for this to make sense. 
		Now rearranging some integrals and swapping x with $-x$ we arive at:

	\[
	  \intinf f(-x)  \intinf\frac 1 {2 \pi} e^{-iy(k-x)} \di y \di x
	=
	  \intinf f(-x)  \delta(k-x) \di x
	=
	f(-k)  	\]
				

Now we can see that the constraint: $ \mathfrak{F}^2 f = \lambda f$
is really just $f(-x) = \lambda f(x)$. 
Two obvious cases come to mind, namely even and odd functions for the eigenvalues $\pm 1$. 
				For any other value of $\lambda$ one could apply the relation twice to get $f(x) = \lambda^2 f(-x)$ which only has 2 roots. 
				Thus those are the only eigenvalues of $\FF^2$.


	\item
		Thanks to $ \mathfrak{F}^2 f[x] = f(-x)$ from the previous problem we know that $\FF^4 = \FF^2\FF^2 f [x] = f(-(-x)) =f(x)$.
				Thus every function is an eigen function of $\FF^4 = $I$\di$, with eigenvalue 1.

	\item
		We know that $\FF^4=$I$\di$, and thus if $\lambda$ is an eigenvalue of $\FF$ then $\lambda^4=1$. Thus the only possible eigenvalules of $\FF$ are 4th roots of unity. 
		Thus the only possible eigenvalues are $\pm1, \pm i$. 

		****************NEED TO DEMONSTRATE VALUES?********************

		\end{enumerate}
\end{Porb}


\begin{Porb}
\begin{Boxed}
	Let 
	\[
		W = \text{span}\{ \phi, \FF \phi, \FF^2 \phi, \dots\}
		\]
	\begin{enumerate}[a)]
		\item Show that W is finite dimensional, and what is its dimension?
		\item Exhibit a basis for W.
		\item It is evident that $\FF$ is a unitary transform of W. Find the bassis representation matrx $[\FF]_B$ relative to the basis B found in part b).
		\item Find the secular determinant, the eigenvalues and the corresponding eigenvectors of $[\FF]_B$.
		\item For W exhibit an alternative basis which consists entirely of eigenvectors of $\FF$, each one labelled by its respective eigenvalue.
		\item What can you say about the eigenvalues of $\FF$ as a transformation on $L^2$ as compared to $[\FF]_B$ which acts ona finite dim. vector space
	\end{enumerate}
\end{Boxed}

	\begin{enumerate}[a)]
	\item
		W clearly has dimension $\leq 4$ by the previous problem since $\FF^4 =$I$\di$. In fact if $\phi$ is even, we have only dimension 2 and the two basis elements of the space are just $\phi$ and $\FF \phi$. Due to the limits of the roots of unity argument above we know that the only number of dimensions can be those two or 1, namely dim = 1,2, or 4.

	\item The possible basis are $\phi$, or $\phi, \FF \phi$, or all 4: $\phi, \FF \phi, \FF^2 \phi, \I \FF \phi$.

	\item
		With the basis: $\phi, \FF \phi, \FF^2 \phi, \I \FF \phi$
		\[
		[\FF]_B 
		=
		\begin{bmatrix}
			0 &1 & 0 & 0\\
			0 &0 & 1 & 0\\
			0 &0 & 0 & 1\\
			1 &0 & 0 & 0
		\end{bmatrix}
		\]
			A classic style of shifting operator on a finite dimensional space.
	\item
		*******************************************************************8
		
	\item

	\end{enumerate}
\end{Porb}


\begin{Porb}
\begin{Boxed}
	Define the equivalent width as 
	\[
		\Delta_t = \left| \frac{ \intinf f(t) \di t } {f(0)} \right|
		\]
	Define the equivalent Fourier width as 
	\[
		\Delta_\omega = \left| \frac{ \intinf \hat f(t) \di t } {\hat f(0)} \right|
		\]
	\begin{enumerate}[a)]
		\item Show that $\Delta_t \Delta_\omega = $const, is independent of the function f, and find its value.
		\item Determine the equivalent width and Fourier width of 
		\[
			e^{-x^2/2b^2}
		\]
			and compare them with its full width as defined by its inflection points.
	\end{enumerate}
\end{Boxed}
	\begin{enumerate}[a)]
		\item
			\[
		\Delta_t \Delta_\omega 
			= \left| \frac{ \intinf \hat f(t) \di t } {\hat f(0)} \right|
			\left| \frac{ \intinf f(t) \di t } {f(0) }\right|
			= \left| \frac{ \intinf \hat f(t) \di t  \intinf f(t) \di t } {\hat f (0) f(0)} \right|
			\]
			\[
			= \left| \frac{ \intinf\intinf \hat f(x) f(y)\di x \di y } {\hat f (0) f(0)} \right|
			= \left| \frac{ \intinf\intinf \intinf f(y) \ispi e^{-ixz} f(z)\di z \di x \di y } {\hat f (0) f(0)} \right|
			=
			\]
			\[
			 \left| \frac{ \intinf\intinf f(y)f(z) \intinf \ispi e^{-ixz} \di x \di z \di y } {\hat f (0) f(0)} \right|
			= \left| \frac{\sqrt{2\pi} \intinf\intinf f(y)f(z) \delta(z) \di z \di y } {\hat f (0) f(0)} \right|
			= \sqrt{2\pi} \left| \frac{\intinf f(y) \di y } {\hat f (0) } \right|
			\]
			\[
			= \sqrt{2\pi} \left| \frac{\intinf f(y) \di y } {\ispi \intinf e^{-i*0*x} f(y) \di y  } \right|
			= \sqrt{2\pi}  \sqrt{2\pi} 
			= 2\pi 
			\]

		\item
			With $f(x) = e^{-x^2/2b^2}$

		Since I stared at completing the squares for way too long to justify not writing this down, here is the Fourier transform of the Gaussian:
			\[
			\hat f(\omega ) 
			= \ispi \intinf  e^{-x^2/2b^2} e^{-i\omega x} \di x 
			= \ispi \intinf  e^{-1/2b^2 \left[x^2+2b^2i\omega x\right]} \di x 
				\]

				We complete the square in the exponent:
			\[
			= \ispi \intinf  e^{-1/2b^2 \left[x^2+2b^2i\omega x  - \omega^2 b^4\right] - \omega^2 b^2/2} \di x 
			= \ispi e^{-\omega^2 b^2/2} \intinf  e^{-1/2b^2 \left[x- \omega b^2\right]^2} \di x 
			\]
			Now with $ |b| u = x - \omega b^2,  |b| \di u = \di x $ we get:
			\[
			= |b| e^{-\omega^2 b^2/2} \intinf \ispi  e^{-u^2/2} \di u 
			= |b| e^{-\omega^2 b^2/2} 
			\]

			So we have:
			$
			\hat f(\omega ) 
			= |b| e^{-\omega^2 b^2/2} 
			$
			and can now do our calculations. 
			(we could before and actually don't need this at all but I'ill be damned if I didn't spend too much time on this part to not just write something)

			The $\Delta_\omega$ is actually the famous Gaussian integral:
			\[
			\Delta_\omega 
			=
			\left| \frac{ \intinf e^{-t^2/2b^2} \di t } {1 }\right|
			=
			\left|  \intinf e^{-t^2/2b^2} \di t \right|
			= 
			\sqrt{ \frac \pi {1/2b^2}}
			=
			\sqrt{ \pi 2b^2}
			=
			|b|\sqrt{ \pi 2}
			\]

			Thanks to the relation
			$
		\Delta_t \Delta_ \omega	= 2 \pi
		$ 
		we see that $\Delta_t$ must be $1/|b| \sqrt{ 2 \pi} $. 

		I am too stuburn to not write this after the above
			\[
		\Delta_t 	
			= \left| \frac{ \intinf \hat f(t) \di t } {\hat f(0)} \right|
			= \left| \frac{ \intinf |b| e^{-\omega^2 b^2/2} \di \omega } {|b|} \right|
			= \left| \intinf  e^{-\omega^2 b^2/2} \di \omega  \right|
			\]
			\[
			= \left| \intinf  e^{-\omega^2 b^2/2} \di \omega  \right|
			= \sqrt{ 2 \pi / b^2} 
			= \frac 1 {|b|} \sqrt{ 2 \pi } 
			\]

			The inflection points are at $\pm b$ and thus its 'inflection witdth' is $2|b|$

	\end{enumerate}
\end{Porb}


\begin{Porb}
\begin{Boxed}
	Define the auto-correlation h of the funciton f:
	\[
		h(y):= \intinf f(x) f(x-y) \di x
	\]
	Compute the Fourier transform of the auto correlation funcation and show that it equals the "spectral intensity" (aka power spectrum) of f whenever f is real valued. 
\end{Boxed}
	\[
	\hat h(k)
	= \frac 1 {\sqrt{ 2\pi}} \intinf e^{-iky}  \intinf f(x) f(x-y) \di x \di y
	= \frac 1 {\sqrt{ 2\pi}} \intinf \intinf e^{-iky} f(x) f(x-y) \di x \di y
	\]
	\[
	= \frac 1 {\sqrt{ 2\pi}} \intinf f(x) \intinf e^{-ik(x-u)} f(u) \di u \di x 
	=  \intinf e^{-ikx}f(x) \hat f (-k) \di x 
	=\hat f (-k)   \intinf e^{-ikx}f(x) \di x 
	\]
	\[
	=\hat f (-k)  \hat f(k) 
	=|\hat f(k)| ^2
	\]
	The $\hat f(-k)= \overline{\hat f(k)} $ is implied by f being real valued and is the only point we make use of this fact. 
\end{Porb}



\begin{Porb}
\begin{Boxed}
	\begin{enumerate}[a)]
		\item
			Compute the total energy \[ \intinf |h(T)|^2 \di T\]
			of the cross correlation $h(T)$ in terms of the Fourier amplitudes of $f_0$ and $f$.
		\item
			*********************************
	\end{enumerate}
\end{Boxed}
************************************
\end{Porb}


\begin{Porb}
\begin{Boxed}
What functions are eigenvectors of $\FF^2$ with eigenvalue $\lambda= 1$?
\end{Boxed}
Already did this.
\end{Porb}


\begin{Porb}
\begin{Boxed}
	Let $\hat g(k) = \FF [g(x)](k)$ and $H(k)= \FF[h](k)$ be the Fourier transforms of g and h.
	Express the following in terms of $\hat g$ and $\hat f$.
	\begin{enumerate}[i)]
		\item $\FF [ \alpha g + \beta h]$ for some constants $\alpha, \beta$
		\item $\FF[ g(x- \xi)]$
		\item $\FF [ e^{i k_0 x} g]$
		\item $\FF [g(ax)]$
		\item $\FF [ \frac{d g}{dx}]$
		\item $\FF[ x g(x)]$
	\end{enumerate}
\end{Boxed}
	\begin{enumerate}[i)]
		\item 
			It is clear by the linearity of the integrals that we have:
			\[
			\FF [ \alpha g + \beta h] 
			=
			\FF [ \alpha g] +\FF[ \beta h] 
			=
			 \alpha \FF [g] +\beta \FF[ h] 
			\]

		\item 
			\[
			\FF[ g(x- \xi)]
			=
			\intinf e^{-ikx} g(x- \xi) \di x
			=
			\intinf e^{-ik(u+\xi)} g(u) \di u 
			=
			e^{-ik(\xi)} \intinf e^{-iku} g(u) \di u 
			=
			e^{-ik(\xi)} \hat g
			\]

		\item 
			\[
			\FF [ e^{i k_0 x} g]
			=
			\intinf e^{-ikx}e^{i k_0 x} g(x) \di x
			=
			\intinf e^{-i(k-k_0)x} g(x) \di x
			=
			\hat g(k-k_0)
			\]

		\item 
			\[
				\FF [g(ax)]
			=
			\intinf e^{-ikx}g(ax) \di x
			=
			\intinf \frac 1 a e^{-iku/a}g(u) \di u
			=
			\frac 1 a \hat g (k/a)
			\]

		\item 
			Using integration by parts:
			\[
			\FF [ \frac{d g}{dx}]
			=
			\intinf e^{-ikx} \frac{d g}{dx}\di x
			=
			- \intinf  g\frac{d e^{-ikx}}{dx}\di x
			=
			\intinf ik ge^{-ikx}\di x
			=
			ik \hat g(k)
			\]

		\item 
			\[
			\FF[ x g(x)]
			=
			\intinf e^{-ikx} x g \di x
			=
			\intinf x e^{-ikx}  g \di x
			=
			\intinf \left[ \frac 1 {-i} \frac d {dk}  e^{-ikx}\right]  g \di x
			\]
			\[
			=
			\frac 1 {-i} \frac d {dk} \intinf  e^{-ikx} g \di x
			=
			\frac 1 {-i} \frac d {dk} \hat g(k)
			\]
	\end{enumerate}
\end{Porb}

\begin{Porb}
\begin{Boxed}
	Show that any periodic function $f(\xi) = f(\xi +a)$ is the convolution of a nonperiodic function with a train of Dirac delta DISTRIBUTIONS.
\end{Boxed}
[I was very stuck on this and stack exchange provided an answer.]\\
Let $a>0$ be the length of the period of the function $f(x)$. Then let $g(x) = \one_{[0,a)} (x) f(x)$ have value in the 'first' period of f and then be 0 elsewhere.
Obviously g is non periodic unless $f=0$ (that case being triival and not relavent). Now consider:
\[
	g \star \left( \suminf n \delta(x-an) \right)	 = 
\suminf n g \star \delta(u-(x-an))  = 
\suminf n g(x-an) = f(x) 
\]
Thus we have written f as a convlution of a non periodic function and a train of delta distributions.
\end{Porb}


\begin{Porb}
\begin{Boxed}
	Find the Fourier specturm of a finite train of identical coherent pulses of the kind shown in Fig. 2.9.
\end{Boxed}
The function in reference is of the form:
	\[
		f_n(t) = e^{-(t-nT)^2/2b^2} e^{i\omega_0(t-nT)} e^{i\delta_n}
	\]
	Which in our specific case is:
	\[
		f_n(t) = e^{-(t-nT)^2/2b^2} e^{i\omega_0(t-nT)} 
	\]
	So our sum is:
	\[
		\sum_{n=-N}^N	f_n(t) 
		=
	\sum_{n=-N}^N	e^{-(t-nT)^2/2b^2} e^{i\omega_0(t-nT)} 
	\]
	*****************

\end{Porb}


\begin{Porb}
\begin{Boxed}
	Verify that
	\[
		f(t) =
		\suminf n 
		e^{-(t-nT)^2/2b^2} e^{i\omega_0(t-nT)} 
		\]
		is a periodic function of t, and that $f(t+T)=f(t)$.
		Find the full 4-ier representation
		\[
			f(t) = \suminf m c_m e^{i\omega_m t}
			\]
			of f by determing $\omega_m$ and $c_m$.
\end{Boxed}
*********************
\end{Porb}





\subsection{Orthonormal Wave Packet Representation}
\setcounter{Prb}{0}

\begin{Porb}
\begin{Boxed}
	Consider the set of functions:
	\[
		\left\{
			P_{jl}(t) = \frac 1 {\sqrt \eps} \int_{j\eps}^{(j+1)\eps} e^{2\pi i l \omega/\eps} 	
			\ispi e^{-i\omega t} \di \omega
			, \quad j,l = 0,\pm1, \pm 2 , \dots
			\right\}
		\]
	\begin{enumerate}[a)]
		\item Show that these wave packets are orthonormal
		\item Show that these wave packets form a complete set.
	\end{enumerate}
\end{Boxed}

\end{Porb}


\begin{Porb}
\begin{Boxed}
	Consider the wave packet
	\[
		Q_{jl}(t) = \ispi \frac 1 {\sqrt \eps} \int_{(j-1/2)\eps}^{(j+1/2)\eps}
		e^{i \omega t} e^{-2 \pi i l \omega/\eps} \di \omega
		\]
		Express the summed wave packets:
	\begin{enumerate}[a)]
	\item 
		\[
			\suminf j Q_{jl} (t)
			\]
	\item
		\[
			\suminf l Q_{jl} (t)
			\]
	\item
		\[
			\suminf l\suminf j Q_{jl} (t)
			\]
	\end{enumerate}
	in terms of appropriate Dirac delta DISTRIBUTIONS if necessary.
\end{Boxed}
	\begin{enumerate}[a)]
	\item 
		\[
			\suminf j Q_{jl} (t)
			=
\ispi \frac 1 {\sqrt \eps} 	\suminf j	\int_{(j-1/2)\eps}^{(j+1/2)\eps}
		e^{i \omega t} e^{-2 \pi i l \omega/\eps} \di \omega
			=
\ispi \frac 1 {\sqrt \eps} \intinf	e^{i \omega t} 
			e^{-2 \pi i l \omega/\eps} \di \omega
			\]
			\[
			=
			 \frac 1 {\sqrt \eps} \intinf	\ispi e^{i \omega (t-2 \pi l/ \eps)} \di \omega
			=
			 \frac 1 {\sqrt \eps} \sqrt{2\pi} \delta ( t-2 \pi l/ \eps)
			\]
	\item
		\[
			\suminf l Q_{jl} (t)
			=
	\suminf l  \ispi \frac 1 {\sqrt \eps} \int_{(j-1/2)\eps}^{(j+1/2)\eps}
		e^{i \omega t} e^{-2 \pi i l \omega/\eps} \di \omega
			=
	 \ispi \frac 1 {\sqrt \eps} \int_{(j-1/2)\eps}^{(j+1/2)\eps}
		e^{i \omega t} \suminf l e^{-2 \pi i l \omega/\eps} \di \omega
			\]
			Now letting $y= -2 \pi \omega/\eps, \di y = -2\pi /\eps \di \omega$, we can change variables and evaluate the sum:
			\[
			=
	  \frac 1 {\sqrt \eps} \int_{(j-1/2)\eps}^{(j+1/2)\eps}
			e^{i \omega t} \sqrt{2 \pi}  \suminf l e^{i l(-2 \pi  \omega/\eps) } \di \omega
			=
			\frac 1 {\sqrt \eps} \int_{(j-1/2)(-2\pi)}^{(j+1/2)(-2\pi)}
			e^{i \eps y/(-2\pi) t} \sqrt{2 \pi}  \suminf l e^{i ly } \di y  \frac \eps {-2\pi }
			\]
			\[
			=
		\frac 1 {-2\pi }\sqrt \eps \int_{(j-1/2)(-2\pi)}^{(j+1/2)(-2\pi)}
			e^{i \eps y/(-2\pi) t} \sqrt{2 \pi}  \delta(y) \di y  
			=
		\frac 1 {-2\pi }\sqrt \eps \sqrt{2 \pi}
			\sum_{|j|\leq 3}
			\delta_{j,0}
			\]
			Since $0\in [j-\frac 1 2, j +\frac 1 2]$ iff $j=0,\pm1,\pm2,\pm3$.

	\item
		By the first part we immediatly see:
		\[
			\suminf l\suminf j Q_{jl} (t)
			=
			\suminf l
			 \frac 1 {\sqrt \eps} \sqrt{2\pi} \delta ( t-2 \pi l/ \eps)
			=
			\sqrt{ \frac {2\pi} {\eps} }
			\suminf l
			 \delta ( t-2 \pi l/ \eps)
		\]



	\end{enumerate}
	*******************DOUBLE CHECK
\end{Porb}



\subsection{Orthonormal Wavelet Representation}

\subsection{Multiresolutions Analysis}
\setcounter{Prb}{0}

\begin{Porb}
\begin{Boxed}
	Show that
	\[
		\overline{ \bigcup_{k=-\infty}^\infty V_k} = L^2 \iff
		\lim_{k\rw \infty} \norm{ P_{V_k} f - f } =0 
	\]
	where $P_{V_k}$ is the projection onto $V_{-k}$ (the sign is fliped to make the limits easier to write) and the norm is the $L^2$ norm.
\end{Boxed}
We go forward first:
	\[
		\overline{ \bigcup_{k=-\infty}^\infty V_k} = L^2
		\]
		Thus given an $f\in L^2, \exists h_n,\ st.$ $\lim_{n \rw \infty} \norm{h_n -f} =0, h_n \in \bigcup V_k$.
		Now $\forall n$ $\exists k \ st. h_n \in V_k$. Now either $h_n = P_{V_k} f$ or $\norm{ h_n - f} \geq \norm{P_{V_k} -f}$ and we can replace $h_n$ with the actual projection without making the approximation any worse. 
		Obviously our new $\bar h_n$ still converges and is made entirly of projections onto subspaces. Thus we have constructed the desired sequence. (We may need to additonally doctor the sequence and insert terms if $h_k$ skipped many subspaces.)
		***********************HATE HOW THIS IS WRITTEN************************8
	
Now we go backwards:
	Let f again be some funciton in $L^2$, then $\norm{ P_{V_k} f -f } \rw 0 $ as $ k \rw \infty$. Thus there exists some sequence $h_k = P_{V_k} f$ where $\norm{h_k -f } \rw 0$ as $k \rw \infty$. 
	Additionally $h_k\in V_k \forall k$ thus $\forall f \in L^2, f \in \overline{ \bigcup V_kk}$.
	The reverse inclusion is obvious and we are done. 
\end{Porb}



\begin{Porb}
\begin{Boxed}
	Show that
	\[
		\bigcap_{k=-\infty}^\infty V_k =\{0\} \iff
		\lim_{k\rw \infty} \norm{ P_{V_k} f } =0 
	\]

\end{Boxed}
First we go forward:\\
	Notice that $\norm{P_{V_k} f}$ is a decreasing sequence and thus has some limit. Now suppose for controdiciton that
	$\norm{P_{V_k} f} \rw \eps >0$ as $k \rw \infty$.
	Notice now that this is a Cauchy sequence in $L^2$ and thus there is some function $g \in L^2$ st. 
	$ \norm{P_{V_k} f -g } \rw 0$.
	Now we wish to show that $g \in V_k, \forall k$. Suppose it is not, there is some $k_0$ st. g is no longer in any of the $V_k$'s after $k_0$.
	But then (since $\overline{\bigcap V_k} = \{0\}$) there would be some nonzero gap that emerges between $P_{V_k}f$ and g, namely that:
	$ \norm{P_{V_k} f -g } \geq d(V_k, g) = \eps_g >0$.
	Thus $g\in V_k, \forall k$ but then $g \in \bigcap V_k$ which then means $g=0$ and thus 
	$ \norm{P_{V_k} f } \rw \norm g =0$.



Then we go back:\\
We do this by controdiciton, so suppose that 
		$\bigcap_{k=-\infty}^\infty V_k \supseteq \{0, g\} $ for some nonzero functions $g$. 
		Then $\norm{g} >0$ and since $g\in \bigcap_{k=-\infty}^\infty V_k, \rw g \in V_k \forall k$. 
		Thus $g \in P_{V_k}$ for all k and $\lim_{k \rw \infty} \norm{ P_{V_k} g} =\norm g >0$.
		This is a controdicition and we see that there is no g.
\end{Porb}

\begin{Porb}
\begin{Boxed}
	\begin{enumerate}[a)]
		\item Show that $V_0$ is discrete translation invarient, ie. whenever $l \in \ZZ$  that:
			\[
				f(t) \in V_0 \iff f(t-l) \in V_0
				\]
		\item Show that $V_k$ is $2^k$ shift invariant, ie with $l,k \in \ZZ$ that:
			\[
				f(t) \in V_0 \iff f(t-2^kl) \in V_0
				\]
	\end{enumerate}
\end{Boxed}
	\begin{enumerate}[a)]
		\item 		
			Suppose $f \in V_0$ then $\exists \alpha_l$ st. $f(t) = \sum_l \alpha_l \phi(t-l)$.
			By the construction of the basis of $V_0$.
			Notice that for $k \in \ZZ$
			\[
				f(t-k) = \sum_l \alpha_l \phi(t-k-l)
				= \sum_{m= k+l} \alpha_{m-k} \phi(t-m)
				= \sum_{m= k+l} \alpha_m' \phi(t-m)
				\]
				Thus we still have an expansion for $f(t-l)$ in terms of the original basis.
		\item Similar tricks:
			\[
				f(t) \in V_k \Rw 
				f(2^kt) \in V_0 \Rw 
				\]
				Now we remember that shifting by a constant value keeps you in $V_0$.
				$
					f(2^k(t-j)) = f(2^kt-2^kj) \in V_0
				$\\
				Now scaling the $t$ by $2^{-k}$ will get us back to $V_k$, that is: $f(t - 2^kj) \in V_k$.
	\end{enumerate}
\end{Porb}




















\begin{Porb}
\begin{Boxed}
	zt
\end{Boxed}
\end{Porb}

\section{Strum-Liouville Theory}
\setcounter{subsection}{2}
\subsection{Strum-Liouville Systems}
\setcounter{Prb}{0}


\begin{Porb}
\begin{Boxed}
	\begin{enumerate}[a)]
		\item Show that any equation of the form
	\[
		u''+ b(x) u'+ c(x) u=0
	\]
		can always be brought into the Shrodinger form:
		\[
			v'' + Q(x) v=0
		\]
	Apply this result to obtain the Schrodinger form for:
	\item 
		\[
			u''-2xu' +\lambda u=0
		\]
	\item 
		\[
			x^2u''+xu'+(x^2-\nu^2)u=0
		\]
	\item 
		\[
			xu''+(1-x)u'+\lambda u=0
		\]
	\item 
		\[
			(1-x^2)u''- xu'+\alpha^2 u=0
		\]
	\item 
		\[
			(pu')'+ (q+\lambda r) u=0
		\]
	\item 
		\[
			\left[ \frac 1 {\sin \theta} \frac \di {\di \theta} \sin \theta  \frac \di {\di \theta} 
			+ l (l+1) - \frac {m^2} {\sin^2\theta} 
			\right]u =0
		\]

	\end{enumerate}
	
\end{Boxed}
\end{Porb}

\begin{Porb}
\begin{Boxed}
	Consider the S-L eigenvalue problem:
	\[
		[Lu_n] (x) = \left( - \frac{ d^2} { dx^2} + x^2\right)u_n(x) = \lambda_n u_n(x), \quad \lim_{x\rw \pm \infty} u(x)= 0
		\]
	Show that the eigenvalues $\lambda_n$ are nondegenerate, ie. show that, except for a constant multiple, the correpsonding eigenfunctions are unique.
\end{Boxed}
*************************************************
\end{Porb}


\begin{Porb}
\begin{Boxed}
	zt
\end{Boxed}
\end{Porb}

\begin{Porb}
\begin{Boxed}
	zt
\end{Boxed}
\end{Porb}

\begin{Porb}
\begin{Boxed}
	zt
\end{Boxed}
\end{Porb}

\begin{Porb}
\begin{Boxed}
	zt
\end{Boxed}
\end{Porb}

\begin{Porb}
\begin{Boxed}
	zt
\end{Boxed}
\end{Porb}



\section{Green's Function Theory}
\setcounter{subsection}{2}
\subsection{Pictorial Definition of a Green's Function}
\setcounter{Prb}{0}

\begin{Porb}
\begin{Boxed}
	\begin{enumerate}[a)]
		\item 
		\item
	\end{enumerate}
\end{Boxed}
\end{Porb}


\setcounter{subsection}{2}
\subsection{The Totally Inhomogeneous Boundary Value Problem}
\setcounter{Prb}{0}

\begin{Porb}
\begin{Boxed}
	Let $L = -\frac{d^2}{dx^2}$ with boundary
\end{Boxed}
\end{Porb}












\begin{Porb}
\begin{Boxed}
	zt
\end{Boxed}
\end{Porb}

\end{document}














