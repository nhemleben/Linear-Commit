
\documentclass[12pt]{article} 
\usepackage{esvect, graphicx,relsize, verbatim, amssymb, amsmath, amsthm, mathabx,dcolumn,mathrsfs, dsfont, enumerate, soul, titlesec}
\newcolumntype{2}{D{.}{}{2.0}}
\textwidth = 7 in
\textheight = 9.5 in
\oddsidemargin = -0.3 in
\evensidemargin = -0.3 in
\topmargin = -0.4 in
\headheight = 0.0 in
\headsep = 0.0 in
\parskip = 0.2in
\parindent = 0.0in
\titlespacing*{\section}
{0pt}{0ex plus .1ex minus .2ex}{.1ex plus .1ex}
\titlespacing*{\subsection}
{0pt}{.5ex plus .1ex minus .2ex}{.3ex plus .2ex}
\DeclareMathOperator{\Gal}{Gal}	
\DeclareMathOperator{\gal}{Gal}	
\DeclareMathOperator{\im}{Image}
\DeclareMathOperator{\Ann}{Ann}	
\DeclareMathOperator{\spn}{span}	
\DeclareMathOperator{\sgn}{sgn}	
\DeclareMathOperator{\sinc}{sinc}	
\newcommand{\CC}{\mathbb{C}}	
\newcommand{\GG}{\mathbb{G}}	
\newcommand{\LL}{\mathbb{L}}	
\newcommand{\NN}{\mathbb{N}}	
\newcommand{\PP}{\mathbb{P}}	
\newcommand{\QQ}{\mathbb{Q}}	
\newcommand{\RR}{\mathbb{R}}	
\newcommand{\ZZ}{\mathbb{Z}}
\newcommand{\FF}{\mathfrak{F}}	
\newcommand{\one}{\mathds{1}}
\newcommand{\eps}{\epsilon}
\newcommand{\floor}[1]{\lfloor #1 \rfloor}
\newcommand{\s}[1]{\sqrt{#1}}
\newcommand{\norm}[1]{\left\lVert#1\right\rVert}
\newcommand{\I}[1]{#1^{-1}}
\newcommand{\Rw}{\Rightarrow}
\newcommand{\Lw}{\Leftarrow}
\newcommand{\lw}{\leftarrow}
\newcommand{\rw}{\rightarrow}
\newcommand{\p}{\partial}
\newcommand{\var}{\mathrm{Var}}
\newcommand{\sumi}{\sum_{i=0}}
\newcommand{\sumo}{\sum_{i=1}}
\newcommand{\num}[1]{{\large \bf  #1)}}
\newcommand{\del}{\Delta}
\newcommand{\lamb}{\lambda}	
\newcommand{\al}{\alpha}	
\newcommand{\nab}{\nabla}	
\newcommand{\pp}[2]{\frac{\p #1}{\p #2}}
\newcommand{\di}{\mathrm{d}}
\newcommand{\dd}[2]{\frac{\di #1}{\di #2}}
\newcommand{\ddt}[2]{\frac{\di^2 #1}{\di #2 ^2}}
\newcommand{\intinf}{\int\limits_{-\infty}^\infty}	
\newcommand{\suminf}[1]{\sum_{#1 = -\infty}^\infty}	
\newcommand{\ispi}{\frac 1 {\sqrt{2 \pi}} }
\newtheorem{thm}{Theorem}
\theoremstyle{definition}
\newtheorem{defi}{Definition}[section]
\newtheorem{lem}{Lemma}[section]
\newcounter{Prb}%[subsection]
\newenvironment{Porb}[1][]{
	\refstepcounter{Prb}\par\medskip 
   \noindent \textbf{\thesubsection~Problem~\thePrb. 
   #1
   }
		       }
   {\medskip \hrule}
   %Just need one hline at the end [otheriwse double thicc lines
\newenvironment{Boxed}{\begin{center}
	           \begin{tabular}{|p{0.9\textwidth}|}
			       \hline\\
	           }
	       { 
           \\\\\hline
	       \end{tabular} 
           \end{center}
		       }
%So basicly the problem envorment does not want ot box stuff 
%To combat this we use 2 enviroments, the first being Problem
%the second being boxed
%Problem goes in box
%Solution goes outside box
%Hlines for everyone and the world is happy
%%%%% End of preamble %%%%%
\begin{document}
\title{Linear Math In Infinte Dimensions}
{\Large Nicholas Hemleben} \hfill
{\large Linear Math In Infinte Dimensions}  
\hfill  \today

\tableofcontents
\clearpage
%\setcounter{Prb}{-1}
%		\begin{Porb}
%		\begin{Boxed}
%			A Demo porblem
%			Inside a Box
%		\end{Boxed}
%			Outside a Box
%		\end{Porb}
\section{Infinite Dimensional Vector Spaces}
\setcounter{subsection}{2}
\subsection{Metric Spaces}
%THe first hrule
\hrule
		\begin{Porb}
		\begin{Boxed}
			Show that a) the hamming distance, b) the Pythagorean distance and c) the Chebyshev distance each satisfy the triangle inequality.
		\end{Boxed}
			\begin{enumerate}[a)]
		\item
			Hamming distance is defined by 
				\[
					d(x,y) = 
					\sum_i |x_i - y_i | 
				\]
			We see that:
				\[
					\sum_i |x_i - y_i | 
					=
					\sum_i |x_i - z_i + z_i - y_i | 
					\leq 
					\sum_i |x_i - z_i | + | z_i - y_i | 
				\]
					\[
					\sum_i |x_i - z_i | + 
					\sum_i| z_i - y_i | 
					= 
					d(x,z) 
					+d(y,z) 
					\]

		\item
Pythagorean distance
				\[
					d(x,y) = 
					\sqrt{ \sum_i (x_i - y_i )^2 }
				\]

		\[
		\sqrt{ \sum_i (x_i - y_i )^2 }
		=
		\sqrt{ \sum_i (x_i- z_i + z_i  - y_i )^2 }
		\]
		\[
			\leq
			\sqrt{ \sum_i (x_i- z_i)^2 + ( z_i  - y_i )^2 }
			=
			\sqrt{ \sum_i (x_i- z_i)^2 + \sum_i ( z_i  - y_i )^2 }
		\leq 
		\sqrt{ \sum_i (x_i- z_i)^2} + \sqrt{\sum_i ( z_i  - y_i )^2 }
		\]
%		which finishes the proof

		\item
			The Chebyshev distance
				\[
					d(x,y) = 
					\max \left \{ |x_i - y_i|  \right \}
				\]
				\[
					\max \left \{ |x_i - y_i|  \right \}
					= 
					\max \left \{ |x_i- z_i + z_i  - y_i|  \right \}
					\leq
					\max \left \{ |x_i- z_i|  + | z_i  - y_i|  \right \}
				\]
				\[
					\leq
					\max \left \{ |x_i- z_i| \right \}  + \max \left \{  | z_i  - y_i|  \right \}
				\]
			\end{enumerate}
		\end{Porb}
\setcounter{subsection}{4}
\setcounter{Prb}{0}
\subsection{Hilber Spaces}
\begin{Porb}
\begin{Boxed}
{\large 
Show that 
\vspace{-2mm}
\[
\langle f , g \rangle = \sum_{k=1}^\infty \bar c_k d_k 
\]
}
with
\[
\langle f , g \rangle = \int_a^b \bar f (x) g(x) \rho(x) \di x
\]
\end{Boxed}
Since we have that $f(x) = \sum_{k=1}^\infty u_k(x) c_k$ we can substitiute and get:
\[
= \int_a^b \overline { \sum_{k=1}^\infty u_k(x) c_k } g(x) \rho(x) \di x
=  \sum_{k=1}^\infty \int_a^b \bar u_k(x) \bar c_k  g(x) \rho(x) \di x
\]
\[
=  \sum_{k=1}^\infty \int_a^b \bar u_k(x) \bar c_k  g (x)\rho (x)\di x
=  \sum_{k=1}^\infty \bar c_k \int_a^b \bar u_k(x)  g (x)\rho (x)\di x
= \sum_{k=1}^\infty \bar c_k d_k 
\]
\end{Porb}


\begin{Porb}
\begin{Boxed}
\[
T f(\omega, t) = 
\int_{-\infty}^\infty \bar g(x-t) e^{-i\omega x} f(x) \di x
\]

\[
\langle h_1, h_2 \rangle = \intinf \intinf \bar h_1(\omega,t) h_2(\omega,t) \di \omega \di t
\]
Find a formula for:
$\langle Tf_1, Tf_2 \rangle $
in terms of 
\[
\intinf \bar f_1 f_2 \di x
\]

\end{Boxed}
\[
\langle Tf_1, Tf_2 \rangle 
=
\intinf \intinf 
\overline{Tf_1} Tf_2  \di \omega \di t
=
\]

\[
\intinf \intinf 
\overline{
\left[
\int_{-\infty}^\infty \bar g(x-t) e^{-i\omega x} f_1(x) \di x
\right]
}
\left[
\int_{-\infty}^\infty \bar g(x-t) e^{-i\omega x} f_2(x) \di x
\right]
\di \omega \di t
\]

\[
=
\intinf \intinf 
\intinf \intinf 
\overline{
\bar g(x-t) e^{-i\omega x} f_1(x) 
}
\bar g(y-t) e^{-i\omega y} f_2(y) 
\di x \di y
\di \omega \di t
\]

We now return to Calc III and need to do a replacment of variables:
\[
\begin{matrix}
u = x-y,& v =y \\
x = u+v,& y =y 
\end{matrix}
\text{ which has deterimnate:\ }
J = 
\left|
\begin{bmatrix}
1 & -1 \\
0 & 1 
\end{bmatrix}
\right|
=1
\]

\[
=
\intinf \intinf 
\intinf  
\overline{
\bar g(u+v-t)  f_1(u+v) 
}
\bar g(v-t)  f_2(v) 
\intinf
e^{i\omega u}
\di \omega 
\di u \di v
\di t
\]

Subbing in the following identity for $\delta$
\[
\intinf
e^{i\omega u}
\di \omega 
= \delta( u)
\]
we see:
\[
=
\intinf \intinf 
\intinf  
\overline{
\bar g(u+v-t)  f_1(u+v) 
}
\bar g(v-t)  f_2(v) 
\delta(u)
\di u \di v
\di t
\]

\[
=
\intinf \intinf 
\overline{
\bar g(v-t)  f_1(v) 
}
\bar g(v-t)  f_2(v) 
 \di v
\di t
=
\intinf 
\bar  f_1(v)  f_2(v) 
 \intinf g(v-t)\bar g(v-t) 
\di t
 \di v
\]
\[
=
\intinf 
\bar  f_1(v)  f_2(v) 
|g|^2
 \di v
=
|g|^2
\intinf \bar f_1 f_2 \di x
= |g|^2 \langle f_1, f_2 \rangle
\]

\end{Porb}
\begin{Porb}
\begin{Boxed}
\begin{enumerate}[i)]
	\item
	Show that the set of functions
	\[
		\left\{
			\frac{ \sin \pi ( 2 \omega z - k) } { \pi( 2 \omega z -k)} = \sinc(2 \omega z-k), \ k \in \ZZ
			\right\}
		\]
		is an orthognal set satisfying:
		\[
			\intinf \sinc(2 \omega z -k) \sinc( 2 \omega z -l) \di z = A \delta_{kl}
		\]
			What is A?
			To show orthogonality follow this 3 step outline:
\begin{enumerate}[a)]
	\item
		\[
			\int_{-\pi}^\pi \delta_N \left( t - \frac{ 2\pi} {2N+1}k \right)
	\delta_N \left( t - \frac{ 2\pi} {2N+1}l \right) \di t
	=	
		\frac{2N+1}{2 \pi} \delta_{kl}
		\]
		where 
		\[
			\delta_N(u) = \frac 1 {2\pi} \frac{ \sin( N+\frac 1 2)u } { \sin \frac u 2} = \frac 1 {2\pi} \sum_{n=-N}^N e^{inu}
			\]
	\item
		Then rescaling the integration domain by using $z= \frac{ N+ \frac 12 } { 2\pi \omega} t$.


	\item
		and finally going to the limit $N \rw \infty$.
\end{enumerate}

\item
	This set of functions 
		\[
			\left\{
				u_k = \frac 1 {\sqrt A} \sinc ( 2 \omega z - k), \ k \in \ZZ
			\right\}
		\]
		is not complete on $L^2$, but is complete on a specific subset. What is this subset, ie what property must a function $f(t)$ satisfy in order to be in this subset?
\end{enumerate}
\end{Boxed}
\begin{enumerate}[i)]
	\item
		If we take $k=0=l$ then we can calculate A:
		\[
			 A = \intinf \sinc(2 \omega z) \sinc( 2 \omega z) \di z  		
			 =
			 \intinf \sinc(2 \omega z)^2  \di z  		
			 =
			 \intinf   		
		\left( \frac{ \sin \pi ( 2 \omega z) } { \pi 2 \omega z)} \right)^2 
		\di z 
		\]
		Now with $u= 2\pi \omega z$, $\di u = 2\pi \omega \di z$ we can change variables and see:
		\[
			 =
			 \intinf   		
		\left( \frac{ \sin (u) } {u} \right)^2 
		\frac 1 {2 \pi\omega} \di u 
			 =
		\frac 1 {2\pi \omega} 
		 \intinf   		
		\left( \frac{ \sin (u) } {u} \right)^2 
		\di u 
		\]
		Now we can apply integration by parts with
		\[
			\begin{cases}
				u = \sin ^2 x & \di u = 2 \sin x \cos x = \sin (2x) \\
				v= - \I x     & \di v = - x^{-2}
			\end{cases}
			\]
		and arrive at:
		\[
			 =
		\frac 1 {2\pi \omega} 
		\left[
			\frac{ -\sin ^2 x } x |_{\pm \infty}
			-
		 \intinf   		
		 \frac{ \sin (2x)} {-x} 
		 \di x
		\right]
			 =
		\frac 1 {2\pi \omega} 
		\left[
		 \intinf   		
		 \frac{ \sin (2x)} {x} 
		 \di x
		\right]
		\]

		With another change of variables $ 2x \rw x$, and noticing that $sinc(x)$ is an even function we get:
		\[
		\frac 1 {2 \pi \omega} 
		 \intinf   		
		 \frac{ \sin (x)} {x} 
		 \di x
		 =
		\frac 1 {\pi \omega} 
		 \int_0^\infty  		
		 \frac{ \sin (x)} {x} 
		 \di x
		 =
		\frac 1 {\pi \omega} 
		\frac \pi 2
		 =
		 \frac 1 2 \I \omega
		 =
		 A
		\]

\begin{enumerate}[a)]
	\item
		Let $u_k = t - \frac{ 2\pi} {2N+1}k $, then
		\[
			\int_{-\pi}^\pi \delta_N \left( t - \frac{ 2\pi} {2N+1}k \right)
	\delta_N \left( t - \frac{ 2\pi} {2N+1}l \right) \di t
	=	
			\int_{-\pi}^\pi \delta_N \left(u_k \right)
	\delta_N \left(u_l \right) \di t
		\]
		\[
	=	
			\int_{-\pi}^\pi
			\left(
 \frac 1 {2\pi} \sum_{n=-N}^N e^{inu_k}
		\right) \left(	
 \frac 1 {2\pi} \sum_{n=-N}^N e^{inu_l}
		\right)
		\di t
		\]
		\[
			=
			\int_{-\pi}^\pi
 \frac 1 {4\pi^2} \sum_{n_1=-N}^N \sum_{n_2=-N}^N 
		e^{i(n_1u_k+ n_2u_l)}
		\di t
			=
 \frac 1 {4\pi^2} 
\sum_{n_1=-N}^N \sum_{n_2=-N}^N 
\int_{-\pi}^\pi
		e^{i(n_1u_k+ n_2u_l)}
		\di t
		\]
		Considering now just the integral:
		\[
\int_{-\pi}^\pi
		e^{i(n_1u_k+ n_2u_l)}
		\di t
		=
\int_{-\pi}^\pi
		\exp\left\{
			i\left(
		n_1\left[t - \frac{ 2\pi} {2N+1}k\right]
		+
		n_2\left[t - \frac{ 2\pi} {2N+1}l\right]
			\right)
			\right\}
		\di t
		\]

		\[
		=
\int_{-\pi}^\pi
		\exp\left\{
		i (n_1 + n_2) t 		
		-
i\left[
		n_1\frac{ 2\pi} {2N+1}k
		+
n_2\frac{ 2\pi} {2N+1}l\right]
			\right\}
		\di t
		\]

		If we let $C_{n_1n_2lk}= 
		\exp\left\{
		-i\frac{ 2\pi} {2N+1}\left[
		n_1k+
n_2l\right]	\right\}
			$
			then we see get:
		\[
		=
		\frac {C_{n_1n_2 lk}} {i(n_1+n_2)}
		\exp\left\{
			i (n_1 + n_2) t \right\}
		|_{t = \pm \pi }
		=
	C_{n_1n_2 lk} 
		\delta_{-n_1,n_2} 2\pi
		\]
		Plugging this back into the above we get:
		\[
			=
 \frac 1 {2\pi} 
\sum_{n_1=-N}^N \sum_{n_2=-N}^N 
		C_{n_1n_2 lk}
		\delta_{-n_1,n_2}
			=
 \frac 1 {2\pi} 
\sum_{n=-N}^N  C_{n,-n lk}
=
 \frac 1 {2\pi} 
\sum_{n=-N}^N  
		\exp\left\{
		-i n \frac{ 2\pi} {2N+1}\left[
		k-l\right]	\right\}
\]

\[
=
 \frac 1 {2\pi} 
\sum_{n=-N}^N  
		\exp\left\{
		-i n \frac{ 2\pi} {2N+1}\left[
		k-l\right]	\right\}
\]
Clearly when $k=l$ then we get $
 \frac 1 {2\pi} 
\sum_{n=-N}^N  1 = \frac { 2N+1} {2\pi}$.
Now for $k \neq l$ we see that this is just 
\[
	\delta_N(\frac{2\pi} { 2N+1} (k-l)) 
	=
	\frac 1 {2\pi} \frac{ \sin(\pi (k-l)) } { \sin \frac \pi {2N+1} (k-l)} 
	\]
	Notice that the denominator is never zero and that the top always is. Thus we arrive at:
	\[
			\int_{-\pi}^\pi \delta_N \left( t - \frac{ 2\pi} {2N+1}k \right)
	\delta_N \left( t - \frac{ 2\pi} {2N+1}l \right) \di t
	=	
	\frac{2N+1} {2\pi} \delta_{kl}
	\]
	\item
		Now we wish to make the change of variable $z = \frac{N+\frac 1 2}{ 2 \pi \omega} t$.
	\[
		\int_{-\frac{N+\frac 1 2}{ 2  \omega} }^{\frac{N+\frac 1 2}{ 2 \omega}}
	\delta_N \left( \frac{ 2\pi} {2N+1} (2z\omega - k) \right) 
	\delta_N \left( \frac{ 2\pi} {2N+1} (2z\omega - l) \right) 
 	\frac{ 4 \pi \omega}{2N+ 1 } \di z
	=	
	\frac{2N+1} {2\pi} \delta_{kl}
	\]
	\[
		\int_{-\frac{N+\frac 1 2}{ 2  \omega} }^{\frac{N+\frac 1 2}{ 2 \omega}}
	\delta_N \left( \frac{ 2\pi} {2N+1} (2z\omega - k) \right) 
	\delta_N \left( \frac{ 2\pi} {2N+1} (2z\omega - l) \right) 
	\di z  
	=	
        \frac 1 2 \left( \frac{2N+1} {2\pi} \right)^2   \I \omega \delta_{kl}
	\]
	\[
	\delta_N \left( \frac{ 2\pi} {2N+1} (2z\omega - k) \right) 
	=
	\frac 1 {2\pi}
	\frac{ 
		\sin\left(( N+\frac 1 2) 
		 \frac{ 2\pi} {2N+1} (2z\omega - k) \right) 
		}
	{
		\sin \left( \frac{ 2\pi} {2N+1} (2z\omega - k) \right) /2	
	}
	\]
	\[
	=
	\frac 1 {2\pi}
	\frac{ 
		\sin\left(\pi (2z\omega - k) \right) 
		}
	{
		\sin \left( \frac{1} {2N+1} \pi(2z\omega - k) \right) 	
	}
		\]
	\item
		Now since we are taking the limit as $N \rw \infty$ we can ignore all higher terms in the $\sin$ expansion and just leave the linear factor.
	\[
	\sim
	\frac 1 {2\pi}
	\frac{ 
		\sin\left(\pi (2z\omega - k) \right) 
		}
	{
		 \frac{1} {2N+1} \pi(2z\omega - k)  	
	}
		\]
	Putting this back into our integral equation yields:
	\[
		\int_{-\frac{N+\frac 1 2}{ 2  \omega} }^{\frac{N+\frac 1 2}{ 2 \omega}}
	\frac 1 {2\pi}
	\frac{ 
		\sin\left(\pi (2z\omega - k) \right) 
		}
	{
		 \frac{1} {2N+1} \pi(2z\omega - k)  	
	}
	\frac 1 {2\pi}
	\frac{ 
		\sin\left(\pi (2z\omega - l) \right) 
		}
	{
		 \frac{1} {2N+1} \pi(2z\omega - l)  	
	}
	\di z  
	=	
        \frac 1 2 \left( \frac{2N+1} {2\pi} \right)^2   \I \omega \delta_{kl}
	\]
 With some simplificiation gives us:
	\[
		\int_{-\frac{N+\frac 1 2}{ 2  \omega} }^{\frac{N+\frac 1 2}{ 2 \omega}}
	\frac{ 
		\sin\left(\pi (2z\omega - k) \right) 
		}
	{
		  \pi(2z\omega - k)  	
	}
	\frac{ 
		\sin\left(\pi (2z\omega - l) \right) 
		}
	{
		  \pi(2z\omega - l)  	
	}
	\di z  
	=	
        \frac 1 2 \I \omega \delta_{kl}
	\]

	Now taking limits we get the desired equality and see that A was in fact $\frac 1 {2 \omega}$
	\[
		\int_{-\infty}^{\infty}
	\frac{ 
		\sin\left(\pi (2z\omega - k) \right) 
		}
	{
		  \pi(2z\omega - k)  	
	}
	\frac{ 
		\sin\left(\pi (2z\omega - l) \right) 
		}
	{
		  \pi(2z\omega - l)  	
	}
	\di z  
	=	
        \frac 1 2 \I \omega \delta_{kl}
	\]
	Thus the set is in fact an orhtogonal set satisfying the above.
\end{enumerate}

	\item Limited bandwith functions????
    ************************************

\end{enumerate}
\end{Porb}



\section{Chapter 2}
\subsection{Subsection 1}
\setcounter{Prb}{0}

\begin{Porb}
\begin{Boxed}
Suppose $f(x) =f(x+2\pi) \ \forall x$ is periodic with period $2\pi$.
Show 
\[
\int_a^{2\pi +a} f(x) \di x
=
\int_0^{2\pi} f(x) \di x
, \ \forall a \in \RR
\]
\end{Boxed}
As all great math proofs, no words are needed just equalities and beautiful integrals.
Let a be given then:
\[
\int_a^{2\pi +a} f(x) \di x
=
\int_a^{2\pi} f(x) \di x +
\int_{2\pi}^{2\pi +a} f(x) \di x
=
\int_a^{2\pi} f(x) \di x +
\int_{0}^{a} f(x+2\pi) \di x
\]
\[
=
\int_a^{2\pi} f(x) \di x +
\int_{0}^{a} f(x) \di x
=
\int_0^{2\pi} f(x) \di x
\]

\end{Porb}
\begin{Porb}
\begin{Boxed}
	{\bf Dirichelet Basis}
	\[
		W_{2N+1} = \text{span} \left\{ \frac 1 {\sqrt{2\pi} } e^{ikt} \right\} _{k=\pm N}
	\]
	Consider the set 
	\[
g_k(t) = \frac {2 \pi} { 2 N+1} \delta_N (t-x_k) = 
\frac 1 {2N+1} \sum_{n=-N}^N e^{in(t-k\pi / (N+\frac 1 2) } 
	\]
	Show that
	\begin{enumerate}[A)]
		\item $B = \{ g_k , k \in 1,2, \cdots \}$ is linearly independent.
		\item B spans $W_{2N+1}$
	\end{enumerate}
\end{Boxed}
	\begin{enumerate}[A)]
	\item
		It suffices to notice that $g_k(x_l) = \delta_{kl}$. Thus we can see for any given k that $g_k$ is independent of all the other's as $\sum_{k' \neq k} \lambda_{k'} g_{k'} (x_k) = 0$. Thus we can not have a non trivial linear relationship between the functions. 
	\item
		It is clear that $g_k(t) \in W_{2N+1}$ since each of the elements in its sum namely $e^{in(t-k\pi/(N+\frac 1 2)}$ is just a multiple of $e^{int}$ a basis element of $W_{2N+1}$.
		Notice there are $2N+1$ of these independnet vectors in the vector space of dimmension $2N+1$. 
			Thus they must be a spanning set and there must exist coefficents for any function in the space to be written as a sum of this basis.

		To actually exhibit coefficents one would use $f(t) = \sum_k f(x_k) g_k(t)$. 
		
	\end{enumerate}

\end{Porb}
\begin{Porb}
\begin{Boxed}
	{\bf Riemann-Lebesgue Lemma}\\
$G(u)$ piecewise continuous and has left and right derivatives on $[0,2\pi]$. Show that 
\[
\lim_{N \rw \infty} \int_0^{2\pi} G(u) \sin ( N +\frac 1 2)u \di u=0
\]
\end{Boxed}
WLOG $\exists a,b \in [0,\pi]$ st. $\forall x \in [a,b]$ $G(x) >0$ or $G(x) <0$.

Now it suffices to show 
\[
\lim_{N \rw \infty} \int_a^{b} G(u) \sin ( N +\frac 1 2)u \di u =0
\]
[since the interval $[0,2\pi]$ can be sliced into a countable number of these intervals, and hten you can sum over them]
WLOG we assume $G(x)$ is positve.

\[
0 \leq 
\lim_{N \rw \infty} \int_a^{b} G(u) \sin ( N +\frac 1 2)u \di u 
\leq
\lim_{N \rw \infty} \int_a^{b} [\max_u G(u)] \sin ( N +\frac 1 2)u \di u 
\]
Let $G_m$ be the max above, then we have
\[
0 \leq G_m\lim_{N \rw \infty} \int_a^{b} \sin ( N +\frac 1 2)u \di u 
= G_m\lim_{N \rw \infty}  
\frac{\cos ( N +\frac 1 2)u }
{N + \frac 1 2}|_a^{b}
\leq
 G_m\lim_{N \rw \infty}  
 \frac 2
{N + \frac 1 2}
\leq 0
\]
Thus we get the 0 value for the limit as desired.

\end{Porb}
\begin{Porb}
\begin{Boxed}
Prove or disprove:
\[
\suminf{m} 
\frac 1 {(m+x)^2 + a^2} = \frac \pi a \frac {\coth \pi a } {\cos^2 \pi x + \sin^2 \pi x \coth^2 \pi a}
\]
\[
\suminf{m} 
\frac 1 {(m+x)^2} = \frac {\pi^2} { \sin^2 \pi x }
\]
\[
\suminf{m} 
\frac 1 {(2 \pi m+x)^2} = \frac {1} { 4 \sin^2 x/2 }
\]


\end{Boxed}
Let $f(m) = \frac 1 { (m+x)^2 +a^2} $, then we wish to find $\sum f(m)$. 
To this end we consider $F(k) = \intinf e^{-ikm} f(m) \di m$. To find this we use the u sub: $u = m+x$
\[
\intinf e^{-ikm} \frac 1 { (m+x)^2 +a^2}  \di m
=
e^{ikx} \intinf e^{-iku} \frac 1 { u^2 +a^2}  \di u
= 
e^{ikx} \frac \pi a e^{-|k| a}
\]


Using the Poisson formula we thus see:
\[
\suminf{m} 
\frac 1 {(m+x)^2 + a^2}  	
=
\suminf{k}
e^{i2\pi kx} \frac \pi a e^{-|2\pi k| a}
=
 \frac \pi a
 \left[
\sum{k\geq0}
e^{i 2 \pi kx}  e^{-| 2 \pi k| a}
+
\sum{k\leq0}
e^{i 2 \pi kx}  e^{-| 2 \pi k| a}
-1
\right]
\]

\[
=
 \frac \pi a
 \left[
\sum{k\geq0}
e^{i 2 \pi k(x-a)}
+
\sum{k\leq0}
e^{i 2 \pi k(x+a)}
-1
\right]
=
 \frac \pi a
 \left[
\sum{k\geq0}
e^{i 2 \pi k(x-a)}
+
e^{-i 2 \pi k(x+a)}
-1
\right]
\]

\[
=
 \frac \pi a
 \left[
\frac{ 1} { 1 -  e^{-i 2 \pi (x+a)} } +
\frac{ 1} { 1 -  e^{i 2 \pi (x-a)} }
-1
\right]
\]
\[
=
 \frac \pi a
 \left[
\frac{
1 -  e^{-i 2 \pi (x+a)}  
+1 -  e^{i 2 \pi (x-a)} 
-
\left(1 -  e^{i 2 \pi (x-a)} \right) 
\left(1 -  e^{-i 2 \pi (x+a)} \right) 
 	}
 { \left(1 -  e^{i 2 \pi (x-a)} \right) 
  \left(1 -  e^{-i 2 \pi (x+a)} \right) 
}
\right]
\]

\[
=
 \frac \pi a
 \left[
\frac{
1-
 e^{i 2 \pi (x-a)} 
e^{-i 2 \pi (x+a)} 
 	}
 { \left(1 -  e^{i 2 \pi (x-a)} \right) 
  \left(1 -  e^{-i 2 \pi (x+a)} \right) 
}
\right]
=
 \frac \pi a
 \left[
\frac{
1-
 e^{-i 4 \pi a} 
 	}
 { \left(1 -  e^{i 2 \pi (x-a)} \right) 
  \left(1 -  e^{-i 2 \pi (x+a)} \right) 
}
\right]
\]

\[
=
 \frac \pi a
 \left[
\frac{
	\left(1-e^{-i 2 \pi a}\right) 
	\left(1+e^{-i 2 \pi a}\right) 
 	}
 { \left(1 -  e^{i 2 \pi (x-a)} \right) 
  \left(1 -  e^{-i 2 \pi (x+a)} \right) 
}
\right]
=
 \frac \pi a
 \left[
\frac{
	\left(1-e^{-i 2 \pi a}\right) 
	\left(1+e^{-i 2 \pi a}\right) 
 	}
 { \left(1 -  e^{i 2 \pi (x-a)} \right) 
  \left(1 -  e^{-i 2 \pi (x+a)} \right) 
}
\right]
\]

************BELIEVE THE ABOVE DISPROVES THE SUM********\\

For the last 2 the answers are somewhat lack luster. 


Let $f(m) = \frac 1 { (m+x)^2} $, then we wish to find $\sum f(m)$. 
To this end we consider $F(k) = \intinf e^{-ikm} f(m) \di m$. To find this we use the u sub: $u = m+x$
\[
\intinf e^{-ikm} \frac 1 { (m+x)^2 }  \di m
=
e^{ikx} \intinf e^{-iku} \frac 1 { u^2 }  \di u
\]
This is an error function producing integral. This is bad.
Notice also that the claimed value on the right hand side of the equation doesn't even make sense for $x=0$.
Thus the equation as stated is clearly false, 
Besides as any good student of Bergelson knows: $\sum 1/n^2 = \frac {\pi^2} 6$



For the final one we again have issues with $\sin (0 ) = 0$
in the denominator.


\end{Porb}
\begin{Porb}
\begin{Boxed}
My man Stephane G. Mallat claims the following:
The family of funcitons $\phi(x-k) k = 0 ,\pm 1, \pm 2, \cdots$ is orthonormal iff 
\[
\suminf{k} |\hat \phi ( \omega +2\pi k)|^2 
\]
is constant wrt $\omega$.
Prove my boy wrong or right.

\end{Boxed}
Stephane is no chump and said a true thing. 
Lets investigate the sum:
\[
\suminf{k} |\hat \phi ( \omega +2\pi k)|^2 
=
\suminf{k} \overline{\hat \phi ( \omega +2\pi k)}
\hat \phi ( \omega +2\pi k) 
\]

Now to avoid a factor out front the rest of the analysis, the $\frac 1 {\sqrt {2\pi}} $ is suppressed when expanding the Fourier transform.
\[
=
\suminf{k} 
\intinf e^{i (\omega+ 2\pi k)y}\bar \phi (y) \di y 
\intinf e^{-i (\omega+ 2\pi k)x} \phi (x) \di x 
\]
\[
=
\suminf{k} 
\intinf\intinf \phi (x)\bar \phi (y) e^{i (\omega+ 2\pi k)y}
 e^{-i (\omega+ 2\pi k)x}  \di x \di y 
\]
\[
=
\intinf\intinf \phi (x)\bar \phi (y) 
\suminf{k} 
 e^{-i (\omega + 2\pi k)(x-y)}  \di x \di y 
\]
Via formula on page 62
\[
=
\intinf\intinf \phi (x)\bar \phi (y) 
 e^{-i \omega(x-y)}  
\suminf{k} 
\delta(x-y-k)  \di x \di y 
\]

Now for the change of variables $u= x-y, v = y$
\[
=
\intinf\intinf \phi (u+v)\bar \phi (v) 
 e^{-i \omega u}  
\suminf{k} \delta(u-k)  \di u \di v 
=
\intinf\bar \phi (v) 
\suminf{k} \intinf e^{-i \omega u} \phi (u+v)\delta(u-k)  \di u \di v 
\]

\[
=
\intinf\bar \phi (v) 
\suminf{k}  e^{-i \omega k} \phi (k+v) \di v 
=
\suminf{k}
e^{-i \omega u}
\intinf\bar \phi (v) 
  \phi (k+v) \di v 
=
\suminf{k}
\langle \phi (v) ,
  \phi (k+v) \rangle 
  e^{-i \omega u}
\]

Thus the sum above is in fact a Fourier series with $c_k = 
\langle \phi (v) , \phi (k+v) \rangle $.
Now this series being constnat is equivalent to $c_k = \delta_{0k}$, which is equivalent to the $\phi(v+k)$'s being an orthogonal system.

Moreover if $c_0 =1$ then we have an orthonormal system aswell. Thus the system is orthonormal if the series is constant and equal to 1.
Now in actuallity we rememeber that we have a secret factor of $\frac 1 {  2 \pi }$ hanging around. Thus the constants value is actually that.


\end{Porb}
\begin{Porb}
\begin{Boxed}
Prove or Disprove the following identities:
\begin{enumerate}[i)]
\item
		\[
	\suminf{m} f( [2m+1] \pi) = \frac 1 {2\pi} \suminf{n} (-1)^n F(n)
		\]
\item
		\[
		2 \pi \suminf{m} (-1)^m f( 2\pi m) = \suminf{n} F(n + \frac 1 2)
		\]
\item
		\[
			\suminf{m} \delta(u- [2m+1] \pi) = \frac 1 {2\pi} \suminf{n} (-1)^n e^{inu}
		\]
\item
	And in greater generality 
		\[
	\suminf{m} f(\frac{ [2m+1] \pi} {a} ) =
		\frac a {2\pi} \suminf{n} (-1)^n F(na)
		\]
\item
		\[
	\suminf{m} \frac 1 {|a|} \delta(u- \frac{[2m+1] \pi} {a}) = \frac 1 {2\pi} \suminf{n} (-1)^n e^{inau}
		\]
\end{enumerate}

\end{Boxed}
The main equation to keep in mind here is the general Poisson formula:
\[
	\frac 1 {2 \pi} \suminf{n} \intinf e^{in(x-t)} f(t) \di t = \suminf{m} f(x+ 2 \pi m)
\]
\begin{enumerate}[i)]
	\item
		Begin with $x=\pi$ in the formula above and we see:
\[
	\frac 1 {2 \pi} \suminf{n} \intinf e^{in(\pi -t)} f(t) \di t = \suminf{m} f([2 m+1]\pi )
\]
\[
=	\frac 1 {2 \pi} \suminf{n} (-1)^n \intinf e^{-int} f(t) \di t
		=	\frac 1 {2 \pi} \suminf{n} (-1)^n F(n)
	\]



	\item
		Begining with the right hand side:
		\[
			\suminf{n} F(n + \frac 1 2) 
	= \suminf{n} \intinf e^{-i(n+\frac 1 2) t} f(t) \di t
	= \suminf{n} \intinf e^{-in t} f(t)e^{-i t/2} \di t
		\]
		Now we notice this is the Fourier transform of not $f$ but of $f(t) e^{-i t/2}$, applying Poisson sum with this:
		\[
			= 2\pi \suminf{m} f(2 \pi m) e^{-i (2 \pi m)/2}
			= 2\pi \suminf{m} f(2 \pi m) e^{-i \pi m}
		=2 \pi \suminf{m} (-1)^m f( 2\pi m) 
		\]

	\item
		It is straightforward to see that this is actually just 1) but in the supressed function notation.
		To see this we note that
		\[
			\delta( u - [2m+1] \pi) \rw f([2m+1] \pi), \quad  e^{inu} \rw \intinf e^{inu} f(u) \di u = F(-n)
		\]
		But wait we get $\suminf n (-1)^n F(-n)$ and not the exact sum we wanted! Thankfully $(-1)^n = (-1)^{-n}$ and we just switch the order of the sum and get the identity.
		
	\item
		Let $\bar f(x) = f(\frac x a )$, then by 1) we have:
		\[
			\suminf m f ( \frac {[2m+1] \pi} a) = \suminf m \bar f ( [2m+1] \pi) =
			\frac 1 {2\pi} \suminf n (-1)^n \bar F (n) 
		\]
		\[
			=
			\frac 1 {2\pi} \suminf n (-1)^n \intinf e^{-int} \bar f (t) \di t
			=
			\frac 1 {2\pi} \suminf n (-1)^n \intinf e^{-int} f (t/a) \di t
		\]
		\[
			=
			\frac 1 {2\pi} \suminf n (-1)^n a  \intinf e^{-inau} f (u) \di u
			=
			\frac 1 {2\pi} \suminf n (-1)^n a F(na) 
			=
			\frac a {2\pi} \suminf n (-1)^n  F(na) 
		\]
		
	\item
		Similar to 3) we note that this is just an earlier identity. A constant is shifted around but this is basicly just 4).
		
\end{enumerate}
\end{Porb}

\subsection{Dirac Delta Distribution}
\setcounter{Prb}{0}

\begin{Porb}
\begin{Boxed}
	Show that 
	\[
		\lim_{\omega \rw \infty} \frac { \sin 2 \pi \omega x } { \pi x}, \ \omega >0
	\]
	is a representation of the Dirac $\delta$ DISTRIBUTION.
\end{Boxed}
This equality can only be expressed inside of an integral, thus we must apply the above to test functions and see that the answer is the same as with the delta \emph{distribution}.

Thus if we consider f continuous on some $[-a,a]$ then we get:
\[
	\lim_{\omega \rw \infty} \int_{-a}^a \delta_\omega (x) f(x) \di x
	=
	\lim_{\omega \rw \infty} \int_{-a}^a\frac { \sin 2 \pi \omega x } { \pi x} f(x) \di x
\]
Now in following with the style of the Fourier series theorem we add and subtract the same term, namely a $f(0)$ (inside some paranthesis but basicly the same)

\[
	=
	\lim_{\omega \rw \infty} \int_{-a}^a\frac { \sin 2 \pi \omega x } { \pi x} \left(f(x)  - f(0) + f(0) \right)\di x
	=
	\lim_{\omega \rw \infty} \int_{-a}^a\frac { \sin 2 \pi \omega x } \pi  \frac{ f(x)  - f(0)}  { x}
+f(0)\frac { \sin 2 \pi \omega x } { \pi x}\di x
\]

Now we can consider WLOG just the positive side of the integral.
\[
	\lim_{\omega \rw \infty} \int_{0}^a\frac { \sin 2 \pi \omega x } \pi  \frac{ f(x)  - f(0)}  { x}
+f(0)\frac { \sin 2 \pi \omega x } { \pi x}\di x
\]
Notice for the exact same resonaning as $G(u)$ on page 57 that we get $\frac{ f(x) - f(0)} {x}$ is continuous at 0 and converges to $f'(0^+)$. 
Thus again we see that the integral:
\[
	\int_{0}^a\frac { \sin 2 \pi \omega x } \pi  \frac{ f(x)  - f(0)}  { x} \di x 
	\rw 0
\]
as $\omega \rw \infty$. Thus we only have:
\[
	\lim_{\omega \rw \infty} \int_0^a f(0)\frac { \sin 2 \pi \omega x } { \pi x}\di x
	=
	f(0) \lim_{\omega \rw \infty} \int_0^a \frac { \sin 2 \pi \omega x } { \pi x}\di x
\]

	\[
	y = 2 \pi \omega x, \quad 
	dy = 2 \pi \omega dx
	\]
%	\quad 
%	\frac y {2 \pi \omega } = x, \quad 
%	\frac {dy} { 2 \pi \omega } = dx

\[
	=
	f(0) \lim_{\omega \rw \infty} \int_0^{a 2 \pi \omega} \frac { \sin y } { \pi y/ (2 \pi \omega) }\frac{ \di y} {2 \pi \omega}
	=
	\frac {f(0)} \pi  \int_0^{\infty} \frac { \sin y } {  y } \di y
	=
	\frac {f(0)} \pi  \frac \pi 2
	=
	\frac {f(0)} 2
\]
Using a isomorphic version of the logic above one can get the $f(0^-)$ term and complete the proof.



\end{Porb}
\begin{Porb}
\begin{Boxed}
	Assuming that $f(x)$ is nearlly linear, that is to say that 
	\[
		f(-a) = f(0) - a f'(0) + \text{ H.O.T.}
	\]
	Show that 
	\[
	I = \intinf \delta ( x+a) f(x) \di x
	\]
	can be evaluated by means of the formal equation:
	\[
	\delta(x+a) = \delta(x) + a \delta'(xx)
	\]

\end{Boxed}

By the definition of the $\delta$ funciton we have:
\[
	I = \intinf \delta ( x+a) f(x) \di x = f(-a)
\]
\[
	=
	f(0) - a f'(0) + \text{ H.O.T.}
	= 
	\intinf  \delta (x)f(x)  - a \delta(x)f'(x) \di x 
	= 
	\intinf  \delta (x)f(x)\di x  - a \intinf \delta(x)f'(x) \di x 
\]
Via integration by parts we know that:
\[
\intinf \delta(x)f'(x) \di x 
=
\delta(x) f(x) |_{\pm \infty}
- \intinf \delta'(x)f(x) \di x 
=
- \intinf \delta'(x)f(x) \di x 
\]
Putting stuff together:
\[
	= 
	\intinf  \delta (x)f(x)\di x  + a \intinf \delta'(x)f(x) \di x 
	= 
	\intinf  \left[\delta (x)+ a \delta'(x)\right]f(x) \di x 
\]
Thus it makes some sense to claim $\delta (x +a ) = \delta(x) + a\delta'(x)$

\end{Porb}


\subsection{The Fourier Integral}
\setcounter{Prb}{0}

\begin{Porb}
\begin{Boxed}
	\begin{enumerate}[a)]
		\item Consider the LInear Operator $\mathfrak{F}^2$ and its eigenvalue equatio
			\[
				\mathfrak{F}^2 f = \lambda f
				\]
				What are the eigenvalues and eigenfunctions of $\mathfrak{F}^2$?
		\item Same with $\mathfrak{F}^4$?
		\item Same with $\mathfrak{F}$?
	\end{enumerate}

\end{Boxed}
For the sake of clarity:
	\[
		\FF (f) =\frac 1 {\sqrt{2 \pi}}  \intinf e^{-ikx} f(x) \di x
		\]
		\begin{enumerate}[a)]
			\item
				Obviously we begin with the calculation in question
	\[
		\FF^2 (f)[k] 
				=\frac 1 {2 \pi}  \intinf e^{-iky}  \intinf e^{-iyx} f(x) \di x\di y
				=\frac 1 {2 \pi}  \intinf  \intinf e^{-i(ky+yx)} f(x) \di x\di y
	\]
				This looks close to the idenity we are given on page 70, namely:
		\[
		\delta(x-t) = \intinf \frac 1 {2\pi} e^{ik(x-t)} dk
		\]
	which carries the note that we must inegrate on the outside with $\di t$ for this to make sense. 
		Now rearranging some integrals and swapping x with $-x$ we arive at:

	\[
	  \intinf f(-x)  \intinf\frac 1 {2 \pi} e^{-iy(k-x)} \di y \di x
	=
	  \intinf f(-x)  \delta(k-x) \di x
	=
	f(-k)  	\]
				

Now we can see that the constraint: $ \mathfrak{F}^2 f = \lambda f$
is really just $f(-x) = \lambda f(x)$. 
Two obvious cases come to mind, namely even and odd functions for the eigenvalues $\pm 1$. 
				For any other value of $\lambda$ one could apply the relation twice to get $f(x) = \lambda^2 f(-x)$ which only has 2 roots. 
				Thus those are the only eigenvalues of $\FF^2$.


	\item
		Thanks to $ \mathfrak{F}^2 f[x] = f(-x)$ from the previous problem we know that $\FF^4 = \FF^2\FF^2 f [x] = f(-(-x)) =f(x)$.
				Thus every function is an eigen function of $\FF^4 = $I$\di$, with eigenvalue 1.

	\item
		We know that $\FF^4=$I$\di$, and thus if $\lambda$ is an eigenvalue of $\FF$ then $\lambda^4=1$. Thus the only possible eigenvalules of $\FF$ are 4th roots of unity. 
		Thus the eigenvalues are $\pm1, \pm i$. 
		\end{enumerate}
\end{Porb}


\begin{Porb}
\begin{Boxed}
	Let 
	\[
		W = \text{span}\{ \phi, \FF \phi, \FF^2 \phi, \dots\}
		\]
	\begin{enumerate}[a)]
		\item Show that W is finite dimensional, and what is its dimension?
		\item Exhibit a basis for W.
		\item It is evident that $\FF$ is a unitary transform of W. Find the bassis representation matrx $[\FF]_B$ relative to the basis B found in part b).
		\item Find the secular determinant, the eigenvalues and the corresponding eigenvectors of $[\FF]_B$.
		\item For W exhibit an alternative basis which consists entirely of eigenvectors of $\FF$, each one labelled by its respective eigenvalue.
		\item What can you say about the eigenvalues of $\FF$ as a transformation on $L^2$ as compared to $[\FF]_B$ which acts ona finite dim. vector space
	\end{enumerate}
\end{Boxed}

	\begin{enumerate}[a)]
	\item
		W clearly has dimension $\leq 4$ by the previous problem since $\FF^4 =$I$\di$. In fact if $\phi$ is even, we have only dimension 2 and the two basis elements of the space are just $\phi$ and $\FF \phi$. Due to the limits of the roots of unity argument above we know that the only number of dimensions can be those two or 1, namely dim = 1,2, or 4.

	\item The possible basis are $\phi$, or $\phi, \FF \phi$, or all 4: $\phi, \FF \phi, \FF^2 \phi, \I \FF \phi$.

	\item
		With the basis: $\phi, \FF \phi, \FF^2 \phi, \I \FF \phi$
		\[
		[\FF]_B 
		=
		\begin{bmatrix}
			0 &1 & 0 & 0\\
			0 &0 & 1 & 0\\
			0 &0 & 0 & 1\\
			1 &0 & 0 & 0
		\end{bmatrix}
		\]
			A classic style of shifting operator on a finite dimensional space.
	\item
		Easily enough we see:
		\[
		\det \left( [\FF]_B  - \lambda I\right)
		=
			\det \left(
		\begin{bmatrix}
			-\lambda &1 & 0 & 0\\
			0 &-\lambda & 1 & 0\\
			0 &0 & -\lambda & 1\\
			1 &0 & 0 & -\lambda
		\end{bmatrix}
			\right)
			= \lambda^4 -1
		\]
		Thus the eigenvalues are $\pm 1, \pm i$. The corresponding vectors are listed below:
			\[
		\begin{matrix}
			\lambda = 1 & \phi + \FF \phi + \FF^2 \phi + \I \FF \phi \\
			\lambda = -1 & \phi - \FF \phi + \FF^2 \phi - \I \FF \phi \\
			\lambda = i & \phi +i \FF \phi - \FF^2 \phi -i \I \FF \phi \\
			\lambda = -i & \phi -i \FF \phi - \FF^2 \phi +i \I \FF \phi \\
		\end{matrix}
				\]
				(Something something permutation matricies)
	\item
		The eigenvalues of $\FF$ are the same viewed as a finite-dimensional vecotr space and as an infinte dimensional one. 
			This seems to have been forced by the simpliity of the characterisitc polynomial more than anything else.

	\end{enumerate}
\end{Porb}

\begin{Porb}
\begin{Boxed}
	Define the equivalent width as 
	\[
		\Delta_t = \left| \frac{ \intinf f(t) \di t } {f(0)} \right|
		\]
	Define the equivalent Fourier width as 
	\[
		\Delta_\omega = \left| \frac{ \intinf \hat f(t) \di t } {\hat f(0)} \right|
		\]
	\begin{enumerate}[a)]
		\item Show that $\Delta_t \Delta_\omega = $const, is independent of the function f, and find its value.
		\item Determine the equivalent width and Fourier width of 
		\[
			e^{-x^2/2b^2}
		\]
			and compare them with its full width as defined by its inflection points.
	\end{enumerate}
\end{Boxed}
	\begin{enumerate}[a)]
		\item
			\[
		\Delta_t \Delta_\omega 
			= \left| \frac{ \intinf \hat f(t) \di t } {\hat f(0)} \right|
			\left| \frac{ \intinf f(t) \di t } {f(0) }\right|
			= \left| \frac{ \intinf \hat f(t) \di t  \intinf f(t) \di t } {\hat f (0) f(0)} \right|
			\]
			\[
			= \left| \frac{ \intinf\intinf \hat f(x) f(y)\di x \di y } {\hat f (0) f(0)} \right|
			= \left| \frac{ \intinf\intinf \intinf f(y) \ispi e^{-ixz} f(z)\di z \di x \di y } {\hat f (0) f(0)} \right|
			=
			\]
			\[
			 \left| \frac{ \intinf\intinf f(y)f(z) \intinf \ispi e^{-ixz} \di x \di z \di y } {\hat f (0) f(0)} \right|
			= \left| \frac{\sqrt{2\pi} \intinf\intinf f(y)f(z) \delta(z) \di z \di y } {\hat f (0) f(0)} \right|
			= \sqrt{2\pi} \left| \frac{\intinf f(y) \di y } {\hat f (0) } \right|
			\]
			\[
			= \sqrt{2\pi} \left| \frac{\intinf f(y) \di y } {\ispi \intinf e^{-i*0*x} f(y) \di y  } \right|
			= \sqrt{2\pi}  \sqrt{2\pi} 
			= 2\pi 
			\]

		\item
			With $f(x) = e^{-x^2/2b^2}$

		Since I stared at completing the squares for way too long to justify not writing this down, here is the Fourier transform of the Gaussian:
			\[
			\hat f(\omega ) 
			= \ispi \intinf  e^{-x^2/2b^2} e^{-i\omega x} \di x 
			= \ispi \intinf  e^{-1/2b^2 \left[x^2+2b^2i\omega x\right]} \di x 
				\]

				We complete the square in the exponent:
			\[
			= \ispi \intinf  e^{-1/2b^2 \left[x^2+2b^2i\omega x  - \omega^2 b^4\right] - \omega^2 b^2/2} \di x 
			= \ispi e^{-\omega^2 b^2/2} \intinf  e^{-1/2b^2 \left[x- \omega b^2\right]^2} \di x 
			\]
			Now with $ |b| u = x - \omega b^2,  |b| \di u = \di x $ we get:
			\[
			= |b| e^{-\omega^2 b^2/2} \intinf \ispi  e^{-u^2/2} \di u 
			= |b| e^{-\omega^2 b^2/2} 
			\]

			So we have:
			$
			\hat f(\omega ) 
			= |b| e^{-\omega^2 b^2/2} 
			$
			and can now do our calculations. 
			(we could before and actually don't need this at all but I'ill be damned if I didn't spend too much time on this part to not just write something)

			The $\Delta_\omega$ is actually the famous Gaussian integral:
			\[
			\Delta_\omega 
			=
			\left| \frac{ \intinf e^{-t^2/2b^2} \di t } {1 }\right|
			=
			\left|  \intinf e^{-t^2/2b^2} \di t \right|
			= 
			\sqrt{ \frac \pi {1/2b^2}}
			=
			\sqrt{ \pi 2b^2}
			=
			|b|\sqrt{ \pi 2}
			\]

			Thanks to the relation
			$
		\Delta_t \Delta_ \omega	= 2 \pi
		$ 
		we see that $\Delta_t$ must be $1/|b| \sqrt{ 2 \pi} $. 

		I am too stuburn to not write this after the above
			\[
		\Delta_t 	
			= \left| \frac{ \intinf \hat f(t) \di t } {\hat f(0)} \right|
			= \left| \frac{ \intinf |b| e^{-\omega^2 b^2/2} \di \omega } {|b|} \right|
			= \left| \intinf  e^{-\omega^2 b^2/2} \di \omega  \right|
			\]
			\[
			= \left| \intinf  e^{-\omega^2 b^2/2} \di \omega  \right|
			= \sqrt{ 2 \pi / b^2} 
			= \frac 1 {|b|} \sqrt{ 2 \pi } 
			\]

			The inflection points are at $\pm b$ and thus its 'inflection witdth' is $2|b|$

	\end{enumerate}
\end{Porb}


\begin{Porb}
\begin{Boxed}
	Define the auto-correlation h of the funciton f:
	\[
		h(y):= \intinf f(x) f(x-y) \di x
	\]
	Compute the Fourier transform of the auto correlation funcation and show that it equals the "spectral intensity" (aka power spectrum) of f whenever f is real valued. 
\end{Boxed}
	\[
	\hat h(k)
	= \frac 1 {\sqrt{ 2\pi}} \intinf e^{-iky}  \intinf f(x) f(x-y) \di x \di y
	= \frac 1 {\sqrt{ 2\pi}} \intinf \intinf e^{-iky} f(x) f(x-y) \di x \di y
	\]
	\[
	= \frac 1 {\sqrt{ 2\pi}} \intinf f(x) \intinf e^{-ik(x-u)} f(u) \di u \di x 
	=  \intinf e^{-ikx}f(x) \hat f (-k) \di x 
	=\hat f (-k)   \intinf e^{-ikx}f(x) \di x 
	\]
	\[
	=\hat f (-k)  \hat f(k) 
	=|\hat f(k)| ^2
	\]
	The $\hat f(-k)= \overline{\hat f(k)} $ is implied by f being real valued and is the only point we make use of this fact. 
\end{Porb}



\begin{Porb}
\begin{Boxed}
	\begin{enumerate}[a)]
		\item
			Compute the total energy \[ \intinf |h(T)|^2 \di T\]
			of the cross correlation $h(T)$ in terms of the Fourier amplitudes of $f_0$ and $f$.
		\item
			Consider
			\[
				h_k(T) = \frac {\intinf \bar f_0 (t-T) f_k(t) \di t}
				{ \left[ 
				\intinf |f_k(t)|^2 \di t
				\right] ^{1/2} }
				\]
	\begin{enumerate}[i)]
		\item
			Show that $h_0(t)$ is the peak intensity, ie $|h_k(T)|^2 \leq |h_0(T)|^2$.
		\item
			Show that equality holds if $f_k(t) = \kappa f_0(t) $ for $\kappa$ some constant.
	\end{enumerate}
	\end{enumerate}
\end{Boxed}

\begin{enumerate}[a)]
	\item
		For a matchd filter we have that $\intinf \bar f_0 (t-T) f(t) \di t = h(T)$, using this:
		Using the fact that $\FF$ is an isometery of $L^2$:
		\[
			\intinf |h(T)|^2 \di T
			=
			\norm{ h}_2
			=
			\norm{ \hat h}_2
			=
			\intinf |\FF h(\omega)|^2 \di \omega
			\]


			\[
			=
			\intinf |
			\intinf e^{-i  T \omega}
			h(T)
			\di T
			|^2 \di \omega
			=
			\intinf |
			\ispi
			\intinf e^{-i  T \omega}
			\intinf \bar f_0 (t-T) f(t) \di t
			\di T
			|^2 \di\omega 
			\]
			\[
			=
			\intinf |
			\ispi
			\intinf 
			f(t) 
			\intinf 
			e^{-i  T \omega}
			\bar f_0 (t-T) 
			\di T
			\di t
			|^2 \di\omega 
			\]
			Let $u= t-T, \ \di u = - \di T$ (the negative sign is lost in the $||$).
			\[
			=
			\intinf |
			\ispi
			\intinf 
			f(t) 
			\intinf 
			e^{-i (t-u) \omega}
			\bar f_0 (u) 
			\di u
			\di t
			|^2 \di\omega 
			=
			\intinf |
			\ispi
			\intinf 
			e^{-i t \omega}
			f(t) 
			\intinf 
			e^{i u \omega}
			\bar f_0 (u) 
			\di u
			\di t
			|^2 \di\omega 
			\]
			Thanks to the relationship between conjugates and the fourier transform, namely $\FF \bar f [k] = \FF f[-k]$ we get:
			\[
			=
			\intinf |
			\ispi
			\hat f_0 (\omega) 
			\intinf 
			e^{-i t \omega}
			f(t) 
			\di t
			|^2 \di\omega 
			=
			\intinf |
			\ispi
			\hat f_0 (\omega) 
			\hat	f(\omega) 
			|^2 \di\omega 
			\]

			\[
			=
			\frac 1 {2 \pi}
			\intinf |
			\hat f_0 (\omega) 
			\hat f(\omega) 
			|^2 \di\omega 
			\]


	\item
\begin{enumerate}[i)]
	\item
		This problem is actually wrong, consider $f_0(t) = \one_{[0,1]} (t)$ and $f_1(t) = \one_{[2,3]} (t)$ and $T=2$.

		Then $\norm{f_1}_2 =1 = \norm{f_0}$ but 
		\[
			\intinf \bar f_0 (t-T) f_0(t) \di t
			=0
		\]
		
		\[
			\intinf \bar f_0 (t-T) f_k(t) \di t
			=1
		\]
		Which gives $h_1(T)>h_0(T)$, contrary to the statment of the problem.

%		$\forall k$ and fixed T we can write 
%		\[
%			f_k(t) = 
%			\frac{ \intinf \bar f_0 (t-T) f_k(t) \di t} 
%			{ \intinf \bar f_0 (t-T) f_0(t) \di t} 
%			f_0(t) + g(t) 
%			\]
%			where $\intinf \bar g(t-T) f_0(t) \di t = 0$
%			By orthogonality of these functions we see that:
%		\[
%			\intinf f_k(t-T) f_k(t) \di T = 
%			\frac{ \intinf \bar f_0 (t-T) f_k(t) \di t} 
%			{ \intinf \bar f_0 (t-T) f_0(t) \di t} 
%			f_0(t) + g(t) 
%			\]
%
%		\[
%			|h_k(T)|^2 
%			=
%			\norm{ 
%			\frac {\intinf \bar f_0 (t-T) f_k(t) \di t}
%			{ \left[ 
%			\intinf |f_k(t)|^2 \di t
%			\right] ^{1/2} }
%			}^2
%			=
%			\frac {\norm{ \intinf \bar f_0 (t-T) f_k(t) \di t }^2 }
%			{ 
%			\intinf |f_k(t)|^2 \di t
%			 }
%		\]
%
%
%
%		\[
%			\leq
%			\frac { \left( \intinf \norm{\bar f_0 (t-T) f_k(t) }\di t \right)^2 }
%			{ 
%			\intinf |f_k(t)|^2 \di t
%			 }
%			 =
%			\frac {  \norm{\bar f_0 (t-T) f_k(t) }_1^2 }
%			{ 
%			\norm{ f_k(t)}_2^2
%			 }
%		\]
%		Since this is true for any k we notice that we can see it is sufficent to show:
%	\[
%			 \norm{\bar f_0 (t-T) f_k(t) }_1 ^2
%			 \leq
%			 \norm{\bar f_0 (t-T) f_0(t) }_1 ^2
%	\]
%		**************************************************\\
%		Or even simplier after taking square roots:
%	\[
%			 \norm{\bar f_0 (t-T) f_k(t) }_1 
%			 \leq
%			 \norm{\bar f_0 (t-T) f_0(t) }_1 
%	\]

		**********************MESSGE GERLACH****************************\\
	\item
		Oddly enough the equality still holds.
		If we have $f_k(t) = \kappa f_0(t)$ then we see:
		\[
			|h_k(T)|^2 
			=
			\norm{ 
			\frac {\intinf \bar f_0 (t-T) \kappa f_0(t) \di t}
			{ \left[ 
			\intinf |\kappa f_0(t)|^2 \di t
			\right] ^{1/2} }
			}^2
			=
			\norm{ 
			\frac {\intinf \bar f_0 (t-T) f_0(t) \di t}
			{ \left[ 
			\intinf |f_0(t)|^2 \di t
			\right] ^{1/2} }
			}^2
			= |h_0(T)|^2
		\]




\end{enumerate}
\end{enumerate}

\end{Porb}


\begin{Porb}
\begin{Boxed}
What functions are eigenvectors of $\FF^2$ with eigenvalue $\lambda= 1$?
\end{Boxed}
Already did this.
\end{Porb}


\begin{Porb}
\begin{Boxed}
	Let $\hat g(k) = \FF [g(x)](k)$ and $H(k)= \FF[h](k)$ be the Fourier transforms of g and h.
	Express the following in terms of $\hat g$ and $\hat f$.
	\begin{enumerate}[i)]
		\item $\FF [ \alpha g + \beta h]$ for some constants $\alpha, \beta$
		\item $\FF[ g(x- \xi)]$
		\item $\FF [ e^{i k_0 x} g]$
		\item $\FF [g(ax)]$
		\item $\FF [ \frac{d g}{dx}]$
		\item $\FF[ x g(x)]$
	\end{enumerate}
\end{Boxed}
	\begin{enumerate}[i)]
		\item 
			It is clear by the linearity of the integrals that we have:
			\[
			\FF [ \alpha g + \beta h] 
			=
			\FF [ \alpha g] +\FF[ \beta h] 
			=
			 \alpha \FF [g] +\beta \FF[ h] 
			\]

		\item 
			\[
			\FF[ g(x- \xi)]
			=
			\intinf e^{-ikx} g(x- \xi) \di x
			=
			\intinf e^{-ik(u+\xi)} g(u) \di u 
			=
			e^{-ik(\xi)} \intinf e^{-iku} g(u) \di u 
			=
			e^{-ik(\xi)} \hat g
			\]

		\item 
			\[
			\FF [ e^{i k_0 x} g]
			=
			\intinf e^{-ikx}e^{i k_0 x} g(x) \di x
			=
			\intinf e^{-i(k-k_0)x} g(x) \di x
			=
			\hat g(k-k_0)
			\]

		\item 
			\[
				\FF [g(ax)]
			=
			\intinf e^{-ikx}g(ax) \di x
			=
			\intinf \frac 1 a e^{-iku/a}g(u) \di u
			=
			\frac 1 a \hat g (k/a)
			\]

		\item 
			Using integration by parts:
			\[
			\FF [ \frac{d g}{dx}]
			=
			\intinf e^{-ikx} \frac{d g}{dx}\di x
			=
			- \intinf  g\frac{d e^{-ikx}}{dx}\di x
			=
			\intinf ik ge^{-ikx}\di x
			=
			ik \hat g(k)
			\]

		\item 
			\[
			\FF[ x g(x)]
			=
			\intinf e^{-ikx} x g \di x
			=
			\intinf x e^{-ikx}  g \di x
			=
			\intinf \left[ \frac 1 {-i} \frac d {dk}  e^{-ikx}\right]  g \di x
			\]
			\[
			=
			\frac 1 {-i} \frac d {dk} \intinf  e^{-ikx} g \di x
			=
			\frac 1 {-i} \frac d {dk} \hat g(k)
			\]
			***************************DOUBLE CHECK LAST COUPLE\\
	\end{enumerate}

To make life easier to see, eveyrthin in a box gives:
	\begin{enumerate}[i)]
		\item $ \FF [ \alpha g + \beta h] = \alpha \FF g + \beta \FF h $ for some constants $\alpha, \beta$
		\item $\FF[ g(x- \xi)] = e^{i k \xi} \FF g [k] $
		\item $\FF [ e^{i k_0 x} g] = \FF g [ k - k_0]$
		\item $\FF [g(ax)] =  \frac 1 a \hat g (k/a)$
		\item $\FF [ \frac{d g}{dx}] = ik \FF g$
		\item $\FF[ x g(x)] = 
			\frac 1 {-i} \frac d {dk} \hat g(k)
			$
	\end{enumerate}



\end{Porb}

\begin{Porb}
\begin{Boxed}
	Show that any periodic function $f(\xi) = f(\xi +a)$ is the convolution of a nonperiodic function with a train of Dirac delta DISTRIBUTIONS.
\end{Boxed}
[I was very stuck on this and stack exchange provided an answer.]\\
Let $a>0$ be the length of the period of the function $f(x)$. Then let $g(x) = \one_{[0,a)} (x) f(x)$ have value in the 'first' period of f and then be 0 elsewhere.
Obviously g is non periodic unless $f=0$ (that case being triival and not relavent). Now consider:
\[
	g \star \left( \suminf n \delta(x-an) \right)	 = 
\suminf n g \star \delta(u-(x-an))  = 
\suminf n g(x-an) = f(x) 
\]
Thus we have written f as a convlution of a non periodic function and a train of delta distributions.
\end{Porb}


\begin{Porb}
\begin{Boxed}
	Find the Fourier specturm of a finite train of identical coherent pulses of the kind shown in Fig. 2.9.
\end{Boxed}
The function in reference is of the form:
	\[
		f_n(t) = e^{-(t-nT)^2/2b^2} e^{i\omega_0(t-nT)} e^{i\delta_n}
	\]
	Which in our specific case is:
	\[
		f_n(t) = e^{-(t-nT)^2/2b^2} e^{i\omega_0(t-nT)} 
	\]
	So our sum is:
	\[
		\sum_{n=-N}^N	f_n(t) 
		=
	\sum_{n=-N}^N	e^{-(t-nT)^2/2b^2} e^{i\omega_0(t-nT)} 
	\]
%	Now we wish to calculate the Fourier transform:
%	\[
%	\sum_{n=-N}^N	\FF e^{-(t-nT)^2/2b^2} e^{i\omega_0(t-nT)} 
%	=
%	\ispi
%	\sum_{n=-N}^N	
%	\intinf
%	e^{-(t-nT)^2/2b^2} e^{i\omega_0(t-nT)} 
%	e^{-i \omega t} \di t 
%	\]

	We could in theory calculate the Fourier transform of each of the elmeents of the sum and then combine. However the text has outlined a process we can just semi blindly follow with less work.

	To this end we notice that with $\delta_n=0= n \Delta \phi $ we have the same form as page 90 with:
	\[
		f(t) = 
		=
	\sum_{n=-N}^N	e^{-(t-nT)^2/2b^2} e^{i\omega_0(t-nT)} e^{in \Delta \phi}
	\]
	but with finite bounds on our sum. Thankfully we can still rewrite it as a convolution:
	\[
		f(t) = 
		\intinf
	e^{-(t-\xi)^2/2b^2} e^{i\omega_0(t-\xi)} 
	\sum_{n=-N}^N	
	\delta ( \xi - n T ) \di \xi
	\]

	Our pulses have the same form as before, but our comb is much shorter this time.
	\[
		\FF[ \text{pulse} ](\omega) 
		=
		\intinf 
	e^{-t^2/2b^2} e^{i\omega_0t} 
	\di t
		=
		b e^{-(\omega-\omega_0)^2/2b^2} 
	\]

	\[
		\FF[ \text{small comb} ](\omega) 
		=
	\sum_{n=-N}^N	
	\FF
	\delta ( \xi - n T ) 
		=
	\sum_{n=-N}^N	
	\ispi
	\intinf
	e^{-i \omega \xi}
	\delta ( \xi - n T ) 
	\di \xi
	\]

	\[
		=
	\ispi
	\sum_{n=-N}^N	
	e^{-i n \omega T }
	=
	\sqrt{2 \pi}
	\frac{ \sin (N+\frac 1 2) \omega T}
	{\sin \omega T/2}
	\]
	In the book they use Poisson's sum formula, here we are not so lucky as the bounds are finite.

	Combining now our expressions for the Fourier transform we get:

	\[
	\FF [f](\omega)
	=
	\sqrt{2 \pi}
	b e^{-(\omega-\omega_0)^2/2b^2} 
	\frac{ \sin (N+\frac 1 2) \omega T}
	{\sin \omega T/2}
	\]
	Thus in the end the spectral envelope ends up being the same as it was determined by the amplidude.
	The spectral lines portion though, is now just a finite approximation. Notably it is a function and has not achieved yet distribution status.
	The spectral lines in this case wobbles much more and has support on the whole real line and not just at integerl multilpes of $2\pi$ plus a $\Delta \phi$ factors.
\end{Porb}


\begin{Porb}
\begin{Boxed}
	Verify that
	\[
		f(t) =
		\suminf n 
		e^{-(t-nT)^2/2b^2} e^{i\omega_0(t-nT)} 
		\]
		is a periodic function of t, and that $f(t+T)=f(t)$.
		Find the full 4-ier representation
		\[
			f(t) = \suminf m c_m e^{i\omega_m t}
			\]
			of f by determing $\omega_m$ and $c_m$.
\end{Boxed}
Verifying periodicity is straightforward:
	\[
		f(t+T) =
		\suminf n 
		e^{-(t+T-nT)^2/2b^2} e^{i\omega_0(t+T-nT)} 
		=
		\suminf n 
		e^{-(t-(n-1)T)^2/2b^2} e^{i\omega_0(t-(n-1)T)} 
		\]
		\[
		=
		\suminf n 
		e^{-(t-nT)^2/2b^2} e^{i\omega_0(t-nT)} 
		=f(t)
		\]

		Now to find the 4-ier representation:

Similar to how $\FF \delta  = 1 \ \Rw \delta(x) = \suminf n e^{inx}$, and $\FF 1 = \delta \ \Rw 1 = \suminf n \delta(n) e^{inx}$, we can do the same with this series.
	(Note the $\delta$ distriutions outside of integrals is 'problematic' but not a problem)

	Or for a more close situation: $\FF e^{iwx}[\omega] = \sqrt{2 \pi} \delta(\omega-w) \ \Rw  e^{iwx} = \suminf n \sqrt{2 \pi} \delta( n- w) e^{inx}$.
	Now something to remmeber here is that we are no longer working with $L^2$ functions where $\FF$ gives us a bijective map to $l^2$. Now we know the specturm of the distribution from $\FF$ but it is not neccisarily true that $ \sum_n \FF f(n) e^{in x} = f(x)$ or that the left hand side even has a meaning.


	So following $e^{iwx} \rw e^{i w x}$ even for $w \not \in \ZZ$. 
		We know from the work done in the book that
		\[
			\FF [f](\omega) = 
	\sqrt{2 \pi}
	b e^{-(\omega-\omega_0)^2/2b^2} 
	\suminf m  \delta \left( \omega T - 2 \pi m \right)
		\]
		The Fourier 'type' series is then:
		\[
			f(t) = 
			\frac 1 {2\pi}
			\sum n
			e^{-i 2 \pi n t} \FF f (2 \pi n)
			=
			\frac 1 {2\pi}
			\sum n 
			e^{-i n t} 
	\sqrt{2 \pi}
	b e^{-(n-\omega_0)^2/2b^2} 
	\suminf m  \delta \left( n T - 2 \pi m \right)
			\]
			\[
			=
			b
			\suminf n 
			e^{-i n t-(n-\omega_0)^2/2b^2} 
	\suminf m  \delta \left( n T - 2 \pi m \right)
			=
			b
			\suminf n 
			e^{-i n t-(n-\omega_0)^2/2b^2} 
	\suminf m   T \delta \left( n - \frac 2 T \pi m \right)
			\]
			Thus we want $n = \frac 2 T \pi m$ to have nonzero values from the infinite sum:
			***************SKETCHY*********************\\
			\[
			=
			\sum_{m} 
			bT
			e^{-(\frac 2 T \pi m -\omega_0)^2/2b^2}
			e^{-i 2\pi m /T t} 
			\]
			$\omega_m = 2 \pi m /T, \ c_m = bT e^{-(m-\omega_0)^2/2b^2}$.
\end{Porb}


\subsection{Orthonormal Wave Packet Representation}
\setcounter{Prb}{0}

\begin{Porb}
\begin{Boxed}
	Consider the set of functions:
	\[
		\left\{
			P_{jl}(t) = \frac 1 {\sqrt \eps} \int_{j\eps}^{(j+1)\eps} e^{2\pi i l \omega/\eps} 	
			\ispi e^{-i\omega t} \di \omega
			, \quad j,l = 0,\pm1, \pm 2 , \dots
			\right\}
		\]
	\begin{enumerate}[a)]
		\item Show that these wave packets are orthonormal
		\item Show that these wave packets form a complete set.
	\end{enumerate}
\end{Boxed}
	\begin{enumerate}[a)]
		\item
		\[
			\intinf P_{jl} (t) P_{j'l'}(t) \di t	
			=
			\intinf 
\frac 1 {\sqrt \eps} \int_{j\eps}^{(j+1)\eps} e^{2\pi i l \omega_1/\eps} 	
			\ispi e^{-i\omega_1 t} \di \omega_1
\frac 1 {\sqrt \eps} \int_{j'\eps}^{(j'+1)\eps}\ispi
			\overline{e^{2\pi i l' \omega_2/\eps} 	
			 e^{-i\omega_2t}} \di \omega_2
\di t
		\]
		\[
			=
\frac 1 {\eps 2 \pi } 
			\intinf 
\int_{j\eps}^{(j+1)\eps} 
 \int_{j'\eps}^{(j'+1)\eps}
			e^{2\pi i l \omega_1/\eps} 	
			e^{-i\omega_1 t} 
 	                e^{-2\pi i l' \omega_2/\eps} 	
			e^{i\omega_2t} \di \omega_1 \di \omega_2
\di t
		\]
		\[
			=
\frac 1 {\eps 2 \pi } 
\int_{j\eps}^{(j+1)\eps} 
 \int_{j'\eps}^{(j'+1)\eps}
 	                e^{-2\pi i l' \omega_2/\eps} 	
			e^{2\pi i l \omega_1/\eps} 	
			\intinf 
			e^{-i(\omega_1-\omega_2)t} 
\di t
			\di \omega_1 \di \omega_2
		\]

		\[
			=
\frac 1 {\eps } 
\int_{j\eps}^{(j+1)\eps} 
 \int_{j'\eps}^{(j'+1)\eps}
 	                e^{-2\pi i l' \omega_2/\eps} 	
			e^{2\pi i l \omega_1/\eps} 	
			\delta(\omega_1-\omega_2)
			\di \omega_1 \di \omega_2
		\]
			We see at this stage that we need $\omega_1 = \omega_2$ on some positive measure set, otherwise the whole endever will be 0. Thus to continue the calculation we can add in a $\delta_{jj'}$ to ensure that the integration domains coincde.

		\[
			=
			\delta_{jj'}
\frac 1 {\eps  } 
\int_{j\eps}^{(j+1)\eps} 
 	                e^{-2\pi i l' \omega_1/\eps} 	
			e^{2\pi i l \omega_1/\eps} 	
			\di \omega_1 	
			=
			\delta_{jj'}
\frac 1 {\eps } 
\int_{j\eps}^{(j+1)\eps} 
			e^{2\pi i (l-l') \omega_1/\eps} 	
			\di \omega_1 
			=
\frac \eps {\eps } 
			\delta_{jj'}
			\delta_{ll'}
			=
			\delta_{jj'}
			\delta_{ll'}
			\]
			The last equality follows from considering the $1/\eps$ periodicity of 
			$e^{2\pi i (l-l') \omega_1/\eps}$ whenever $l \neq l'$.



		\item
			As a student I once had would say: "we write it down and bash"
			\[
			\suminf j \suminf l P_{jl} (t) \bar P_{jl}(t')
			=
			\suminf j \suminf l 
\frac 1 {\sqrt \eps} \int_{j\eps}^{(j+1)\eps} e^{2\pi i l \omega/\eps} 	
			\ispi e^{-i\omega t} \di \omega
\frac 1 {\sqrt \eps} \int_{j\eps}^{(j+1)\eps} e^{-2\pi i l \omega/\eps} 	
			\ispi e^{i\omega t'} \di \omega
				\]
\[
			=
			\frac 1 {\eps 2 \pi}
			\suminf j \suminf l 
 \int_{j\eps}^{(j+1)\eps} e^{2\pi i l \omega_1/\eps} 	
    e^{-i\omega_1 t} \di \omega_1
 \int_{j\eps}^{(j+1)\eps} e^{-2\pi i l \omega_2/\eps} 	
			e^{i\omega_2 t'} \di \omega_2
				\]
		\[
			=
			\frac 1 {\eps 2 \pi}
			\suminf j \suminf l 
 \int_{j\eps}^{(j+1)\eps} 
 \int_{j\eps}^{(j+1)\eps} 
 e^{2\pi i l (\omega_1-\omega_2)/\eps} 	
    e^{-i(\omega_1t-\omega_2t') } 
    \di \omega_2 \di \omega_1
				\]
		\[
			=
			\frac 1 {\eps 2 \pi}
		\intinf 
		\intinf 
    e^{-i(\omega_1t-\omega_2t') } 
		\suminf l 
 e^{2\pi i l (\omega_1-\omega_2)/\eps} 	
    \di \omega_2 \di \omega_1
				\]

		\[
			=
			\frac 1 {\eps 2 \pi}
		\intinf 
		\intinf 
    e^{-i(\omega_1t-\omega_2t') } 
    \delta((\omega_1-\omega_2)/\eps)
    \di \omega_2 \di \omega_1
			=
			\frac 1 {2 \pi}
		\intinf 
    e^{-i\omega (t-t') } 
    \di \omega
    =
			\delta(t-t')
				\]
				Bashing complete, we arrive at the answer.
	\end{enumerate}
\end{Porb}


\begin{Porb}
\begin{Boxed}
	Consider the wave packet
	\[
		Q_{jl}(t) = \ispi \frac 1 {\sqrt \eps} \int_{(j-1/2)\eps}^{(j+1/2)\eps}
		e^{i \omega t} e^{-2 \pi i l \omega/\eps} \di \omega
		\]
		Express the summed wave packets:
	\begin{enumerate}[a)]
	\item 
		\[
			\suminf j Q_{jl} (t)
			\]
	\item
		\[
			\suminf l Q_{jl} (t)
			\]
	\item
		\[
			\suminf l\suminf j Q_{jl} (t)
			\]
	\end{enumerate}
	in terms of appropriate Dirac delta DISTRIBUTIONS if necessary.
\end{Boxed}
	\begin{enumerate}[a)]
	\item 
		\[
			\suminf j Q_{jl} (t)
			=
\ispi \frac 1 {\sqrt \eps} 	\suminf j	\int_{(j-1/2)\eps}^{(j+1/2)\eps}
		e^{i \omega t} e^{-2 \pi i l \omega/\eps} \di \omega
			=
\ispi \frac 1 {\sqrt \eps} \intinf	e^{i \omega t} 
			e^{-2 \pi i l \omega/\eps} \di \omega
			\]
			\[
			=
			 \frac 1 {\sqrt \eps} \intinf	\ispi e^{i \omega (t-2 \pi l/ \eps)} \di \omega
			=
			 \frac 1 {\sqrt \eps} \sqrt{2\pi} \delta ( t-2 \pi l/ \eps)
			\]
	\item
		\[
			\suminf l Q_{jl} (t)
			=
	\suminf l  \ispi \frac 1 {\sqrt \eps} \int_{(j-1/2)\eps}^{(j+1/2)\eps}
		e^{i \omega t} e^{-2 \pi i l \omega/\eps} \di \omega
			=
	 \ispi \frac 1 {\sqrt \eps} \int_{(j-1/2)\eps}^{(j+1/2)\eps}
		e^{i \omega t} \suminf l e^{-2 \pi i l \omega/\eps} \di \omega
			\]
			Now letting $y= -2 \pi \omega/\eps, \di y = -2\pi /\eps \di \omega$, we can change variables and evaluate the sum:
			\[
			=
	  \frac 1 {\sqrt \eps} \int_{(j-1/2)\eps}^{(j+1/2)\eps}
			e^{i \omega t} \sqrt{2 \pi}  \suminf l e^{i l(-2 \pi  \omega/\eps) } \di \omega
			=
			\frac 1 {\sqrt \eps} \int_{(j-1/2)(-2\pi)}^{(j+1/2)(-2\pi)}
			e^{i \eps y/(-2\pi) t} \sqrt{2 \pi}  \suminf l e^{i ly } \di y  \frac \eps {-2\pi }
			\]
			\[
			=
		\frac 1 {-2\pi }\sqrt \eps \int_{(j-1/2)(-2\pi)}^{(j+1/2)(-2\pi)}
			e^{i \eps y/(-2\pi) t} \sqrt{2 \pi}  \delta(y) \di y  
			=
		\frac 1 {-2\pi }\sqrt \eps \sqrt{2 \pi}
			\sum_{|j|\leq 3}
			\delta_{j,0}
			\]
			Since $0\in [j-\frac 1 2, j +\frac 1 2]$ iff $j=0,\pm1,\pm2,\pm3$.

	\item
		By the first part we immediatly see:
		\[
			\suminf l\suminf j Q_{jl} (t)
			=
			\suminf l
			 \frac 1 {\sqrt \eps} \sqrt{2\pi} \delta ( t-2 \pi l/ \eps)
			=
			\sqrt{ \frac {2\pi} {\eps} }
			\suminf l
			 \delta ( t-2 \pi l/ \eps)
		\]



	\end{enumerate}
	*******************DOUBLE CHECK
\end{Porb}



\subsection{Orthonormal Wavelet Representation}

\subsection{Multiresolutions Analysis}
\setcounter{Prb}{0}
%1
\begin{Porb}
\begin{Boxed}
	Show that
	\[
		\overline{ \bigcup_{k=-\infty}^\infty V_k} = L^2 \iff
		\lim_{k\rw \infty} \norm{ P_{V_k} f - f } =0 
	\]
	where $P_{V_k}$ is the projection onto $V_{-k}$ (the sign is fliped to make the limits easier to write) and the norm is the $L^2$ norm.
\end{Boxed}
We go forward first:
	\[
		\overline{ \bigcup_{k=-\infty}^\infty V_k} = L^2
		\]
		Thus given an $f\in L^2, \exists h_n,\ st.$ $\lim_{n \rw \infty} \norm{h_n -f} =0, h_n \in \bigcup V_k$.
		Now $\forall n$ $\exists k \ st. h_n \in V_k$. Now either $h_n = P_{V_k} f$ or $\norm{ h_n - f} \geq \norm{P_{V_k} -f}$ and we can replace $h_n$ with the actual projection without making the approximation any worse. 
		Obviously our new $\bar h_n$ still converges and is made entirly of projections onto subspaces. Thus we have constructed the desired sequence. (We may need to additonally doctor the sequence and insert terms if $h_k$ skipped many subspaces.)
		***********************HATE HOW THIS IS WRITTEN************************8
	
Now we go backwards:
	Let f again be some funciton in $L^2$, then $\norm{ P_{V_k} f -f } \rw 0 $ as $ k \rw \infty$. Thus there exists some sequence $h_k = P_{V_k} f$ where $\norm{h_k -f } \rw 0$ as $k \rw \infty$. 
	Additionally $h_k\in V_k \forall k$ thus $\forall f \in L^2, f \in \overline{ \bigcup V_kk}$.
	The reverse inclusion is obvious and we are done. 
\end{Porb}


%2
\begin{Porb}
\begin{Boxed}
	Show that
	\[
		\bigcap_{k=-\infty}^\infty V_k =\{0\} \iff
		\lim_{k\rw \infty} \norm{ P_{V_k} f } =0 
	\]

\end{Boxed}
First we go forward:\\
	Notice that $\norm{P_{V_k} f}$ is a decreasing sequence and thus has some limit. Now suppose for controdiciton that
	$\norm{P_{V_k} f} \rw \eps >0$ as $k \rw \infty$.
	Notice now that this is a Cauchy sequence in $L^2$ and thus there is some function $g \in L^2$ st. 
	$ \norm{P_{V_k} f -g } \rw 0$.
	Now we wish to show that $g \in V_k, \forall k$. Suppose it is not, there is some $k_0$ st. g is no longer in any of the $V_k$'s after $k_0$.
	But then (since $\overline{\bigcap V_k} = \{0\}$) there would be some nonzero gap that emerges between $P_{V_k}f$ and g, namely that:
	$ \norm{P_{V_k} f -g } \geq d(V_k, g) = \eps_g >0$.
	Thus $g\in V_k, \forall k$ but then $g \in \bigcap V_k$ which then means $g=0$ and thus 
	$ \norm{P_{V_k} f } \rw \norm g =0$.



Then we go back:\\
We do this by controdiciton, so suppose that 
		$\bigcap_{k=-\infty}^\infty V_k \supseteq \{0, g\} $ for some nonzero functions $g$. 
		Then $\norm{g} >0$ and since $g\in \bigcap_{k=-\infty}^\infty V_k, \rw g \in V_k \forall k$. 
		Thus $g \in P_{V_k}$ for all k and $\lim_{k \rw \infty} \norm{ P_{V_k} g} =\norm g >0$.
		This is a controdicition and we see that there is no g.
\end{Porb}

%3
\begin{Porb}
\begin{Boxed}
	\begin{enumerate}[a)]
		\item Show that $V_0$ is discrete translation invarient, ie. whenever $l \in \ZZ$  that:
			\[
				f(t) \in V_0 \iff f(t-l) \in V_0
				\]
		\item Show that $V_k$ is $2^k$ shift invariant, ie with $l,k \in \ZZ$ that:
			\[
				f(t) \in V_0 \iff f(t-2^kl) \in V_0
				\]
	\end{enumerate}
\end{Boxed}
	\begin{enumerate}[a)]
		\item 		
			Suppose $f \in V_0$ then $\exists \alpha_l$ st. $f(t) = \sum_l \alpha_l \phi(t-l)$.
			By the construction of the basis of $V_0$.
			Notice that for $k \in \ZZ$
			\[
				f(t-k) = \sum_l \alpha_l \phi(t-k-l)
				= \sum_{m= k+l} \alpha_{m-k} \phi(t-m)
				= \sum_{m= k+l} \alpha_m' \phi(t-m)
				\]
				Thus we still have an expansion for $f(t-l)$ in terms of the original basis.
		\item Similar tricks:
			\[
				f(t) \in V_k \Rw 
				f(2^kt) \in V_0 \Rw 
				\]
				Now we remember that shifting by a constant value keeps you in $V_0$.
				$
					f(2^k(t-j)) = f(2^kt-2^kj) \in V_0
				$\\
				Now scaling the $t$ by $2^{-k}$ will get us back to $V_k$, that is: $f(t - 2^kj) \in V_k$.
	\end{enumerate}
\end{Porb}



















%4
\begin{Porb}
\begin{Boxed}
	\begin{enumerate}[a)]
		\item Point out why this inner product is the $(l,l')$th entry of the $\sqrt 2$- multilple of a unitary matrix, which is independent of k.
		\item Show that $\suminf l \bar h_l h_{l-2l'} = \delta_{0l'}$
	\end{enumerate}
\end{Boxed}

	\begin{enumerate}[a)]
		\item
			Let M be the map from $V_k \rw V_{k+1}$ that is Change of basis from two orhtonormal basis? So it sunitary and stuff?
			**********************************************************************

		\item 
			If $l' =0$ then we have:
			\[
			\suminf l \bar h_l h_{l-2l'} 
			=
			\suminf l | h_l |^2
			=
			\suminf l \frac 1 4 | \intinf \bar \phi ( u-l) \phi ( u/2) \di u |^2
			\]
			We notice now that $h_l$ is the coefficent of the projection of $\phi(u/2)$ onto the space $V_0$. Thankfully $\phi(u/2) \i V_1 \subset V_0$. 
			Thus we see that by Parsevel's that:
			\[
			\suminf l \bar h_l h_{l-2l'} 
			=
			\norm{ \phi(u/2)}^2 \frac 1 4
			=
			\frac 4 4 = 1
			\]

			If $l \neq 0$ then:
			\[
			\suminf l \bar h_l h_{l-2l'} 
			=
			\suminf l  
			\overline{
		\intinf \bar \phi ( u-l) \phi(u/2) \di u
				}
			\intinf \bar \phi ( u-(l-2l')) \phi(u/2) \di u
			\]


		********************************\\	


	\end{enumerate}
\end{Porb}

%5
\begin{Porb}
\begin{Boxed}
Verify the validity of the funcitonal constatin:
	\[
		|H(\omega)|^2+ |H(\omega+\pi)|^2 =1
		\]
\end{Boxed}
Begin with :
	\[
		\suminf n | \hat \phi  ( \omega + 2\pi n)|^2 = \frac 1 {2\pi}
		,
		\hat \phi( 2 \omega) = H ( \omega) \hat \phi( \omega)
		\]
		\[
			H( \omega ) = \frac{ \sqrt 2 } 2 \suminf l h_l e^{i\omega l}
			\]
		

\end{Porb}


%6
\begin{Porb}
\begin{Boxed}
Consider a function $\phi (t)$ having the property
	\[
		\left|
		\intinf \phi(t) \di t
		\right|
		\neq 0
		\]

		Find the solution to the scaling equation:
		\[
		\hat \phi( 2 \omega) = H ( \omega) \hat \phi( \omega)
		\]
	Answer/Hint: 
	\[
	\hat \phi (\omega) = \hat \phi (0) \prod_{k=1}^\infty H(\omega/2^k)
	\]
\end{Boxed}
	

\end{Porb}


%7
\begin{Porb}
\begin{Boxed}
	Let $\phi^+(t)$ be soltuion to the scaling equation
	\[
		\phi(t) = \sqrt 2 \suminf l h_l \phi( 2 t -l)
		\]
		\begin{enumerate}
			\item
				Point out why
				\[
					\hat \phi^- = \begin{cases}
						\hat \phi^+ ( \omega) & \omega \geq 0 \\
						-\hat \phi^+ ( \omega) & \omega \leq 0 
					\end{cases}
					\]
					is the Fourier transform of a second independent solution to the above scaling equaiton.
			\item
				Show that these two solutions are orhtogonal:
				\[
					\intinf \bar \phi^+ \phi^- \di t = 0
					\]
					whenever $\phi(t)$ is a real funciton or whenever its Fourier transform is an even function of $\omega$.
		\end{enumerate}
\end{Boxed}



\end{Porb}





%8
\begin{Porb}
\begin{Boxed}
	Validate conclusion \# II of the theorem on page 145. Point out why, whenever $k \neq k'$, the funcitons in $O_k$ are orthogonal to $O_{k'}$.
\end{Boxed}


\end{Porb}





\section{Strum-Liouville Theory}
\setcounter{subsection}{2}
\subsection{Strum-Liouville Systems}
\setcounter{Prb}{0}

%1
\begin{Porb}
\begin{Boxed}
	\begin{enumerate}[a)]
		\item Show that any equation of the form
	\[
		u''+ b(x) u'+ c(x) u=0
	\]
		can always be brought into the Shrodinger form:
		\[
			v'' + Q(x) v=0
		\]
	Apply this result to obtain the Schrodinger form for:
	\item 
		\[
			u''-2xu' +\lambda u=0
		\]
	\item 
		\[
			x^2u''+xu'+(x^2-\nu^2)u=0
		\]
	\item 
		\[
			xu''+(1-x)u'+\lambda u=0
		\]
	\item 
		\[
			(1-x^2)u''- xu'+\alpha^2 u=0
		\]
	\item 
		\[
			(pu')'+ (q+\lambda r) u=0
		\]
	\item 
		\[
			\left[ \frac 1 {\sin \theta} \frac \di {\di \theta} \sin \theta  \frac \di {\di \theta} 
			+ l (l+1) - \frac {m^2} {\sin^2\theta} 
			\right]u =0
		\]

	\end{enumerate}
	
\end{Boxed}
*************************************************
	\begin{enumerate}[a)]
		\item
			We consider:
	\[
		u''+ b(x) u'+ c(x) u=0
	\]
	*********************PACKT******************\\
			Let u be some solution, let us try $v(x) = F(x) u(x)$ 
			then:
			\[
				v' = F' u + F u', 
				v'' = F'' u + 2F'u' + F u''
				\]

			Now we plug this into $v''+Qv$ and find:
			\[
				 F'' u + 2F'u' + F u''
				 +
				 Fu
				 =
				 F'' u + 2F'u' + F (u'' +Qu)
				\]
				
				\[
				 =
				 F \left[ 
				  u'' +2F'/Fu' +(F''/F +Q)u 
				  \right]
				\]
				We know that $u'' = -bu'-cu$
				\[
				 =
				 F \left[ 
				  (2F'/F-b)u' +(F''/F +Q-c)u 
				  \right]
				\]
				If we let $Q = c- F''/F$ then all we have to do is solve $2F'/F =b$.
				This leads to
				\[
					2 F'/F -b=0 \Rw 2 \int^x F'/F = \int^x b
					\Rw 2 \ln F = \int^x b
				\]

					\[
						\Rw F = \exp \{ \frac 1 2 \int^x b \}
					\]

			Thus our subsitition ends up being: $v(x) = \exp \{ \frac 1 2 \int^x b \} u(x)$. 
			Note that $F' = \frac {b} 2 F, \ F'' = \frac {b'+ b^2/2} 2 F$
 and our equation gets:
			\[
				Q= c- F''/F 
				=
				 c- F''/F
				 =
				c- \frac {b'+ b^2/2} 2
			\]
			All togther we have:
	\[
		u \rw v = \exp \{ \frac 1 2 \int^x b \} u(x), \quad
		\]
		\[
		u''+ b(x) u'+ c(x) u=0
		\ \rw \
		v''+Q(x) v=0, \ Q(x) = 	c- \frac {b'+ b^2/2} 2
	\]


	\item 
		\[
			u''-2xu' +\lambda u=0
		\]

		Thus the things we have to calcuate are:
		\[
F = \exp \{ \frac 1 2 \int^x b \}, \quad
Q= c- \frac {b'+ b^2/2} 2
\]
		\[
F = \exp \{ \frac 1 2 \int^x -2 y \di y \}, \quad
Q= \lambda- \frac {-2+ 4 x^2/2} 2
\]
	And our equations become:
\[
	v= \exp \{  -x^2/2   \} u, \quad
v'' + \left(\lambda+1- x^2 \right) v =0
	\]


	\item 
		\[
			x^2u''+xu'+(x^2-\nu^2)u=0
			=
			u''+u'/x+(1-\frac{\nu^2}{x^2})u=0
		\]
		We divide by $x^2$ here to get ride of the coefficent on u''.
		Now following the formeioli
\[
F = \exp \{ \frac 1 2 \int^x b \}, \quad
Q= c- \frac {b'+ b^2/2} 2
\]
		\[
F = \exp \{ \frac 1 2 \int^x \frac 1 y \di y \}, \quad
Q= 1-\frac{\nu^2}{x^2} - \frac { - 1/x^2 + \frac 1 {x^22} } 2
\]
\[
Q= 1-\frac{\nu^2}{x^2} + \frac 1 {4x^2} 
= 1-\frac{4\nu^2-1}{4 x^2}  
\]
	And our equations become:
\[
	v= \exp \{  \ln (x) /2   \} u = \sqrt x u , \quad
v'' + \left(1-\frac{4\nu^2-1}{4 x^2}  \right) v =0
	\]







	\item 
		\[
			xu''+(1-x)u'+\lambda u=0
			= u''+(1/x-1)u'+ \frac \lambda x u
		\]
		Again we do the divide by the whole something that could be zero trick to deal with a coefficent.
\[
F = \exp \{ \frac 1 2 \int^x 1/y-1 \di y \}, \quad
Q= \frac \lamb x - \frac {- \frac 1 {x^2}+ (1/x-1)^2/2} 2
\]
\[
	F = \exp \{ \frac 1 2 [\ln x-x ] \}, \quad
Q= \frac \lamb x - \frac {- \frac 1 {2x^2}+ -1/x+1/2} 2
\]
\[
	F = \exp \{ \frac 1 2 [\ln x-x ] \}, \quad
Q= \frac \lamb x + \frac 1 {4x^2}+ 1/2x-1/4
\]

********************************\\



	\item 
		\[
			(1-x^2)u''- xu'+\alpha^2 u=0
		\]

\[
F = \exp \{ \frac 1 2 \int^x b \}, \quad
Q= c- \frac {b'+ b^2/2} 2
\]






	\item 
		\[
			(pu')'+ (q+\lambda r) u=0
		\]

\[
F = \exp \{ \frac 1 2 \int^x b \}, \quad
Q= c- \frac {b'+ b^2/2} 2
\]






	\item 
		\[
			\left[ \frac 1 {\sin \theta} \frac \di {\di \theta} \sin \theta  \frac \di {\di \theta} 
			+ l (l+1) - \frac {m^2} {\sin^2\theta} 
			\right]u =0
		\]

\[
F = \exp \{ \frac 1 2 \int^x b \}, \quad
Q= c- \frac {b'+ b^2/2} 2
\]







	\end{enumerate}
\end{Porb}


%2
\begin{Porb}
\begin{Boxed}
	Consider the S-L eigenvalue problem:
	\[
		[Lu_n] (x) = \left( - \frac{ d^2} { dx^2} + x^2\right)u_n(x) = \lambda_n u_n(x), \quad \lim_{x\rw \pm \infty} u(x)= 0
		\]
		on the infinite interval $(-\infty, \infty)$\\
	Show that the eigenvalues $\lambda_n$ are nondegenerate, ie. show that, except for a constant multiple, the correpsonding eigenfunctions are unique.
\end{Boxed}
	By Abel's theorem we have that if two solutions to the above have the same eigenvalue then (since p(x) =1 here)
	\[
		u_m u_n' - u_m' u_n = const.
		\]
		From here it suffices to show that this constant is zero. Once there the same logic as in the end of theorem 3 applies and we would see that $\frac {u_m'} {u_m} = \frac {u_n'} {u_n} \Rw u_m = k u_n$.
		To this end notice that 

		\[
		\lim_{x\rw \infty}
			 u_m u_n' - u_m' u_n = C \Rw 
			\]

			******************************************\\

**********Packet Note*********************\\


\end{Porb}



%3
\begin{Porb}
\begin{Boxed}
	Consider the "parity" operator $P: L^2 \rw L^2$ ($L^2 = L^2(-\infty, \infty)$) defined by 
	\[
		P\psi (x) = \psi (-x)
		\]
		\begin{enumerate}
			\item For a given function $\psi$ what are the eigenvlaues and eigenfunction of P?
			\item Show that the eigenfunctions of the operator L defined in problem 3.3.2 are eigenfunctions of P. Do this by computing 
				\[
					\I P L P \psi (x) 
					\]
					for $\psi \in L^2$ and the pointing out how $\I P L P$ is related to L.
						Next point out how this relationship applied to an eigenfunction $u_n$ of the previous problem leads ot the result $P  u_n = \mu u_n$.
		\end{enumerate}
\end{Boxed}
		\begin{enumerate} [i)]
			\item
				The given funciton part of the question is a typeo. 
				The eigenvalues are $\pm 1$ ($\psi(x) = \lamb \psi(-x)$) and the eigenfuncitons are even and odd functions.
			\item
				\[
					\I P L P \psi(x) =
					\I P L \psi(-x) =
					\I P \left[- \frac{ d^2} { dx^2} + x^2 \right] \psi(-x) 
					\]

					\[
					=
					\I P \left[- \frac{ d^2 \psi (-x)} { dx^2} + x^2 \psi(-x) \right] 
					=
					 \left[- \I P\frac{ d^2 \psi (-x)} { dx^2} + \I Px^2 \psi(-x) \right] 
					\]
					\[
					=
					 \left[- \I P\frac{ -d \psi' (-x)} { dx} + x^2 \psi(x) \right] 
					=
					 \left[- \I P \psi'' (-x)  + x^2 \psi(x) \right] 
					=
					 - \psi'' (x)  + x^2 \psi(x)  
					 = L \psi
					\]

					Thus $\I P L P = L$. 
					*******************************************\\

					Thus if $\mu, u_n$ are eigenvector 



		\end{enumerate}

\end{Porb}

%4
\begin{Porb}
\begin{Boxed}
	Consider the S-L eigenvalue problem:
	\[
		[Lu_n] (x) = \left( - \frac{ d^2} { dx^2} + x^2\right)u_n(x) = \lambda_n u_n(x), \quad \lim_{x\rw \pm \infty} u(x)= 0
		\]
		on the infinite interval $(-\infty, \infty)$\\
		We are now blessed with the knowledge that these eigenvalues are nondegenerate and are $\lambda_n = 2n +1$.
		Consider now the Fourier transform on $L^2$
		\[
			\FF u = \ispi \intinf e^{-ikx } u \di x
			\]
		\begin{enumerate} [a)]
			\item
				By computing $\FF L \I \FF \hat \psi $ for an aribitary $\hat \psi \in L^2$, determine the Fourier represnetation of $\FF L \I \FF = \hat L$, of the operator $ L = - \frac{ d^2} {d x^2} + x^2$.
			\item By viewing $\FF$ as a map from $L^2$ to itself, compare the operators $\hat L$ and $L$.
				State in english senence and in math equation.
			\item  Us the result of b to show that each eigenfunction $u_n$ of the S-L operator L is also an eigenfunction of $\FF$.
				\[
					\FF u_n = \mu u_n
					\]
					By applying the result (e) of the exercise on page 75, determine the only allowed values for $\mu$. What \emph{is} the 4-ier transform of a Hermite-Gauss polynomial?
		\end{enumerate}
\end{Boxed}

		\begin{enumerate}
			\item
				\[
		\FF L \I \FF \hat \psi 
				=
		\FF L \psi 
				=
\FF \left( - \frac{ d^2} { dx^2} + x^2\right)\psi(x) 
				\]
				\[
				=
\FF \left( - \frac{ d^2} { dx^2} + x^2\right)\psi(x) 
				=
				-\FF  \frac{ d^2\psi (x)} { dx^2} + \FF x^2\psi(x) 
				\]
				\[
					=
				- (i \omega )^2 \FF \psi [\omega]  + i^2 \dd { ^2} {\omega^2}  \FF \psi  [\omega]
					=
				 \omega^2 \FF \psi [\omega] -  \dd { ^2} {\omega^2}  \FF \psi [\omega]
				 =
				 \left[ \omega^2  -  \dd { ^2} {\omega^2} \right]  \hat \psi [\omega]
				\]

				Thus 
				\[
					\hat L 
					=
				  \omega^2  -  \dd { ^2} {\omega^2}
				\]



			\item
				They do the same thing, one in frequency space and the other in real space.
				\[
					\hat L \hat \psi (\omega ) = L \phi (\omega)
					\]

					******************DOUBLE CHECK********************\\

			\item
				**************************\\



		\end{enumerate}

\end{Porb}

%5
\begin{Porb}
\begin{Boxed}
Consider the S-L System:
	\[
		\left[ 
		\frac d {dx} p \frac d {dx} + q + \lambda \rho
		\right] u
		=0
		, \quad a < x <b 
		\]
		\[
			\alpha u(a) + \alpha ' u' (a) =0; 
			\quad
			\beta u(a) + \beta ' u' (a) =0
			\]
		Let $\omega(x, \lambda)$ be that unqiue soltuion to the above with boundary conditions satisified. 
		Then $\omega_n(x) = \omega(x, \lambda_n)$ is an eigenfunction with eigenvalue $\lambda_n$. Calculate $\int_a^b \omega_n^2 \rho \di x$ as follows:
		
		\begin{enumerate}
			\item
				\[
					(\lambda - \lamb_n) \int_a^b \omega_n(x) \omega(x, \lambda) \rho(x) \di x= 
			p(x) W(\omega, \omega_n) |^{x=b}_a
					\]
			\item
				By taking the limit $\lamb \rw \lamb_n$ show that:
				\[
					\int_a^b \omega_n^2= 
					p(b) \left[
						w'_n(b) \frac{d \omega(b, \lamb)} {d \lamb} |_{\lamb=\lamb_n} 
						- \omega(b) 
						\frac d {d \lamb} 
						w'_n(b, \lamb) |_{\lamb=\lamb_n} 
						\right]
					\]
					primes here refering to $\frac d {dx}$.
		\end{enumerate}
\end{Boxed}
	\begin{enumerate}
		\item Thanks to the first 2 steps of the 3 step proof on page 168 of the orthogonality of eigenvalues we see:
		\[
			(\lambda - \lamb_n) \int_a^b \omega_n(x) \omega(x, \lambda) \rho(x) \di x= 
			p(x) W(\omega, \omega_n) |^{x=b}_a
		\]
		In fact we realize this is true for any S-L system with any boundary condiotns. With that aside aside, we consider the case at hand.

		\[
			p(x) W(\omega, \omega_n) |^{x=b}_a
			=
			p(b) W(\omega, \omega_n)(b) 
			-
			p(a) W(\omega, \omega_n)(a) 
			\]

			note that the boundary condition gives: $ \omega( a, \lamb) = - \alpha'/\alpha \omega'(a, \lamb)$
			Thus we have:

			\[
			p(a) W(\omega, \omega_n)(a) 
			=
			p(a) (\omega(a) \omega_n'(a) - \omega'(a) \omega_n(a)) 
			\]
			\[
			=
			p(a) (-\omega'(a) \alpha'/\alpha \omega_n'(a) + \omega'(a) \omega_n'(a) \alpha'/\alpha ) = 
			0
			\]

			Thus we get:
		\[
			(\lambda - \lamb_n) \int_a^b \omega_n(x) \omega(x, \lambda) \rho(x) \di x= 
			p(x) W(\omega, \omega_n) |^{x=b}
		\]


	\item
		We first divide by $\lamb - \lamb_n$:
		\[
		 \int_a^b \omega_n(x) \omega(x, \lambda) \rho(x) \di x= 
			\frac 1 {\lambda - \lamb_n}	p(x) W(\omega, \omega_n) |^{x=b}
		\]

		Now we take the limit $\lamb \rw \lamb_n$:
			the left hand side clearly does nothing funky and becomes:
		\[
		 \int_a^b \omega_n(x)^2  \rho(x) \di x
			\]

			Now for the right hand side:
			\[
			\frac 1 {\lambda - \lamb_n}	p(x) W(\omega, \omega_n) |^{x=b}
			=
			\frac 1 {\lambda - \lamb_n}	p(b) 
			(\omega(b) \omega_n'(b) - \omega'(b) \omega_n(b)) 
				\]
				\[
			=
			 	p(b) 
			(\frac{ \omega(b)} {\lambda - \lamb_n} \omega_n'(b) - \frac{ \omega'(b)} {\lambda - \lamb_n} \omega_n(b)) 
				\]
				We add and subtract the same term:
				\[
			=
			 	p(b) 
			(\frac{ \omega(b) - \omega_n(b)} {\lambda - \lamb_n} \omega_n'(b) - \frac{ \omega'(b) - \omega_n'(b) } {\lambda - \lamb_n} \omega_n(b)) 
				\]

				Pasing to the limit we get:
				\[
					=
					p(b) \left[
						w'_n(b) \frac{d \omega(b, \lamb)} {d \lamb} |_{\lamb=\lamb_n} 
						- \omega(b) 
						\frac d {d \lamb} 
						w'_n(b, \lamb) |_{\lamb=\lamb_n} 
						\right]
					\]
					which is the desired result.
	\end{enumerate}
\end{Porb}



%6
\begin{Porb}
\begin{Boxed}
	Consider the S-L problem
	\[
		\left[
			- \dd {} x x \dd {} x + \frac {\nu^2} {x} 
			\right]u = \lambda x u
		\]
	Here u, $\dd u x$ bounded as $x\rw 0, u(1) =0$ and $\nu \in \RR$.
		\begin{enumerate}
			\item Using the sub $t= \sqrt {\lamb x}$ show that the above differential equation reduces to Bessel's equation of order $\nu$. One solution which is bouned as $t \rw 0$ is $J_\nu (t)$; a second linearly indep. solution, denoted by $Y_\nu(t)$ is unbounded as $t \rw 0$.

			\item
				SHow that the eigenvalues of the given problem are the squares fo the positive zeros of $J_\nu ( \sqrt \lamb)$ and that the corresponding eigenfunctions are 
				\[
					u_n (x) = J_\nu ( \sqrt {\lamb_n} x)
					\]

			\item
				Show that the eigenfunctions $u_n$ satisfy the orhtogonality relaiton:
				\[
					\int_0^1 x u_m u_n \di x =0, \quad m \neq n
					\]

			\item
				For the case $\nu =0$, apply the method of the previous problem to exhibit the set of orhtonormalized eiqgenfuncitons.
			
			\item Determine the coeffiencts of the Fourier-Bessel sereis expnaison:
				\[
					f(x) = \sum_{n=1}^\infty c_n u_n(x)
					\]
		\end{enumerate}
\end{Boxed}

		\begin{enumerate}
			\item 
				The bessel equation is:
				\[
					\left[
						x^2 \dd {^2} {x^2} + x \dd {} x + x^2 - \alpha^2 \right] u = 0
					\]

				So begining with:
				Let $t = \sqrt {\lamb }x $, then $ t/\sqrt \lamb = x,\text{ and } \dd t x = \sqrt \lamb$, 
				$\dd {} x = \dd {} t \dd t x = \dd {} t \sqrt \lamb $
				Our equation is:
		\[
			\left[
			- \dd {} x x \dd {} x + \frac {\nu^2} {x} 
			\right]u = \lambda x u
		\]
				We replace with $\dd {} t$ to get:
		\[
			\left[
			-   \dd t x\dd {} t t/\sqrt \lamb  \dd t x\dd {} t + \frac {\nu^2} {t/\sqrt \lamb} 
			\right]u = \lambda t/ \sqrt \lamb u
		\]
		\[
			\left[
			-   \sqrt \lamb \dd {} t t \dd {} t +\sqrt \lamb \frac {\nu^2} {t} 
			\right]u = t \sqrt \lamb u
		\]
				Now multiplying across by $t /\sqrt \lamb$ we get:
		\[
			\left[
			-    t \dd {} t t \dd {} t +  \nu^2 
			\right]u = t^2   u
		\]
		\[
			\left[
			-    t \dd {} t t \dd {} t +  \nu^2 
			- t^2 \right]u = 0
		\]
		\[
			\left[
				t \dd {^2} {t^2}+   \dd {} t -  \nu^2 
			+ t^2 \right]u = 0
		\]
				Which is plainly the Bessel equation of order $\nu$.

%				Let $t = \sqrt {\lamb x} $, then $ t^2/\lamb = x,\text{ and } \dd t x = \sqrt \lamb \frac 1 {2 \sqrt x}$, 
%				$\dd {} x = \dd {} t \dd t x = \dd {} t \sqrt \lamb $
%		\[
%			\left[
%			- \dd {} x x \dd {} x + \frac {\nu^2} {x} 
%			\right]u = \lambda x u
%		\]
%				We replace with $\dd {} t$ to get:
%		\[
%			\left[
%			-   \dd t x\dd {} t t^2/\lamb  \dd t x\dd {} t + \frac {\nu^2} {t^2/\lamb} 
%			\right]u = \lambda t^2/\lamb u
%		\]
%		\[
%			\left[
%			-  \sqrt \lamb \frac 1 {2 \sqrt x} \dd {} t  t^2/\lamb \sqrt \lamb \frac 1 {2 \sqrt x} \dd {} t + \frac {\lamb \nu^2} {t^2} 
%			\right]u = t^2 u
%		\]
%		\[
%			\left[
%				-   \frac 1 {2 t/\sqrt \lamb} \dd {} t  t^2  \frac 1 {2 t/\sqrt \lamb} \dd {} t + \frac {\lamb \nu^2} {t^2} 
%			\right]u = t^2 u
%		\]
%		\[
%			\left[
%				-  \lamb \frac 1 {4 t} \dd {} t  t \dd {} t + \frac {\lamb \nu^2} {t^2} 
%			\right]u = t^2 u
%		\]
%		\[
%			\left[
%				-  \lamb \frac 1 {4 t}  \left(  t\dd { ^2} {t^2} + \dd {} t \right) + \frac {\lamb \nu^2} {t^2} 
%			\right]u = t^2 u
%		\]
%		\[
%			\left[
%				-   \frac t {4}  \left(  t\dd { ^2} {t^2} + \dd {} t \right) + \nu^2  
%				\right]u = \frac {t^4} \lamb u
%		\]
%		\[
%			\left[
%				-   \frac 1 {4}  t^2\dd { ^2} {t^2} - \frac t 4 \dd {} t + \nu^2  -  \frac {t^4} \lamb 
%				\right]u = 0
%		\]
%				From here we can see that the quesion has some issues and that the equation can not go into the form that we wanted with t as the variable.
%
%		\[
%			\left[
%				-   \frac 1 {4} \left( t^2\dd { ^2} {t^2} +  t \dd {} t \right) + \nu^2  -   t^2 x  
%				\right]u = 0
%		\]
%				Rewriten as above we see that we simply eed to change the original equation to:
%		\[
%			\left[
%			- 4 \dd {} x x \dd {} x + \frac {\nu^2} {x} 
%			\right]u = \lambda u
%		\]
%				TO make the suggested substitiution work and the resulting ODE the Bessel equation of the order we wanted, namely $\nu$.
%


			\item
				\[
					u_n (x) = J_\nu ( \sqrt {\lamb_n} x)
					\]
					Let $u_n(x) $ be a solution of the above equation, then 
					
					**********************************\\

			\item
				We know that in genral for a S-L problem with eigenvectors $u_n, u_m$ that we have:
				\[
					(\lambda_m - \lamb_n) \int_a^b \omega_n(x) \omega_m(x) \rho(x) \di x= 
			p(x) W(\omega_m, \omega_n) |^{x=b}_a
				\]
				In our specific case we have:
				\[
					(\lambda_m - \lamb_n) \int_0^1 \omega_n(x) \omega_m(x) x \di x= 
			p(x) W(\omega_m, \omega_n) |^{x=1}_0
				\]
				Since the eigenvalues are disitinct it suffices to show from here that 
				\[
			p(x) W(\omega_m, \omega_n) |^{x=1}_0 =0
			\]
			Since $u(1)=0$ we see that $
			p(x) W(\omega_m, \omega_n) |^{x=1} =0$.
			and since $p(x) =x, p(0)= 0$ and we get our result.

			\item

				We have that
				\[
					\omega ( x, \lamb) = J( \sqrt \lamb x), \quad 
					\omega ' = \sqrt \lamb J'( \sqrt \lamb x), \quad
					\dd {} \lamb \omega ( x, \lamb) = \frac x {2 \sqrt \lamb}  J'( \sqrt \lamb x)
					\]

					\[
					\dd {} \lamb \omega' ( x, \lamb) =
					\dd {} \lamb\sqrt \lamb J'( \sqrt \lamb x)
					=
					\frac 1 {2 \sqrt \lamb}  J'( \sqrt \lamb x)
					+ 
					\sqrt \lamb \frac x {2 \sqrt \lamb}  J''( \sqrt \lamb x)
					\]

					\[
					=
					\frac 1 {2 \sqrt \lamb}  J'( \sqrt \lamb x)
					+ \frac x {2}  J''( \sqrt \lamb x)
					\]



					And the normalization integral from the previous problem
				\[
					\int_0^1 \omega_n^2 \di x
					=
					p(1) \left[
						w'_n(1) \frac{d \omega(1, \lamb)} {d \lamb} |_{\lamb=\lamb_n} 
						- \omega(1) 
						\frac d {d \lamb} 
						w'_n(1, \lamb) |_{\lamb=\lamb_n} 
						\right]
					\]

				\[
				=
					\sqrt \lamb J'( \sqrt \lamb 1)
					\frac 1 {2 \sqrt \lamb}  J'( \sqrt \lamb )
					-  
					J( \sqrt \lamb_n )
					\left(
					\frac 1 {2 \sqrt \lamb}  J'( \sqrt \lamb )
					+ \frac 1 {2}  J''( \sqrt \lamb )
					\right)
				\]

				\[
				=\frac 1 {2 }
				\left[
				J'( \sqrt \lamb )^2
					-  
					J( \sqrt \lamb )
					\left(
					\frac 1 {\sqrt \lamb}  J'( \sqrt \lamb )
					+   J''( \sqrt \lamb )
					\right)
					\right]
				\]


					***********************************\\

			
			\item
				\[
					f(x) = \sum_{n=1}^\infty c_n u_n(x)
					\]
		\end{enumerate}

\end{Porb}






%7
\begin{Porb}
\begin{Boxed}
	Consider the S-L problem
	\[
		\left[
			- \dd {} x (1-x^2) \dd {} x + \frac {m^2} {1-x^2} 
			\right]u = \lambda  u
		\]
		Here u, $\dd u x$ bounded as $x \rw \pm 1$. Here m is an integr. The solutions to this S-L problem are $u_n = P_n^m(x)$ the "asosciated Legendre polynomials", corresponding to $\lambda_n = n(n+1)$ n being an integer. SHow that:
		\[
			\int_{-1}^1 P_n^m P_{n'}^m \di x = 0 \quad \lambda_n \neq \lambda_{n'}
			\]
\end{Boxed}


	For $\lambda_n \neq \lambda_{n'}$ we have (via the same Green's idneity derived integral as before)
	\[
		(\lambda_n - \lambda_{n'} ) \int_{-1}^1 P_n^m P_{n'}^m \di x = 
			p(x) W(\lambda_{n}, \lambda_{n'}) |^{x=1}_{-1}
	\]
	Here we notice that $p(x) = 1-x^2$ and that $p(\pm 1) = 0$, (also the boundedness of $W$ as $x \rw \pm 1$ helps) thus we get:
	\[
			p(x) W(\lambda_{n}, \lambda_{n'}) |^{x=1}_{-1}
			=0
			\]
			and the problem is done.
\end{Porb}








%Ch4
\section{Green's Function Theory}
\setcounter{subsection}{2}
\subsection{Pictorial Definition of a Green's Function}
\setcounter{Prb}{0}

%1
\begin{Porb}
\begin{Boxed}
			Find the adjoint $L^*$ and the space on which it acts:
	\begin{enumerate}[a)]
		\item 
			\[
				Lu= u'' + a(x) u'+ b(x) u 
				\]
				with $u(0)=u'(1), \quad u(1) = u'(0)$
		\item
			\[
				Lu= -(p(x) u')' + q(x) u 
				\]
			 with $u(0)=u(1), \quad u'(1) = u'(0)$

	\end{enumerate}
\end{Boxed}
	\begin{enumerate}[a)]
		\item 
			\[
				\langle Lu, v \rangle 
				= \int_0^1 \left[u'' + a(x) u'+ b(x) u \right]v \di x
				\]
				Now we use integration by parts to get:
				\[
					= \left[u'v + a(x) uv \right]_0^1 - \int_0^1 u'v' + a(x) uv'- b(x) u v \di x
					= \left[u'v + a(x) uv - uv' \right]_0^1 + \int_0^1 uv'' - a(x) uv' + b(x) u v \di x
				\]
				Thus we see that $L^* v = v'' - a(x) v' + b(x) v $.
				Now considering the boundary conditions we see that the boundary terms are:
				\[
					u'(1)v(1) + u(1) \left(a(1) v(1) - v'(1) \right) 
					- \left[ u'(0)v(0) + u(0)\left(a(0) v(0) - v'(0) \right)  \right]
				\]
				\[
					=
					u'(1)v(1) + u(1) \left(a(1) v(1) - v'(1) \right) 
					- \left[ u(1)v(0) + u'(1)\left(a(0) v(0) - v'(0) \right)  \right]
				\]
				\[
					=
					u'(1) (v(1) -v(0))+ 
					u(1) \left(a(1) v(1) - v'(1) - \left(a(0) v(0) - v'(0) \right) \right) 
				\]
				From the $u'(1)$ coefficent we see we need $v(1) = v(0)$, this then leaves
				\[
					u(1) \left(- v'(1) + v'(0) \right) 
					\]
					which gives us $v'(1) = v'(0)$.

				Overall we see $L^* v  =v'' - a(x) v' + b(x) v $ and the adjoint domain being
				$\{ v | v'(1) = v'(0), v(1) = v(0), v \in C^2\}$.

				****************DOUBLE CHECK a(x) IS WIERD********************\\


		\item
			\[
				\langle Lu, v \rangle 
				= \int_0^1 \left[  -(p(x) u')' + q(x) u \right]v \di x
				= \int_0^1  -(p(x) u')'v + q(x) uv \di x
				\]
				Again we apply integration by parts:
				\[
				= -(p(x) u')'v |_0^1 + 
				\int_0^1  p(x) u'v' \di x +
				\int_0^1q(x) uv \di x
				\]

				\[
				= -p(x) u'v |_0^1 + 
				 u p(x) v'|_0^1 
				-\int_0^1   u (p(x) v')' \di x +
				\int_0^1q(x) uv \di x
				\]
				\[
				= -p(x) u'v |_0^1 + 
				 u p(x) v'|_0^1 
				+ \int_0^1   - u (p(x) v')' + q(x) uv \di x
				\]
				$L^* v = (p(x) v')' + q(x) v $

				********************************************\\
	\end{enumerate}
\end{Porb}

%2
\begin{Porb}
\begin{Boxed}
	Let L be a operator defined on $S$ and $L^*, S^*$ the adjoint and its domain satisifying $B_1(u)=0 = B_2(u), \ B_1^*(v)=0 = B_2^*(v)$ respectivly.
	Let $u,\lamb$, $v, \lamb'$ be eigenvalues, eigenvectors of $L$ and $L^*$
	\begin{enumerate}[a)]
		\item 
			Make a guess as to the relationship between the eigenvalue of $L$ and $L^*$.
		\item
			Prove: If $ \lambda \neq \bar \lambda'$ then $\langle u, v \rangle =0$.
	\end{enumerate}
\end{Boxed}
	Since part (ii) gives a guess we might as well say $\lambda$ cooresponds with $\bar \lamb$ for eigenvalues between L and $L^*$.

	We can see this with u and v as in the satement of the problem
	\[
		\langle  L u , v \rangle = \bar \lambda \langle u, v \rangle
		\]
		from the definition of eigenvalue/function.
	\[
	= 	\langle   u , L^* v \rangle = \lambda' \langle u, v \rangle
		\]
		Thus we see that if $\langle u , v \rangle \neq 0$ that $\bar \lamb = \lamb '$.


		*********8MAKE NOTE FOR LATER POST*****Problem is Done though*************\\
		In fact it is clear from the above that $\bar \lamb$ being an eigenvalue of the adjoint is implied by $\exists v, \langle u, v \rangle \neq 0$.
		Thus if an operator is self adjoint we see that $\lamb = \bar \lamb$ and the eigenvalues must all be real valued.

\end{Porb}


%3
\begin{Porb}
\begin{Boxed}
	Find the Green's function for the Bessel operators:

		\begin{enumerate}[a)]
			\item
				\[
					Lu(x) = \dd {} x x \dd {u(x)} x
					\]
			\item
				\[
					Lu(x) = \dd {} x x \dd {u(x)} x - \frac { n^2} x u(x) 
					\]
					with $y(0)$ finite and $y(1)=0$.
		\end{enumerate}
		ie. solve the equations $Lu = - \delta(x- \xi)$ with the given boundary conditions.
\end{Boxed}
		\begin{enumerate}[a)]
			\item
				Green's funciton is defined by:

			\[
				LG( x; \zeta) =  \dd {} x x \dd {G(x; \zeta) } x =- \delta( x - \zeta) 
			\]
				Now if we integrate both sides we get:
			\[
				 x \dd {G(x; \zeta) } x =- H( x - \zeta) 
				 \Rw
				  \dd {G(x; \zeta) } x =- \frac 1 x H( x - \zeta) 
			\]
				Where H is the heavyside function. Now we can integrate once more:
			\[
				G(x; \zeta)  =-\int^{x_0} \frac 1 x H( x - \zeta) \di x
			\]

**********************************************\\
THere are methods to solve these problems but they are from later sections and it is unclear if these are special cases or if I should just use the general method.
					
			\item
**********************************************\\
				\[
					Lu(x) = \dd {} x x \dd {u(x)} x - \frac { n^2} x u(x) 
					\]
					with $y(0)$ finite and $y(1)=0$.
		\end{enumerate}

\end{Porb}

%4
\begin{Porb}
\begin{Boxed}
	\begin{enumerate}
	\item	Find Green's function for the operator
	\[
		L = \ddt {} x + \omega^2
		\]
		with $u(a)= u(b) =0, \ a <b$ and $\omega^2$ some fixed cosntant.
	\item Does this Green's function exist $\forall \omega$? If not what values fail?

	\item
		Having found the Green's function in (1), go find the Green's function for the same operator but different boudnary conditions namely $u(a) = u'(a) =0$. Do this with minimal work.
	\end{enumerate}
\end{Boxed}

	\begin{enumerate}
	\item	
	\[
		L = \ddt {} x + \omega^2
		\]

		Same stuff as 4.7.2
		***********************************\\
	\item
	\item
	\end{enumerate}
\end{Porb}

%5
\begin{Porb}
\begin{Boxed}
	Let $Lu = u''$
	\[
		a_1 u(0) + b_1 u'(0) + c_1 u(1) + d_1 u'(1) =0 
		\]
	\[
		a_2 u(0) + b_2 u'(0) + c_2 u(1) + d_2 u'(1) =0 
		\]

\begin{enumerate}
	\item
		Find $L^*$ and the space on which it acts with $\langle u, v \rangle = \int_0^1 uv \di x$
	\item
		For what values of the cosntants is the operator self adjoint?
\end{enumerate}
\end{Boxed}

\begin{enumerate}
	\item
		As is standard we apply integration by parts.
		\[
		\langle Lu , v \rangle = 
		\int_0^1 u'' v \di x = 
		u' v |_{0}^1 -	\int_0^1 u' v' \di x = 
		u' v |_{0}^1 -u v'|_0^1 + \int_0^1 u v'' \di x 
		\]
		Thus $L^* =L$ which makes sense since the problem is asking about the operator being self adjoint.
		Now the boundary terms are:
		\[
			\left[	u' v  -u v'\right]_0^1 
			=
		u'(1) v (1) -u (1)v'(1) 
		- u'(0) v (0) +u (0)v'(0) 
		\]

	\item
		**********************************\\

\end{enumerate}

\end{Porb}



\setcounter{subsection}{6}
\subsection{The Totally Inhomogeneous Boundary Value Problem}
\setcounter{Prb}{0}
%1
\begin{Porb}
\begin{Boxed}
	Let $L = - \ddt {} x$ with boundary contions $u(0) = 0, u'(0) = u(1)$, so that $S = \{ u | Lu$ is square integral and satisfies b.c.$\}$.
\begin{enumerate}
	\item
		Find $S^*$ with
		\[
			\langle u, v \rangle = \int_0^1 \bar u v \di x
			\]
			and compare S with its cooler twin $S^*$.
	\item
		Compare the eigenvalues of $L$ and $L^*$.
		IF the two sequences are different point out the distinction, if the are the same justify the result.
	\item
		Exhibit the corresponding eigenfunctions.
	\item
		Is $\lamb=0$ an eigenvalue? Why or why not?
	\item
		Verify that $\int_0^1 \bar v_n u_m \di x =0 $ for $n \neq m$.

\end{enumerate}

\end{Boxed}

\begin{enumerate}
	\item
		I have never not started a problem with integration by parts
		\[
			\langle L u, v \rangle 
			= \int_0^1 \overline {L u} v \di x
			= \int_0^1 \overline {-\ddt u x} v \di x
	          \]
		  \[
			= 
			-u' v |_0^1
			+\int_0^1 \overline {\dd u x} \dd v x \di x
			= 
			-u' v |_0^1 + u v' |_0^1
			-\int_0^1 \overline {u} \ddt v x \di x
	          \]
		  Thus $L^* = -\ddt {} x$.
		  Our boundary terms are (with accounting for the boundary contiions on u):
		  \[
			u' v |_0^1 - u v' |_0^1
			=
			u' v |_0^1 - u v' |^1
			u'(1) v(1)  - u(1) ( v(0)+ v'(1) ) 
		\]
		Thus we see the adjoint boundary conditions are $v(1) = 0 , v(0) = - v'(1)$. Thus the problem is not self adjoint even thought the operator itself satisfies $L^* =L$.
		

	\item

		********8same problem as 4.7.2
		*****************************************\\

	\item


	\item


	\item


\end{enumerate}
\end{Porb}


%2
\begin{Porb}
\begin{Boxed}
Attack the eigenvalue problem:
	\[
		-u''(x) = \lamb u(x) , \ 0<x<1, u'(1) = \lamb u(1), \ u(0)=0
		\]
		as follows:\\
	Let $U = \begin{pmatrix} u(x) \\ u_1 \end{pmatrix}$ be a two-component vector whose first component is a twice differentiable function $u(x)$, and whose second componnent is a real number $u_1$.
		Consdier the corresponding vector space $\mathfrak{H}$ with inner product
		\[
			\langle U, V \rangle = \int_0^1 u(x) v(x) \di x + u_1v_1
			\]
			Let $S \subset \mathfrak{H}$ be the subspace
			\[
				S= \{   U: U = \begin{pmatrix} u(x) \\ u(1) \end{pmatrix} ;\ u(0) =0 \}  
				\]
				and let:
				\[
					LU = \begin{pmatrix}- u''(x) \\ u'(1) \end{pmatrix} 
					\]
			The above eigenvalue problem can now be rewritten in standard form 
			\[
				LU = \lamb U, \quad U \in S
				\]
\begin{enumerate}
	\item
		Prove or disprove that L is self adjoint.
	\item
		Prove or disprove that L is postiive-definite, ie that $\langle U , L U \rangle > 0 \forall U \neq \vec 0$. 
	\item
		Find the (transcendental) equation for the eigenvalues of L.
	\item
		Denoting these eigenvalues by $\lamb_1, \lamb_2,\lamb_3, \cdots$ exhibit the orthonrmalized eigenvectors $U_n$ associated with these eigenvalues.
\end{enumerate}
\end{Boxed}
\begin{enumerate}
	\item
		\[
			\langle - LU, V \rangle = \int_0^1 u''(x) v(x) \di x - u'(1)v_1
			=
			u'v|_0^1
			-\int_0^1 u'(x) v'(x) \di x - u'(1) v_1 + u'(1) v'(1) - u'(1) v'(1)
			\]
			\[
			=
			u'v|_0^1
			-
			uv'|_0^1
			- u'(1) \left( v_1+ v'(1)\right)
			+\int_0^1 u(x) v''(x) \di x  - u'(1) v'(1)
			\]
			We have that $u(0)=0$ and thus:
			\[
			=
			u'v|_0^1
			-
			uv'|^1
			-
			u'(1) \left( v_1- v'(1)\right)
			+
			\langle U, - L V \rangle
			\]
			Thus we can see from here that the 'extra' terms are 
			\[
			u'v|_0^1
			-
			u(1) v'(1)
			-
			u'(1) \left( v_1- v'(1)\right)
			\]
			We want to know if L is self adjoint on S, thus we have also that $v_1=v(1)$ and $v(0)=0$.
			With this we get 
			\[
			- 
			u(1) v'(1)
			+
			u'(1) \left( v'(1)\right)
			\]
			So we get 
			\[
				\langle LU , V \rangle 
				=
				\langle U , LV \rangle
				+
				\det \begin{bmatrix}
					u(1) & v(1) \\
					u'(1) & v'(1) 
				\end{bmatrix}
				\]

				Thus L is self adjoint whenever the determinat $=0$.
				Thus the most obvious space is the eigenspace where $u(1) = \lamb u'(1)$.
				
				All in all the operator is NOT self adjoint on S but it is self adjoint on $S \cap $ the eigenspace.

	\item
		Just calculating We get:
		\[
			\langle LU, U \rangle = \int_0^1 -u''(x) u(x) \di x + u(1)u'(1)
		\]
		Integration by parts 
		and with $u(0)=0$ we get:
		\[
		=
		-u' u |_0^1 +
		\int_0^1 u'(x) u'(x) \di x + u(1)u'(1)
		=
		\int_0^1 u'(x) u'(x) \di x 
		\]
		which is then just $
		\int_0^1 u'(x)^2 \di x 
		= \norm { u'}_2^2
		$

		Now is $\langle LU, U \rangle = \norm { u'}_2^2 >0$ $\forall U \neq 0$.
		Now if $u'$ is ever non zero then the abouve would be positive, so if there is a counter exapmle it would have to have $u'=0$. 
		This is doabled, we have a whole host of constant functions to chose from! 
		However we also have the condition that $u(0) =0$. With this we see that any function that satisfies:
$\langle LU, U \rangle =0$ and is in S must be zero.

All together we see that $L$ is positive semi definite.

	\item
			\[
				-\ddt {} x u(x) = \lamb u(x)
			\]
			We know that there has to be a fourier series for this function by the set up of the this problem.
			Thus we try and see if what the fourier basis elements do under this transform.
			\[
				-\ddt {} x e^{i 2 \pi k x} = \lamb  e^{i2 \pi  k x}
				= 4 \pi^2  k^2  e^{i 2 \pi  k x} 
			\]
			We arrive the eigenvalues being:
			\[
			\lamb	 = 4 \pi^2   k^2 
			\]
			Thus we see that $ \pm k $ gets maped to the same eigenvalues. Since these eigenvalues are non degenerate we realize we have to combine them to get the actual function for our problem.

	\item
			We notice that 
			\[
				\frac {e^{i 2 \pi k x} - e^{i 2 \pi (-k) x} } {2 } 
			\]
			satisfies our boundary conditions, for normality we need to divide by 2.
			Now we notice that this is in fact just $u_k(x) = \sin(2 \pi x)  $. 

			***************************CHECK NORMALIZZATION*************************\\

			So our actual eigenvetors are:
			\[
				U_n =
				\begin{pmatrix}
					u_k (x) \\
					u(1)
				\end{pmatrix}
				=
				\begin{pmatrix}
					\sin (k \pi x) \\
					u(1)
				\end{pmatrix}
				\]
				**********************\\

\end{enumerate}
\end{Porb}

%3
\begin{Porb}
\begin{Boxed}
	The eigenvalue equaion for 4.7.1 is 
	\[
		\sin \lamb^{\frac 1 2}  = \lamb^{\frac 1 2}
		\]
		Prove or disprove that an asymptotic formula for the roots is 
		\[
			\lamb^{\frac 1 2} \sim (2m + \frac 1 2) \pi - \frac{2 \log (2 m1)\pi }{ (4m+1)\pi} \pm i \log ( 4m +1) \pi
			\]
\end{Boxed}
**********************HAS HINT**************************\\
	Let $\lamb^{\frac 1 2} = \alpha +i \beta$
	so that 
	\[
		\sin \alpha \cosh \beta = \alpha 
		\quad
		\cos \alpha \sinh \beta = \beta
	\]
	asdflkfj






\end{Porb}



%4
\begin{Porb}
\begin{Boxed}
Consider the eigenvalue problem 
	\[
		Lu = \lamb u \quad L = \alpha \ddt {} x + \beta \dd {} x + \gamma
		\]
		\[
			B_1(u) =B_2(u) =0
			\]
			and its ajdoint
		\[
			L^* v = \bar \lamb v \quad 
			B^*_1(u) =B^*_2(u) =0
			\]
	with respect to the inner products $\langle v,u \rangle = \int_a^b \bar v u \di x$. One can show and you may safely assume, that the eigenvalue spectra of these two problems are complex conjugates of each other (this in factfollows from previous exercises).
\begin{enumerate}
	\item
		Prove that the solution $u(x, \lamb)$ for the problem
		\[
			L u - \lamb u  = -f (x)
			\]
			\[
			B_1(u) =B_2(u) =0
			\]
			is given by 
			\[
				u(x, \lamb) = \sum_n \frac{ \langle v_n, f\rangle } {\lamb - \lamb_n} u_n(x)
				\]
				where $u_n, v_n$ are the eigenfunctions of L and $L^*$ and have been normalized to satisfy:
				\[
					\langle v_n, u_m = \delta_{nm}
					\]
	\item
		Show that the Green's function is 
		\[
			G_\lamb ( x| \zeta) = \sum_n \frac{ u_n(x) \bar v_n (x)} {\lamb - \lamb_n} 
			\]
\end{enumerate}
\end{Boxed}
*******************************************\\
\begin{enumerate}
	\item

	\item

\end{enumerate}
\end{Porb}

%5
\begin{Porb}
\begin{Boxed}
	Obtain the o.n. set of eigen functions for the S-L problem
	\[
		Lu = - \ddt u x = \omega^2 u
		\]
		\[
			u(a) = u(b) =0
			\]
			
			by applying the complex integration technique to the Green's function $G_\omega (x , \zeta)$.
			\[
				(L^2 - \omega^2) G = - \ddt {G_\omega} x - \omega^2 G_\omega = \delta(x-\zeta) \quad a < x, \zeta < b
				\]
				\[
					G_\omega(a | \zeta) = 0  \quad G_\omega(b | \zeta) = 0 ,\quad  a < \zeta < b
					\]
\end{Boxed}
*************************************\\
\end{Porb}



\setcounter{subsection}{8}
\subsection{}
\setcounter{Prb}{0}

%1
\begin{Porb}
\begin{Boxed}
	Consider the inhomogenous Fredholm equation of the second kind:
	\[
		u(x) = \lamb \ispi \intinf K(x; \zeta) u(\zeta) \di \zeta + \phi(x)
		\]
		Here $\lamb$ is a paramter and $\phi$ is a known and given function. Also the integration kernal K, which in this problem is given to be translation invarient.
		ie. you should assume that $K(x; \zeta) = v(x-\zeta)$, where v is a given function whose Fourier transform exists.
		Solve the integral equation by finding the function $u$ in terms of what is given.
\end{Boxed}
We notice that this is a convolutuion:
	\[
		u(x) = \lamb \ispi \intinf K(x - \zeta) u(\zeta) \di \zeta + \phi(x)
		=
		\lamb \ispi K \star u + \phi(x)
		\]
		Now if we apply the 4-ier convolution thoerem we get:
		\[
		\hat u  (\omega )
		=
		\lamb \ispi \hat K (\omega ) \hat  u  (\omega )+ \hat \phi (\omega )
		\]
		Now we subtract over the $\hat u$ terms and then divide to get:
		\[
		\hat u  (\omega )
		=
		\frac  {\hat \phi (\omega )} { 1-\lamb \ispi \hat K (\omega )}
		\]
		So our actual function $u$ ends up being
		\[
		u  (x)
		=
		\I \FF \left[ 
		\frac  {\hat \phi (\omega )} { 1-\lamb \ispi \hat K (\omega )}
		\right]
		\]



\end{Porb}

%2
\begin{Porb}
\begin{Boxed}
	Look up an integral equation of the 2nd kind, either of the Volterra or of the Fredholm type. Submit it and its solution.
\end{Boxed}
	Problem:
	\[
		\phi(x)  = x^2 -x^4 + \lamb \int_{-1}^1 ( x^4 + 5 x^3 y) \phi(y) \di y
		\]
	Solution:\\
	This is calle dthe 'direct computation' approach from Stack Exchange
	\[
		\phi(x)  
		=
		x^2 -x^4 + \lamb \int_{-1}^1 ( x^4 + 5 x^3 y) \phi(y) \di y
		=
		x^2 -x^4 + \lamb \left[
			x^4 \int_{-1}^1 \phi(y) \di y
			+ 5 x^3 \int_{-1}^1  y) \phi(y) \di y
			\right]
		\]
		Now let 
		\[
			c_1 =\int_{-1}^1 \phi(y) \di y \quad 
		 c_2 = \int_{-1}^1  y) \phi(y) \di y
		\]
		and we have
		\[
			\phi(x) =
		x^2 -x^4 + \lamb \left[
			x^4 c_1	
			+ 5 x^3 c_2
			\right]
			=
		x^2 + 5 \lamb x^3 c_2 + ( \lamb c_1 - 1) x^4 
		\]

		Now if we plug this expression back into the first equaiton for $\phi(y)$ we get:
	\[
		\phi(x)  
		=
		x^2 -x^4 + 
		\lamb \int_{-1}^1 ( x^4 + 5 x^3 y) 
		\left( 
		y^2 + 5 \lamb y^3 c_2 + ( \lamb c_1 - 1) y^4 
		\right) 
		 \di y
		\]
		\[
		=
		x^2 -x^4 + 
		\lamb x^4 \int_{-1}^1 
		y^2 + 5 \lamb y^3 c_2 + ( \lamb c_1 - 1) y^4 
		 \di y
		 +
		\lamb5 x^3  \int_{-1}^1  
		y^3 + 5 \lamb y^4 c_2 + ( \lamb c_1 - 1) y^5 
		 \di y
		\]
		The region is symetric around 0 so all the odd powers of y die. Thus the only integrals we actually have to deal with are:
		\[
		=
		x^2 -x^4 + 
		\lamb x^4 \int_{-1}^1 
		y^2 + ( \lamb c_1 - 1) y^4 
		 \di y
		 +
		\lamb5 x^3  \int_{-1}^1  
		5 \lamb y^4 c_2 
		 \di y
		\]
		\[
		=
		x^2 -x^4 + 
		2 \lamb x^4 \left[ 
		\frac  1 3  + ( \lamb c_1 - 1) \frac 1 5 
		 \right]
		 +
		\lamb 10  x^3   
		 \lamb  c_2 
		\]
		
		Now we remember this is equal to $\phi(x) $ and get
		\[
		x^2 + 5 \lamb x^3 c_2 + ( \lamb c_1 - 1) x^4 
		=
		x^2 -x^4 + 
		2 \lamb x^4 \left[ 
		\frac  1 3  + ( \lamb c_1 - 1) \frac 1 5 
		 \right]
		 +
		\lamb 10  x^3   
		 \lamb  c_2 
		\]
		Now equaint coefficents of the powers of x gives:
		\begin{align}
			1 &=& 1 \\
		5 \lamb  c_2  &=& \lamb^2 10  c_2  \\
			\lamb c_1 - 1 &=& 
		2 \lamb \left[  \frac  1 3  + ( \lamb c_1 - 1) \frac 1 5  \right] -1 
		\end{align}
		So from this we see that 
		\[
		  c_2  = 0
		\]
		Now with some effort we go after $c_1$:
		\[
			\lamb c_1 - 1=
 	\frac 2 5   \lamb^2 c_1 
 	+ 2 \lamb \left[  \frac  1 3  -\frac 1 5  \right] -1
		\]
		\[
			c_1 
			( - \frac 2 5   \lamb^2 
			+ \lamb )  
			= 
 	+ 2 \lamb \left[  \frac  1 3  -\frac 1 5  \right] 
		\]

		\[
			c_1 
			= 
			\frac{
			  \frac  4 {15}  
	}
	{  - \frac 2 5   \lamb	+  1  
			}
			= 
			  \frac  4 {3}  
			  \frac 1 
	{   5  - 2   \lamb	
			}
		\]
		So big deal we calculated some coefficents, what about $\phi$?

		\[
			\phi(x) = 
		x^2 + 5 \lamb x^3 c_2 + ( \lamb c_1 - 1) x^4 
		=
		x^2 + \left(    \frac  4 {3}  
 \frac \lamb 	{   5  - 2   \lamb }
		- 1 \right) x^4
		\]
		% https://math.stackexchange.com/questions/349133/solution-of-an-integral-equation-phix-int1-0-xtxt-phit-dt-x-0/393672#393672
\end{Porb}


\setcounter{subsection}{10}
\subsection{}
\setcounter{Prb}{0}


%1
\begin{Porb}
\begin{Boxed}
	Over the interval $-\infty < x< \infty$ consider
	\[
		\ddt G x + \lamb G = -\delta ( x - \zeta)
		\]
	\[
		\ddt u x + \lamb u = -f(x)
		\quad 
		\text{and}
		\quad 
		\ddt \phi  x + \lamb \phi = 0 
		\]
		We are looking for solutions in $L^2$ assume that f is in $L^2$.
\begin{enumerate}
	\item
		Show that there are tow candidates for G, namely
		\[
			G = G^{out} (x| \zeta ; \lamb )=
			\frac i {2 \sqrt \lamb } \exp ( i \sqrt \lamb |x-\zeta|)
			\]
			and 
		\[
			G^{in} (x| \zeta ; \lamb ) =
			- \frac i {2 \sqrt \lamb } \exp ( i \sqrt \lamb |x-\zeta|)
			\]
	\item
		Given the fact that $\sqrt \lamb = \alpha + i \beta$ with $\beta>0$, point out why only one of them is square- integrable.
	\item
		Consider the contour integral $\oint G \di \lamb$ over a large circle of radius R. Demonstrate that 
		\[
			\lim_{R \rw \infty} \frac 1 {2 \pi i} \oint G \di \lamb = - \delta (x - \zeta) 
			\]
	\item
		Next deform the contour until it fits snugly around the branch cut of $\sqrt \lamb$, and show that 
		\[
			\delta(x-\zeta) = \int_0^\infty \cdots \di \lamb
			\]
			and then show that the above can be rewriteten as 
		\[
			\delta(x-\zeta) =  \frac 1 {2\pi} \intinf e^{i \omega ( x- \zeta) } \di \omega 
			\]
			for $x< \zeta,$ and $ \zeta < x$.
	\item
		Express $u(x)$ as a Fourier integral in terms of f.
	\item
		Express $G$ in the same way, ie obtain a bilinear expansion for G.
\end{enumerate}
\end{Boxed}

\begin{enumerate}
	\item
		***************************************\\
	\item
	\item
	\item
	\item
	\item
\end{enumerate}
\end{Porb}


%2
\begin{Porb}
\begin{Boxed}
	
	Again consider
	\[
		\ddt G x + \lamb G = - \delta( x-\zeta)
		\]
		\[
			\ddt u x + \lamb u = -f(x) , \text{ and } \ddt \phi x + \lamb \phi =0
			\]
over the interval $-\infty < x < \infty$, but leav the boundary conditons as yet to be specified.
\begin{enumerate}
	\item
		Express $u(x)$ as a Fourier integral in terms of f.
	\item
		Express $G(x, \zeta, \lamb) $ in the same way, ie obtain a bilinear expansion for G.
	\item
		How, do you think, should on incorporate boundary conditions into these expressions?
\end{enumerate}
\end{Boxed}

\begin{enumerate}
	\item
	\item
	\item
\end{enumerate}
\end{Porb}





%ch5
\section{Special Function Theory}
\subsection{The Helmholtz Equation}
\setcounter{Prb}{0}

%1
\begin{Porb}
\begin{Boxed}
	Evaluate the integral:
	\[
		\int_C e^{i \rho \cos \alpha + i \nu \alpha} \di \alpha
		\]
		along the curve C (in the complex $\alpha$ plane below) in terms of the two kinds of Hankel functions.

		************Include a Tixz picutre?**********************\\
%https://tex.stackexchange.com/questions/307100/improvement-of-a-complex-contour
\end{Boxed}

*****************************\\
\end{Porb}

%2
\begin{Porb}
\begin{Boxed}
	In the complex $\beta$ plane, determine those semi-infinte strip regions where the line integral
	\[
		\int_C e^{i \rho \cos \beta- i \nu \beta} \di \beta
		\]
		converges if the integration limits of the path C are extended to infinity in each of a pair of such strips.
\end{Boxed}
*****************\\
\end{Porb}

%3
\begin{Porb}
\begin{Boxed}
	By slightly deforming the integration path prove or disprove that the integral
	In the complex $\beta$ plane, determine those semi-infinte strip regions where the line integral
	\[
		\intinf e^{i \rho \cos \beta- i \nu \beta} \di \beta
		\]
		can be expressed in terms of a Hankel function, what kind and which order?
\end{Boxed}
****************************\\
\end{Porb}

%4
\begin{Porb}
\begin{Boxed}
	Apply 
	\[
	t = \zeta \cosh \tau, \ z= \zeta \sinh \tau , \quad 0 < \zeta < \infty, - \infty < \tau < \infty	
	\]
	to the wave equation 
	\[
		- \pp {^2 \psi} {t^2} + \pp {^2 \psi} {z^2}  - k^2 \psi =0
		\]
		in order to obtain the wave equation relative to the coordinates $\zeta, \tau$. TO do this take advantage of the fact that letting 
		\[
			r = \zeta, \quad \theta = i \tau
			\]
			and 
			\[
				x = t, \quad y = iz 
			\]
			yields the hyperbolic transformation of the wave euqation.

\begin{enumerate}
	\item
		Write down the wave equation in terms of the "pseudo" polar coordinates $\zeta, \tau$.

	\item
		Consider a solution whihc is a "psuedo" rotation eigenfunction $\psi_\omega$:
		\[
			\pp {\psi_\omega} \tau = -i \omega \psi_\omega
			\]
			and determine the differential euqation:
			\[
				\left[
					\alpha (\zeta) \ddt {} \zeta +
					\beta(\zeta) \dd {} \zeta + \gamma (\zeta)
					\right] \psi =0 
				\]
				it satisfies.
	\item
		Verify that the translation (in the t,z plane) eigenfunction
		\[
			\psi = e^{-i(k_0 t - k_z z)}
			\]
			is a solution ot the wave equation, whenever the two k's satisfy $k_0^2-k_z^2 =k^2$.
			Then using $k_0 = k \cosh \alpha, k_z = k \sinh \alpha$ and $t = \zeta \cosh \tau, z = \zeta \sinh \tau$, and the hiyperbolic angle additon formula, rewrite the phase and hence the wave funciton $\psi$ in terms of $\zeta, \tau$.
	\item
		Construct a superpositoin (as an integral over $\alpha$) of waves $\psi$ whihc is a "pseudo" ortation eigenfunction ie satsifies
		\[
			\pp {\psi_\omega} \tau = - i \omega \psi_\omega
			\]
			where $\psi_\omega$ is that superposition.
	\item
		Exhibit two indendpent soutions $\psi_\omega$ to the wave equation corresponding to two different integration contours. WHat are they? If your solutions are porooprtional to Hankel funcitons, specifiy what kind and identify their order.
\end{enumerate}

\begin{enumerate}
	\item
	\item
	\item
	\item
	\item
\end{enumerate}

\end{Boxed}
\end{Porb}

\subsection{Properties of Hankel and Bessel Functions}
\setcounter{Prb}{0}

\begin{Porb}
\begin{Boxed}
	Show that:
	\[
		H^{(1)}_{-n} (\rho)  = (-1)^n H^{(1)}_{n} (\rho) 
		\]
		\[
		H^{(2)}_{-n} (\rho)  = (-1)^n H^{(2)}_{n} (\rho) 
		\]
		\[
		N_{-n} (\rho) 	     = (-1)^n N_{n} (\rho) 
		\]
%	\begin{align*}
%		H^{(1)}_{-n} (\rho) & = (-1)^n H^{(1)}_{n} (\rho) \\
%		H^{(2)}_{-n} (\rho) & = (-1)^n H^{(2)}_{n} (\rho) \\
%		N_{-n} (\rho) 	    & = (-1)^n N_{n} (\rho) 
%	\end{align*}
\end{Boxed}
	\[
		H^{(1)}_{-n} (\rho)  =
		c_1 \int_{\alpha_1}^{\alpha_2} e^{i \rho \cos \alpha + i (-n) \alpha} \di \alpha
		\]
		Now we introduce $\alpha = \pi - \bar \alpha, \rw \alpha -\pi = - \bar \alpha$ which shifts the bounds of the integrals by $\theta$ but this does not matter as shown on page 302 under the no angular dependence property.
		With thi ssub we see
		\[
			c_1 \int^{\alpha_1+\pi}_{\alpha_2+\pi} e^{i \rho \cos ( \pi - \bar\alpha)  + i (-n) (\pi -\bar\alpha)} \di \alpha
			=
		c_1 \int^{\alpha_1+\pi}_{\alpha_2+\pi} e^{i \rho \cos ( \pi - \bar\alpha)  + i (-n) (\pi -\bar\alpha)} (-1) \di \bar \alpha
		\]

		Since $\cos ( \pi -x) = \cos (x)$ (think of the unit circle, or idk trig identities or something) and we swap the limits and switch the sign of the integral.
		\[
			=
		c_1(-1)^n \int_{\alpha_1+\pi}^{\alpha_2+\pi} e^{i \rho \cos ( \bar\alpha)  + i (-n) (\pi -\bar\alpha)} \di \bar \alpha
			=
		c_1(-1)^n \int_{\alpha_1+\pi}^{\alpha_2+\pi} e^{i \rho \cos ( \bar\alpha)  + i n\bar\alpha} \di \bar \alpha
		=
		(-1)^n H^{(1)}_{n} (\rho) 
		\]

		Now for the 2nd identity we remmeber that 
		\[
			J_\nu = \frac 1 2 \left[ H_\nu^1 + H_\nu^2\right] \Rw
			2J_\nu - H_\nu^1 =  H_\nu^2
		\]
		and thus just from the last 2 identities we have:

		\[
			2J_{-n} - H_{-n}^1 =
			(-1)^n \left( 2J_{n} - H_{n}^1\right) =
			(-1)^n H_\nu^2
		\]

		Similarly we have 
		\[
			N_\nu = \frac 1 {2i} \left[ H_\nu^1 - H_\nu^2\right] 
		\]
		and thus:
		\[
			N_{-n} = (-1)^n \frac 1 {2i} \left[ H_n^1 - H_n^2\right] 
			=
			(-1)^n N_{-n}
		\]
		as was deeply desired.
\end{Porb}

\subsection{Applications of Hankel and Bessel Functions}
\setcounter{Prb}{0}
%1
\begin{Porb}
\begin{Boxed}
	The transvers amplitude of an axially symmetric wave propagating in a cylindrical pipe of radius a is determined by the following eigenvalue problem:
	\[
		- \dd {} r r \dd u r = k^2 r u \quad 0 \leq r \leq a
		\]
		\[
			u(0) = \text{finite} \quad u(a) =0
		\]
		The eigenfunctions are $u_m(r) = J_0(rk_m)$ where the boundary condiiton $J_0 (a k_m) =0$ determines the eigenvalues $K_m^2, m \in \NN$.

\begin{enumerate}
	\item
		Show that $\{ J_0 ( r k_m) \}$ is an orthogonal set of eigenfunctions on $(0,a)$.
	\item
		Using the problem 3.3.5 find the squared norm of $J_0(rk_m)$.
	\item
		Exhibit the set of orthonormalized eigenfunctions.
	\item
		Find the Green's function for the above boundary value problem.
\end{enumerate}
\end{Boxed}
\begin{enumerate}
	\item
	
		Let $u= J_0( k_m r)$ then with $x= k_m r$ then we notice that the Bessel equation gives: (we have 1 less factor of r than the standard Bessel form)
	\[
		\dd {} x x \dd u x  = 
		x \ddt {}x +  \dd u x  = 
		x u
		\]
		Now we see that $\dd {} x = \dd {} r \dd r x = \frac 1 {k_m}$. 
		Thus our equaiton becomes:
		\[
			\frac{	k_m} { k_m^2} \ddt {}r +  \frac 1 {k_m} \dd u r  = 
		k_m r u
		\]
		Which moving around some constnats yields:
		\[
			 \ddt {}r +  \dd u r  = 
		k_m^2 r u
		\]
		Which is the ODE we have. 

		By theorem 1 on page 166 we see that these eigenvalues are nondegenertae and that they are orthogonal. (Have one endpoint set to zero of a S-L system).

	\item
		Problem 3.3.5 tells us 
		\[
			\int_0^a J_0( r k_m)^2 \di r =
						J'_0(a k_m) \frac{d J_0(a \lamb)} {d \lamb} |_{\lamb=k_m} 
						- J_0(ak_m) 
						\frac d {d \lamb} 
						J'_0(a\lamb) |_{\lamb=k_m} 
			\]
			By construction we have that $J_0(ak_m)=0$, thus we only need to figure out $J'_0(x)$.
			\[
					J'_0(a k_m) \frac{d J_0(a \lamb)} {d \lamb} |_{\lamb=k_m} 
					- J_0(ak_m) 
					\frac d {d \lamb} 
					J'_0(a\lamb) |_{\lamb=k_m} 
					=
					a J'_0(a k_m)^2  
			\]
			******************DERIVE IDENTITY LATER?**********\\

	\item
		Thus to normalize the eigenfucntions we would simply normalize by the norm above:
		\[
			 \frac { J_0( ak_m) } {\sqrt a J'_0( a k_m)}   = ????
			\]

	\item
			******************Greens FUnction?**********\\
\end{enumerate}
\end{Porb}

%2
\begin{Porb}
\begin{Boxed}
	On a circula disc of radius a find an orthonormal set of eigenfunctions for the system defined by the eigenvalue problem
	\[
		- \nabla^2 \psi = k^2 \psi
		\]
		\[
			\pp \psi r ( r=a, \theta) = 0 
		\]
		\[
			\psi ( r=0 , 0 ) = \text{finite}, \ 0 \leq \theta \leq 2\pi
			\]
			Here $\nabla^2 =  \frac 1 r \pp {} r r \pp {} r + \frac 1 {r^2} \pp {^2} {\theta^2} $ and exhibit these eigenfunctions in their optimally simple form, ie without refering to any derivatives.
\end{Boxed}
************************************\\
\end{Porb}

%3
\begin{Porb}
\begin{Boxed}
	Consider a wave disturbance $\psi$ which is governed by the wave equatioin.
	\[
		\left[
			\pp {^2} {r^2} +\frac 1 r \pp {} r r \pp {} r + \frac 1 {r^2} \pp {^2} {\theta^2} 
			+\pp {^2} {z^2} 
			\right]
			\psi 
			= 
			\frac 1 {c^2} \pp {^2 \psi} {t^2}
		\]
		Let this wave propogate inside an infintely long cylinder; in other words, it satisfies
		\[
			\pp \psi z = i k_z \psi
			\]
			where $k_z$ is some real number not equal to zero. Assume that the boundary coditions satisfied by $\psi$ is 
		\[
			\psi(r=a) =0, 	\psi ( r=0 ) = \text{finite}, \quad \text{ a = radius of cylinder}
			\]

\begin{enumerate}
	\item
		Find the 'cut off' frequencey ie the frequency belwo which no propogation in the infite cylinder is possible.
	\item
		Note that this frequency dpeends on the angular integer m and the radial integr j. For fixed j give an argument which supports the result that smallr m means msaller critical frequency
	\item What is the smallest critical frequency in terms of a and c to an accuracy of $2 \%$ or better?
\end{enumerate}
\end{Boxed}
*************************************\\
\begin{enumerate}
	\item
	\item
	\item
\end{enumerate}
\end{Porb}

%4
\begin{Porb}
\begin{Boxed}
	Consider the sector $S=\{(r,\theta)| 0 \leq r \leq a, 0 \leq \theta \leq \alpha \}$.
\begin{enumerate}
	\item
		Exhibit the set of those normalized eigenfunctions for this sector whic satisfy 
		\[
			(\nabla^2 + k^2) \psi =0, \quad \psi =0 \text{ on } \partial S
			\]
	\item
		Compare the set of normal modes of a circular drum with the set of nromal modes in part a) when $\alpha = 2 \pi$.
\end{enumerate}
\end{Boxed}

\begin{enumerate}
	\item
		**********************************************\\
	\item
\end{enumerate}
\end{Porb}

%5
\begin{Porb}
\begin{Boxed}
Consdier 
\begin{enumerate}
	\item a circular membarne of radius a.
	\item a square membrane
	\item a rectangular membrane which is twice as long as it is wide.
\end{enumerate}

Assume the two membranes
\begin{enumerate}
	\item
		have the same area
	\item
		obey the same wave equation $\nabla^2 \psi = \frac 1 {c^2} \pp {^2} {t^2}$
	\item Have the same boundary conditons, ie $\psi =0$ on the boundary.
\end{enumerate}


\begin{enumerate}[A)]
	\item 

		\begin{enumerate}[i)]
		\item
			the 3 lowest frequencies for each of the two membranes
		\item
			all the concomitant normal modes
	\end{enumerate}

	\item
		FOr each of the normal modes of the circular mebrane draw a picutre of the nodes, ie the locus of points where $\psi =0$ Label each of the pictures
	\item
		Do the same for the other membrane (watch out for degeneracies!)
\end{enumerate}
\end{Boxed}

		**********************************************\\
\end{Porb}


\subsection{More Properties of Hankel and Bessel Functions}
\setcounter{Prb}{0}

\begin{Porb}
\begin{Boxed}
	Express $J_n(x_1+x_2)$ as a sum of products of Bessel functions of $x_1$ and $x_2$ respectively.
\end{Boxed}

\end{Porb}

\subsection{The Method of Steepest Descent and Stationary Phase}
\setcounter{Prb}{0}
\begin{Porb}
\begin{Boxed}

\begin{enumerate}
	\item
		Using the method of steepest descent find anasymptotic expression for $H_\nu^2$ and for $J_\nu$ where $\nu << \rho$.
	\item
		THe gamma function $\Gamma$ for which $Re \omega >-1$ is represented by 
		\[
			\Gamma ( \omega +1) = \int_0^\infty e^{-\tau} \tau^\omega \di \tau
			\]
			Using the steepest descent approach, find and asymptotic expression for $\Gamma ( \omega+1)$ when $Re \omega >> 1$. Why doesn't it work? Try again by susbstituting $\omega z$ for $\tau$ and obtaining:
			\[
				\Gamma ( \omega +1) = \omega^{\omega+1} \int_0^\infty e^{-\omega z } z ^\omega \di z
				=
				\omega^{\omega+1} \int_0^\infty e^{\omega ( \ln z - z) }\di z
				\]
\end{enumerate}
\end{Boxed}

\begin{enumerate}
	\item
		*********************************\\
	\item
\end{enumerate}
\end{Porb}





%ch6
\section{Partial Differential Equations}
\setcounter{subsection}{1}
\subsection{System of Partial Differntial Equations: How to solve Maxwell's equations using Linear Algebra}
\setcounter{Prb}{0}
%1
\begin{Porb}
\begin{Boxed}
	Consider the current-charge density to an isolated moving charge:
	\[
		\vec J ( x,y,z,t) =
		q \intinf \dd {\vec X (\tau)} \tau \delta(x-X(\tau)) \delta ( z-Z(\tau)) \delta(t-T(\tau)) \di \tau
		\]
	\[
		 \rho ( x,y,z,t) =
		q \intinf \dd {\vec T (\tau)} \tau \delta(x-X(\tau)) \delta ( z-Z(\tau)) \delta(t-T(\tau)) \di \tau
		\]

\begin{enumerate}
	\item
		Show that this current-charge density satisifies
		\[
			\nabla \cdot \vec J + \pp \rho t =0
			\]

	\item
		By taking advantge of the fact $\dd {T(\tau)} \tau >0$, evaluate the $\tau$-integrals, and obtain explicit expressions for the components $\vec J$ and $\rho$.

		Answer:
		\[
		 \rho ( x,y,z,t) =
		q \delta(x-X(\tau)) \delta ( y-Y(\tau)) \delta ( z-Z(\tau)) \delta(t-T(\tau)) 
		\]
	\[
		\vec J ( x,y,z,t) =
		q \dd {\vec X } t \delta(x-X(\tau)) \delta ( y-Y(\tau)) \delta ( z-Z(\tau)) \delta(t-T(\tau)) \di \tau
		\]
		where $\vec X(t) = \vec X(\tau)$ evaluated  at $\tau$ as determined by $\delta ( t- T(\tau))$.
\end{enumerate}

		\[
			\nabla \cdot \vec J + \pp \rho t =0
			\]
\end{Boxed}
	\begin{enumerate}
		\item
		\[
			\nabla \cdot \vec J + \pp \rho t =
			\]
		\[
			\nabla \cdot \vec J= 
			\sum_{i} \pp {} {x_i} \vec J 
			=
		q \intinf \dd {\vec X (\tau)} \tau \sum_{i} \pp {} {x_i}
		\delta(x-X(\tau)) \delta ( z-Z(\tau)) \delta(t-T(\tau)) \di \tau
			\]

			\[
			=
		q \intinf \dd {\vec X (\tau)} \tau
		\sum_{i} \delta' ( x_i - X_i (\tau) ) \prod_{j \neq i} \delta ( x_i - X_i (\tau) )
		\delta(t-T(\tau)) \di \tau
			\]


	\[
\pp \rho t
=
	\pp {} t 
		q \intinf \dd {\vec T (\tau)} \tau \delta(x-X(\tau)) \delta ( z-Z(\tau)) \delta(t-T(\tau)) \di \tau
	\]

	\[
=
			q \intinf \dd {\vec T (\tau)} \tau \pp {} t \delta(x-X(\tau)) \delta ( z-Z(\tau)) \delta(t-T(\tau)) \di \tau
	\]

	\[
=
		q \intinf \dd {\vec T (\tau)} \tau
		\sum_{i} - \delta' ( x_i - X_i (\tau) )\dd {X_i(\tau)} t  \prod_{j \neq i} \delta ( x_i - X_i (\tau) )
		\di \tau
	\]




			*******************\\
		\item Since $\dd T \tau >0$ we see that $T$ is an injective function of $\tau$ and thus there is only one specific value where $t= T(\tau)$ (if one exists at all). 



\end{enumerate}
\end{Porb}


%2
\begin{Porb}
\begin{Boxed}
	zt
\end{Boxed}
\end{Porb}


%3
\begin{Porb}
\begin{Boxed}
	zt
\end{Boxed}
\end{Porb}


%4
\begin{Porb}
\begin{Boxed}
	zt
\end{Boxed}
\end{Porb}


%5
\begin{Porb}
\begin{Boxed}
	zt
\end{Boxed}
\end{Porb}


%6
\begin{Porb}
\begin{Boxed}
	zt
\end{Boxed}
\end{Porb}


%7
\begin{Porb}
\begin{Boxed}
	zt
\end{Boxed}
\end{Porb}


%8
\begin{Porb}
\begin{Boxed}
	zt
\end{Boxed}
\end{Porb}

%9
\begin{Porb}
\begin{Boxed}
	zt
\end{Boxed}


\end{Porb} 
\end{document} 







