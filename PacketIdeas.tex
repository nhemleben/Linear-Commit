%Packet ideas and such

\documentclass[12pt]{article} 
\usepackage{esvect, graphicx,relsize, verbatim, amssymb, amsmath, amsthm, mathabx,dcolumn,mathrsfs, dsfont, enumerate, soul, titlesec}
\newcolumntype{2}{D{.}{}{2.0}}
\textwidth = 7 in
\textheight = 9.5 in
\oddsidemargin = -0.3 in
\evensidemargin = -0.3 in
\topmargin = -0.4 in
\headheight = 0.0 in
\headsep = 0.0 in
\parskip = 0.2in
\parindent = 0.0in
\titlespacing*{\section}
{0pt}{0ex plus .1ex minus .2ex}{.1ex plus .1ex}
\titlespacing*{\subsection}
{0pt}{.5ex plus .1ex minus .2ex}{.3ex plus .2ex}
\DeclareMathOperator{\Gal}{Gal}	
\DeclareMathOperator{\gal}{Gal}	
\DeclareMathOperator{\im}{Image}
\DeclareMathOperator{\Ann}{Ann}	
\DeclareMathOperator{\spn}{span}	
\DeclareMathOperator{\sgn}{sgn}	
\DeclareMathOperator{\sinc}{sinc}	
\newcommand{\CC}{\mathbb{C}}	
\newcommand{\GG}{\mathbb{G}}	
\newcommand{\LL}{\mathbb{L}}	
\newcommand{\NN}{\mathbb{N}}	
\newcommand{\PP}{\mathbb{P}}	
\newcommand{\QQ}{\mathbb{Q}}	
\newcommand{\RR}{\mathbb{R}}	
\newcommand{\ZZ}{\mathbb{Z}}
\newcommand{\FF}{\mathfrak{F}}	
\newcommand{\one}{\mathds{1}}
\newcommand{\eps}{\epsilon}
\newcommand{\floor}[1]{\lfloor #1 \rfloor}
\newcommand{\s}[1]{\sqrt{#1}}
\newcommand{\norm}[1]{\left\lVert#1\right\rVert}
\newcommand{\I}[1]{#1^{-1}}
\newcommand{\Rw}{\Rightarrow}
\newcommand{\Lw}{\Leftarrow}
\newcommand{\lw}{\leftarrow}
\newcommand{\rw}{\rightarrow}
\newcommand{\p}{\partial}
\newcommand{\var}{\mathrm{Var}}
\newcommand{\sumi}{\sum_{i=0}}
\newcommand{\sumo}{\sum_{i=1}}
\newcommand{\num}[1]{{\large \bf  #1)}}
\newcommand{\del}{\Delta}
\newcommand{\lamb}{\lambda}	
\newcommand{\al}{\alpha}	
\newcommand{\nab}{\nabla}	
\newcommand{\pp}[2]{\frac{\p #1}{\p #2}}
\newcommand{\di}{\mathrm{d}}
\newcommand{\dd}[2]{\frac{\di #1}{\di #2}}
\newcommand{\ddt}[2]{\frac{\di^2 #1}{\di #2 ^2}}
\newcommand{\intinf}{\int\limits_{-\infty}^\infty}	
\newcommand{\suminf}[1]{\sum_{#1 = -\infty}^\infty}	
\newcommand{\ispi}{\frac 1 {\sqrt{2 \pi}} }
\newtheorem{thm}{Theorem}
\theoremstyle{definition}
\newtheorem{defi}{Definition}[section]
\newtheorem{lem}{Lemma}[section]

%%%%% End of preamble %%%%%
\begin{document}
\title{Linear Math In Infinte Dimensions}
{\Large Nicholas Hemleben} \hfill
{\large packets? }
\hfill  \today


Properties of the Henkel and Bessel Functions 
\begin{enumerate}
	\item Linear superopoition of plane waves

	\item
		They are Integration contours in the complex plane
	\item
		The integral rep. of the two Hankel functions do not depend on any real changes in the integration limits.

	\item
		The cylinder harmonics are eigenfunctions of the rotational gneerator $L_\theta = \frac 1 i \pp {} \theta$.

	\item
		They satisfy the Helmholtz's euqation, whihc in polar coordinates beocome bessels equation 

	\item
		The domain of a cylinder hrarmonic is the r and $\theta$ coordinate transverse cross section of a cylinder. A cylinder harmonic itself is the r and $\theta$ dependent part of a cylinder wave.

	\item
		The two Hnakel functions are distinguished by the direction and shape of their contour in the complex plane.

	\item
		The bessel function is the average of the two types of hankel function

	\item
		Neuman is the 'sin' average of the two hankel funcitosn (the arugnemnt of the hankels are not negated though)

	\item
		Analog to exponential functions

	\item
		$J_\nu(\rho)$ is real when $\nu$ is real and $J_0(0)=1$. Moreover $J_\nu$ satisifies the reflection principal from complex analysis.
	\item
		For integer $\nu= m$ we have 
		\[
			J_m ( \rho ) = \frac 1 \pi \int_0^\pi \cos ( \rho \sin \beta - m \beta ) \di \beta
			\]

	\item
		There is a very ugly power expansion of $J_\nu$

	\item
		\[
			H_\nu^{1} = \frac { e^{- i \nu \pi } J_\nu - J_{-\nu}} {-i \sin \pi \nu}
			\]

	\item
		\[
			\overline { H_\nu^2} = H_\nu^1
			\]

	\item
		\[
			Z_{\nu+1} + Z_{\nu-1}  = \frac { 2 \nu} \rho Z_\nu
			\]
		\[
			Z_{\nu+1} - Z_{\nu-1}  = -2 \dd {} \rho Z_\nu
			\]

	\item
		\[
			L_+Z_\nu e^{i\nu \theta} = 
			- Z_{\nu+1} e^{i(\nu+1) \theta }
			\]
		\[
			L_-Z_\nu e^{i\nu \theta} = 
			- Z_{\nu-1} e^{i(\nu-1) \theta }
			\]

	\item
		Plane wave in the Euclidian plane can be represented as a linear combination of cylinder harmonics of integral order.


	\item

	\item
	\item
	\item
	\item
\end{enumerate}











	Since part (ii) gives a guess we might as well say $\lambda$ cooresponds with $\bar \lamb$ for eigenvalues between L and $L^*$.

	We can see this with u and v as in the satement of the problem
	\[
		\langle  L u , v \rangle = \bar \lambda \langle u, v \rangle
		\]
		from the definition of eigenvalue/function.
	\[
	= 	\langle   u , L^* v \rangle = \lambda' \langle u, v \rangle
		\]
		Thus we see that if $\langle u , v \rangle \neq 0$ that $\bar \lamb = \lamb '$.


		*********8MAKE NOTE FOR LATER POST*****Problem is Done though*************\\
		In fact it is clear from the above that $\bar \lamb$ being an eigenvalue of the adjoint is implied by $\exists v, \langle u, v \rangle \neq 0$.
		Thus if an operator is self adjoint we see that $\lamb = \bar \lamb$ and the eigenvalues must all be real valued.








%Basicly this is just a slight modificaiton of the proof found on page 168 for theorem 1 of this section.
%		\begin{enumerate}[1)]
%			\item
%				\[
%					Lu_n =  - \frac{ d^2u_n} { dx^2} + x^2 u_n(x)  = \lambda_n u_n(x)
%					\]
%					Now multiply both sides by $u_m$
%				\[
%				\bar u_m Lu_n =  -\bar u_m \frac{ d^2u_n} { dx^2} + \bar u_m x^2 u_n(x)  = \lambda_n \bar u_m u_n(x)
%					\]
%					Now we multiply by $-1$ and then add the analgous equation with $u_m$ to get:
%					\[
%						u_n L \bar u_m - \bar u_m Lu_n  
%						= 
%						\dd {} x \left[  \bar u_m u_n '  - u_n \bar u_m ' \right] 
%						\]
%				Now the right hand side is just the difference of the eigenvalues times the functions:
%				\[
%					= (\lambda_n - \lamb_m ) \bar u_m u_n
%					\]
%		\item
%			Integrating both sides we get: 
%			\[
%			 (\lambda_n - \lamb_m ) \intinf \bar u_m u_n \di x
%			 = 
%			 \intinf  u_n L \bar u_m - \bar u_m Lu_n   \di x
%			\]
%			Which we can see the right hand side is just
%			\[
%				= 
%				\left[  \bar u_m u_n '  - u_n \bar u_m ' \right]_{\pm \infty} 
%			\]
%					thanks to the work above (aka Lagrange's Identity). 
%
%			\item
%				We wish to show that 
%				$\left[  \bar u_m u_n '  - u_n \bar u_m ' \right]_{\pm \infty} =0$
%				This is straight forward as $u_m \rw 0$ as $ x \rw \pm \infty$.
%		\end{enumerate}







			We consider:
	\[
		u''+ b(x) u'+ c(x) u=0
	\]
	*********************PACKT******************\\
			Let u be some solution, let us try $v(x) = F(x) u(x)$ 
			then:
			\[
				v' = F' u + F u', 
				v'' = F'' u + 2F'u' + F u''
				\]

			Now we plug this into $v''+Qv$ and find:
			\[
				 F'' u + 2F'u' + F u''
				 +
				 Fu
				 =
				 F'' u + 2F'u' + F (u'' +Qu)
				\]
				
				\[
				 =
				 F \left[ 
				  u'' +2F'/Fu' +(F''/F +Q)u 
				  \right]
				\]
				We know that $u'' = -bu'-cu$
				\[
				 =
				 F \left[ 
				  (2F'/F-b)u' +(F''/F +Q-c)u 
				  \right]
				\]
				If we let $Q = c- F''/F$ then all we have to do is solve $2F'/F =b$.
				This leads to
				\[
					2 F'/F -b=0 \Rw 2 \int^x F'/F = \int^x b
					\Rw 2 \ln F = \int^x b
				\]

					\[
						\Rw F = \exp \{ \frac 1 2 \int^x b \}
					\]

			Thus our subsitition ends up being: $v(x) = \exp \{ \frac 1 2 \int^x b \} u(x)$. 
			Note that $F' = \frac {b} 2 F, \ F'' = \frac {b'+ b^2/2} 2 F$
 and our equation gets:
			\[
				Q= c- F''/F 
				=
				 c- F''/F
				 =
				c- \frac {b'+ b^2/2} 2
			\]
			All togther we have:
	\[
		u \rw v = \exp \{ \frac 1 2 \int^x b \} u(x), \quad
		\]
		\[
		u''+ b(x) u'+ c(x) u=0
		\ \rw \
		v''+Q(x) v=0, \ Q(x) = 	c- \frac {b'+ b^2/2} 2
	\]




















\end{document}
